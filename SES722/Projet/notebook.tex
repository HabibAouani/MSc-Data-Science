
% Default to the notebook output style

    


% Inherit from the specified cell style.




    
\documentclass[11pt]{article}

    
    
    \usepackage[T1]{fontenc}
    % Nicer default font (+ math font) than Computer Modern for most use cases
    \usepackage{mathpazo}

    % Basic figure setup, for now with no caption control since it's done
    % automatically by Pandoc (which extracts ![](path) syntax from Markdown).
    \usepackage{graphicx}
    % We will generate all images so they have a width \maxwidth. This means
    % that they will get their normal width if they fit onto the page, but
    % are scaled down if they would overflow the margins.
    \makeatletter
    \def\maxwidth{\ifdim\Gin@nat@width>\linewidth\linewidth
    \else\Gin@nat@width\fi}
    \makeatother
    \let\Oldincludegraphics\includegraphics
    % Set max figure width to be 80% of text width, for now hardcoded.
    \renewcommand{\includegraphics}[1]{\Oldincludegraphics[width=.8\maxwidth]{#1}}
    % Ensure that by default, figures have no caption (until we provide a
    % proper Figure object with a Caption API and a way to capture that
    % in the conversion process - todo).
    \usepackage{caption}
    \DeclareCaptionLabelFormat{nolabel}{}
    \captionsetup{labelformat=nolabel}

    \usepackage{adjustbox} % Used to constrain images to a maximum size 
    \usepackage{xcolor} % Allow colors to be defined
    \usepackage{enumerate} % Needed for markdown enumerations to work
    \usepackage{geometry} % Used to adjust the document margins
    \usepackage{amsmath} % Equations
    \usepackage{amssymb} % Equations
    \usepackage{textcomp} % defines textquotesingle
    % Hack from http://tex.stackexchange.com/a/47451/13684:
    \AtBeginDocument{%
        \def\PYZsq{\textquotesingle}% Upright quotes in Pygmentized code
    }
    \usepackage{upquote} % Upright quotes for verbatim code
    \usepackage{eurosym} % defines \euro
    \usepackage[mathletters]{ucs} % Extended unicode (utf-8) support
    \usepackage[utf8x]{inputenc} % Allow utf-8 characters in the tex document
    \usepackage{fancyvrb} % verbatim replacement that allows latex
    \usepackage{grffile} % extends the file name processing of package graphics 
                         % to support a larger range 
    % The hyperref package gives us a pdf with properly built
    % internal navigation ('pdf bookmarks' for the table of contents,
    % internal cross-reference links, web links for URLs, etc.)
    \usepackage{hyperref}
    \usepackage{longtable} % longtable support required by pandoc >1.10
    \usepackage{booktabs}  % table support for pandoc > 1.12.2
    \usepackage[inline]{enumitem} % IRkernel/repr support (it uses the enumerate* environment)
    \usepackage[normalem]{ulem} % ulem is needed to support strikethroughs (\sout)
                                % normalem makes italics be italics, not underlines
    

    
    
    % Colors for the hyperref package
    \definecolor{urlcolor}{rgb}{0,.145,.698}
    \definecolor{linkcolor}{rgb}{.71,0.21,0.01}
    \definecolor{citecolor}{rgb}{.12,.54,.11}

    % ANSI colors
    \definecolor{ansi-black}{HTML}{3E424D}
    \definecolor{ansi-black-intense}{HTML}{282C36}
    \definecolor{ansi-red}{HTML}{E75C58}
    \definecolor{ansi-red-intense}{HTML}{B22B31}
    \definecolor{ansi-green}{HTML}{00A250}
    \definecolor{ansi-green-intense}{HTML}{007427}
    \definecolor{ansi-yellow}{HTML}{DDB62B}
    \definecolor{ansi-yellow-intense}{HTML}{B27D12}
    \definecolor{ansi-blue}{HTML}{208FFB}
    \definecolor{ansi-blue-intense}{HTML}{0065CA}
    \definecolor{ansi-magenta}{HTML}{D160C4}
    \definecolor{ansi-magenta-intense}{HTML}{A03196}
    \definecolor{ansi-cyan}{HTML}{60C6C8}
    \definecolor{ansi-cyan-intense}{HTML}{258F8F}
    \definecolor{ansi-white}{HTML}{C5C1B4}
    \definecolor{ansi-white-intense}{HTML}{A1A6B2}

    % commands and environments needed by pandoc snippets
    % extracted from the output of `pandoc -s`
    \providecommand{\tightlist}{%
      \setlength{\itemsep}{0pt}\setlength{\parskip}{0pt}}
    \DefineVerbatimEnvironment{Highlighting}{Verbatim}{commandchars=\\\{\}}
    % Add ',fontsize=\small' for more characters per line
    \newenvironment{Shaded}{}{}
    \newcommand{\KeywordTok}[1]{\textcolor[rgb]{0.00,0.44,0.13}{\textbf{{#1}}}}
    \newcommand{\DataTypeTok}[1]{\textcolor[rgb]{0.56,0.13,0.00}{{#1}}}
    \newcommand{\DecValTok}[1]{\textcolor[rgb]{0.25,0.63,0.44}{{#1}}}
    \newcommand{\BaseNTok}[1]{\textcolor[rgb]{0.25,0.63,0.44}{{#1}}}
    \newcommand{\FloatTok}[1]{\textcolor[rgb]{0.25,0.63,0.44}{{#1}}}
    \newcommand{\CharTok}[1]{\textcolor[rgb]{0.25,0.44,0.63}{{#1}}}
    \newcommand{\StringTok}[1]{\textcolor[rgb]{0.25,0.44,0.63}{{#1}}}
    \newcommand{\CommentTok}[1]{\textcolor[rgb]{0.38,0.63,0.69}{\textit{{#1}}}}
    \newcommand{\OtherTok}[1]{\textcolor[rgb]{0.00,0.44,0.13}{{#1}}}
    \newcommand{\AlertTok}[1]{\textcolor[rgb]{1.00,0.00,0.00}{\textbf{{#1}}}}
    \newcommand{\FunctionTok}[1]{\textcolor[rgb]{0.02,0.16,0.49}{{#1}}}
    \newcommand{\RegionMarkerTok}[1]{{#1}}
    \newcommand{\ErrorTok}[1]{\textcolor[rgb]{1.00,0.00,0.00}{\textbf{{#1}}}}
    \newcommand{\NormalTok}[1]{{#1}}
    
    % Additional commands for more recent versions of Pandoc
    \newcommand{\ConstantTok}[1]{\textcolor[rgb]{0.53,0.00,0.00}{{#1}}}
    \newcommand{\SpecialCharTok}[1]{\textcolor[rgb]{0.25,0.44,0.63}{{#1}}}
    \newcommand{\VerbatimStringTok}[1]{\textcolor[rgb]{0.25,0.44,0.63}{{#1}}}
    \newcommand{\SpecialStringTok}[1]{\textcolor[rgb]{0.73,0.40,0.53}{{#1}}}
    \newcommand{\ImportTok}[1]{{#1}}
    \newcommand{\DocumentationTok}[1]{\textcolor[rgb]{0.73,0.13,0.13}{\textit{{#1}}}}
    \newcommand{\AnnotationTok}[1]{\textcolor[rgb]{0.38,0.63,0.69}{\textbf{\textit{{#1}}}}}
    \newcommand{\CommentVarTok}[1]{\textcolor[rgb]{0.38,0.63,0.69}{\textbf{\textit{{#1}}}}}
    \newcommand{\VariableTok}[1]{\textcolor[rgb]{0.10,0.09,0.49}{{#1}}}
    \newcommand{\ControlFlowTok}[1]{\textcolor[rgb]{0.00,0.44,0.13}{\textbf{{#1}}}}
    \newcommand{\OperatorTok}[1]{\textcolor[rgb]{0.40,0.40,0.40}{{#1}}}
    \newcommand{\BuiltInTok}[1]{{#1}}
    \newcommand{\ExtensionTok}[1]{{#1}}
    \newcommand{\PreprocessorTok}[1]{\textcolor[rgb]{0.74,0.48,0.00}{{#1}}}
    \newcommand{\AttributeTok}[1]{\textcolor[rgb]{0.49,0.56,0.16}{{#1}}}
    \newcommand{\InformationTok}[1]{\textcolor[rgb]{0.38,0.63,0.69}{\textbf{\textit{{#1}}}}}
    \newcommand{\WarningTok}[1]{\textcolor[rgb]{0.38,0.63,0.69}{\textbf{\textit{{#1}}}}}
    
    
    % Define a nice break command that doesn't care if a line doesn't already
    % exist.
    \def\br{\hspace*{\fill} \\* }
    % Math Jax compatability definitions
    \def\gt{>}
    \def\lt{<}
    % Document parameters
    \title{Econometrie}
    
    
    

    % Pygments definitions
    
\makeatletter
\def\PY@reset{\let\PY@it=\relax \let\PY@bf=\relax%
    \let\PY@ul=\relax \let\PY@tc=\relax%
    \let\PY@bc=\relax \let\PY@ff=\relax}
\def\PY@tok#1{\csname PY@tok@#1\endcsname}
\def\PY@toks#1+{\ifx\relax#1\empty\else%
    \PY@tok{#1}\expandafter\PY@toks\fi}
\def\PY@do#1{\PY@bc{\PY@tc{\PY@ul{%
    \PY@it{\PY@bf{\PY@ff{#1}}}}}}}
\def\PY#1#2{\PY@reset\PY@toks#1+\relax+\PY@do{#2}}

\expandafter\def\csname PY@tok@w\endcsname{\def\PY@tc##1{\textcolor[rgb]{0.73,0.73,0.73}{##1}}}
\expandafter\def\csname PY@tok@c\endcsname{\let\PY@it=\textit\def\PY@tc##1{\textcolor[rgb]{0.25,0.50,0.50}{##1}}}
\expandafter\def\csname PY@tok@cp\endcsname{\def\PY@tc##1{\textcolor[rgb]{0.74,0.48,0.00}{##1}}}
\expandafter\def\csname PY@tok@k\endcsname{\let\PY@bf=\textbf\def\PY@tc##1{\textcolor[rgb]{0.00,0.50,0.00}{##1}}}
\expandafter\def\csname PY@tok@kp\endcsname{\def\PY@tc##1{\textcolor[rgb]{0.00,0.50,0.00}{##1}}}
\expandafter\def\csname PY@tok@kt\endcsname{\def\PY@tc##1{\textcolor[rgb]{0.69,0.00,0.25}{##1}}}
\expandafter\def\csname PY@tok@o\endcsname{\def\PY@tc##1{\textcolor[rgb]{0.40,0.40,0.40}{##1}}}
\expandafter\def\csname PY@tok@ow\endcsname{\let\PY@bf=\textbf\def\PY@tc##1{\textcolor[rgb]{0.67,0.13,1.00}{##1}}}
\expandafter\def\csname PY@tok@nb\endcsname{\def\PY@tc##1{\textcolor[rgb]{0.00,0.50,0.00}{##1}}}
\expandafter\def\csname PY@tok@nf\endcsname{\def\PY@tc##1{\textcolor[rgb]{0.00,0.00,1.00}{##1}}}
\expandafter\def\csname PY@tok@nc\endcsname{\let\PY@bf=\textbf\def\PY@tc##1{\textcolor[rgb]{0.00,0.00,1.00}{##1}}}
\expandafter\def\csname PY@tok@nn\endcsname{\let\PY@bf=\textbf\def\PY@tc##1{\textcolor[rgb]{0.00,0.00,1.00}{##1}}}
\expandafter\def\csname PY@tok@ne\endcsname{\let\PY@bf=\textbf\def\PY@tc##1{\textcolor[rgb]{0.82,0.25,0.23}{##1}}}
\expandafter\def\csname PY@tok@nv\endcsname{\def\PY@tc##1{\textcolor[rgb]{0.10,0.09,0.49}{##1}}}
\expandafter\def\csname PY@tok@no\endcsname{\def\PY@tc##1{\textcolor[rgb]{0.53,0.00,0.00}{##1}}}
\expandafter\def\csname PY@tok@nl\endcsname{\def\PY@tc##1{\textcolor[rgb]{0.63,0.63,0.00}{##1}}}
\expandafter\def\csname PY@tok@ni\endcsname{\let\PY@bf=\textbf\def\PY@tc##1{\textcolor[rgb]{0.60,0.60,0.60}{##1}}}
\expandafter\def\csname PY@tok@na\endcsname{\def\PY@tc##1{\textcolor[rgb]{0.49,0.56,0.16}{##1}}}
\expandafter\def\csname PY@tok@nt\endcsname{\let\PY@bf=\textbf\def\PY@tc##1{\textcolor[rgb]{0.00,0.50,0.00}{##1}}}
\expandafter\def\csname PY@tok@nd\endcsname{\def\PY@tc##1{\textcolor[rgb]{0.67,0.13,1.00}{##1}}}
\expandafter\def\csname PY@tok@s\endcsname{\def\PY@tc##1{\textcolor[rgb]{0.73,0.13,0.13}{##1}}}
\expandafter\def\csname PY@tok@sd\endcsname{\let\PY@it=\textit\def\PY@tc##1{\textcolor[rgb]{0.73,0.13,0.13}{##1}}}
\expandafter\def\csname PY@tok@si\endcsname{\let\PY@bf=\textbf\def\PY@tc##1{\textcolor[rgb]{0.73,0.40,0.53}{##1}}}
\expandafter\def\csname PY@tok@se\endcsname{\let\PY@bf=\textbf\def\PY@tc##1{\textcolor[rgb]{0.73,0.40,0.13}{##1}}}
\expandafter\def\csname PY@tok@sr\endcsname{\def\PY@tc##1{\textcolor[rgb]{0.73,0.40,0.53}{##1}}}
\expandafter\def\csname PY@tok@ss\endcsname{\def\PY@tc##1{\textcolor[rgb]{0.10,0.09,0.49}{##1}}}
\expandafter\def\csname PY@tok@sx\endcsname{\def\PY@tc##1{\textcolor[rgb]{0.00,0.50,0.00}{##1}}}
\expandafter\def\csname PY@tok@m\endcsname{\def\PY@tc##1{\textcolor[rgb]{0.40,0.40,0.40}{##1}}}
\expandafter\def\csname PY@tok@gh\endcsname{\let\PY@bf=\textbf\def\PY@tc##1{\textcolor[rgb]{0.00,0.00,0.50}{##1}}}
\expandafter\def\csname PY@tok@gu\endcsname{\let\PY@bf=\textbf\def\PY@tc##1{\textcolor[rgb]{0.50,0.00,0.50}{##1}}}
\expandafter\def\csname PY@tok@gd\endcsname{\def\PY@tc##1{\textcolor[rgb]{0.63,0.00,0.00}{##1}}}
\expandafter\def\csname PY@tok@gi\endcsname{\def\PY@tc##1{\textcolor[rgb]{0.00,0.63,0.00}{##1}}}
\expandafter\def\csname PY@tok@gr\endcsname{\def\PY@tc##1{\textcolor[rgb]{1.00,0.00,0.00}{##1}}}
\expandafter\def\csname PY@tok@ge\endcsname{\let\PY@it=\textit}
\expandafter\def\csname PY@tok@gs\endcsname{\let\PY@bf=\textbf}
\expandafter\def\csname PY@tok@gp\endcsname{\let\PY@bf=\textbf\def\PY@tc##1{\textcolor[rgb]{0.00,0.00,0.50}{##1}}}
\expandafter\def\csname PY@tok@go\endcsname{\def\PY@tc##1{\textcolor[rgb]{0.53,0.53,0.53}{##1}}}
\expandafter\def\csname PY@tok@gt\endcsname{\def\PY@tc##1{\textcolor[rgb]{0.00,0.27,0.87}{##1}}}
\expandafter\def\csname PY@tok@err\endcsname{\def\PY@bc##1{\setlength{\fboxsep}{0pt}\fcolorbox[rgb]{1.00,0.00,0.00}{1,1,1}{\strut ##1}}}
\expandafter\def\csname PY@tok@kc\endcsname{\let\PY@bf=\textbf\def\PY@tc##1{\textcolor[rgb]{0.00,0.50,0.00}{##1}}}
\expandafter\def\csname PY@tok@kd\endcsname{\let\PY@bf=\textbf\def\PY@tc##1{\textcolor[rgb]{0.00,0.50,0.00}{##1}}}
\expandafter\def\csname PY@tok@kn\endcsname{\let\PY@bf=\textbf\def\PY@tc##1{\textcolor[rgb]{0.00,0.50,0.00}{##1}}}
\expandafter\def\csname PY@tok@kr\endcsname{\let\PY@bf=\textbf\def\PY@tc##1{\textcolor[rgb]{0.00,0.50,0.00}{##1}}}
\expandafter\def\csname PY@tok@bp\endcsname{\def\PY@tc##1{\textcolor[rgb]{0.00,0.50,0.00}{##1}}}
\expandafter\def\csname PY@tok@fm\endcsname{\def\PY@tc##1{\textcolor[rgb]{0.00,0.00,1.00}{##1}}}
\expandafter\def\csname PY@tok@vc\endcsname{\def\PY@tc##1{\textcolor[rgb]{0.10,0.09,0.49}{##1}}}
\expandafter\def\csname PY@tok@vg\endcsname{\def\PY@tc##1{\textcolor[rgb]{0.10,0.09,0.49}{##1}}}
\expandafter\def\csname PY@tok@vi\endcsname{\def\PY@tc##1{\textcolor[rgb]{0.10,0.09,0.49}{##1}}}
\expandafter\def\csname PY@tok@vm\endcsname{\def\PY@tc##1{\textcolor[rgb]{0.10,0.09,0.49}{##1}}}
\expandafter\def\csname PY@tok@sa\endcsname{\def\PY@tc##1{\textcolor[rgb]{0.73,0.13,0.13}{##1}}}
\expandafter\def\csname PY@tok@sb\endcsname{\def\PY@tc##1{\textcolor[rgb]{0.73,0.13,0.13}{##1}}}
\expandafter\def\csname PY@tok@sc\endcsname{\def\PY@tc##1{\textcolor[rgb]{0.73,0.13,0.13}{##1}}}
\expandafter\def\csname PY@tok@dl\endcsname{\def\PY@tc##1{\textcolor[rgb]{0.73,0.13,0.13}{##1}}}
\expandafter\def\csname PY@tok@s2\endcsname{\def\PY@tc##1{\textcolor[rgb]{0.73,0.13,0.13}{##1}}}
\expandafter\def\csname PY@tok@sh\endcsname{\def\PY@tc##1{\textcolor[rgb]{0.73,0.13,0.13}{##1}}}
\expandafter\def\csname PY@tok@s1\endcsname{\def\PY@tc##1{\textcolor[rgb]{0.73,0.13,0.13}{##1}}}
\expandafter\def\csname PY@tok@mb\endcsname{\def\PY@tc##1{\textcolor[rgb]{0.40,0.40,0.40}{##1}}}
\expandafter\def\csname PY@tok@mf\endcsname{\def\PY@tc##1{\textcolor[rgb]{0.40,0.40,0.40}{##1}}}
\expandafter\def\csname PY@tok@mh\endcsname{\def\PY@tc##1{\textcolor[rgb]{0.40,0.40,0.40}{##1}}}
\expandafter\def\csname PY@tok@mi\endcsname{\def\PY@tc##1{\textcolor[rgb]{0.40,0.40,0.40}{##1}}}
\expandafter\def\csname PY@tok@il\endcsname{\def\PY@tc##1{\textcolor[rgb]{0.40,0.40,0.40}{##1}}}
\expandafter\def\csname PY@tok@mo\endcsname{\def\PY@tc##1{\textcolor[rgb]{0.40,0.40,0.40}{##1}}}
\expandafter\def\csname PY@tok@ch\endcsname{\let\PY@it=\textit\def\PY@tc##1{\textcolor[rgb]{0.25,0.50,0.50}{##1}}}
\expandafter\def\csname PY@tok@cm\endcsname{\let\PY@it=\textit\def\PY@tc##1{\textcolor[rgb]{0.25,0.50,0.50}{##1}}}
\expandafter\def\csname PY@tok@cpf\endcsname{\let\PY@it=\textit\def\PY@tc##1{\textcolor[rgb]{0.25,0.50,0.50}{##1}}}
\expandafter\def\csname PY@tok@c1\endcsname{\let\PY@it=\textit\def\PY@tc##1{\textcolor[rgb]{0.25,0.50,0.50}{##1}}}
\expandafter\def\csname PY@tok@cs\endcsname{\let\PY@it=\textit\def\PY@tc##1{\textcolor[rgb]{0.25,0.50,0.50}{##1}}}

\def\PYZbs{\char`\\}
\def\PYZus{\char`\_}
\def\PYZob{\char`\{}
\def\PYZcb{\char`\}}
\def\PYZca{\char`\^}
\def\PYZam{\char`\&}
\def\PYZlt{\char`\<}
\def\PYZgt{\char`\>}
\def\PYZsh{\char`\#}
\def\PYZpc{\char`\%}
\def\PYZdl{\char`\$}
\def\PYZhy{\char`\-}
\def\PYZsq{\char`\'}
\def\PYZdq{\char`\"}
\def\PYZti{\char`\~}
% for compatibility with earlier versions
\def\PYZat{@}
\def\PYZlb{[}
\def\PYZrb{]}
\makeatother


    % Exact colors from NB
    \definecolor{incolor}{rgb}{0.0, 0.0, 0.5}
    \definecolor{outcolor}{rgb}{0.545, 0.0, 0.0}



    
    % Prevent overflowing lines due to hard-to-break entities
    \sloppy 
    % Setup hyperref package
    \hypersetup{
      breaklinks=true,  % so long urls are correctly broken across lines
      colorlinks=true,
      urlcolor=urlcolor,
      linkcolor=linkcolor,
      citecolor=citecolor,
      }
    % Slightly bigger margins than the latex defaults
    
    \geometry{verbose,tmargin=1in,bmargin=1in,lmargin=1in,rmargin=1in}
    
    

    \begin{document}
    
    
    \maketitle
    
    

    
    \section{Projet SES722 2018-2019}\label{projet-ses722-2018-2019}

\emph{BEC Alexandre, FABIEN Maël}

    \section{Partie I - Regression}\label{partie-i---regression}

    \subsection{Question 1}\label{question-1}

\begin{itemize}
\tightlist
\item
  Lire le fichier \textbf{mroz.txt}. 
\end{itemize}

https://www.rdocumentation.org/packages/car/versions/2.1-6/topics/Mroz

\begin{itemize}
\tightlist
\item
  Ne sélectionner que les observations pour lesquelles la variable
  \textbf{wage} est strictement positive.
\end{itemize}

    \begin{Verbatim}[commandchars=\\\{\}]
{\color{incolor}In [{\color{incolor}129}]:} \PY{n}{df\PYZus{}filt}\PY{o}{.}\PY{n}{head}\PY{p}{(}\PY{p}{)}
\end{Verbatim}


    \begin{Verbatim}[commandchars=\\\{\}]
On dispose de  428  observations.

    \end{Verbatim}

\begin{Verbatim}[commandchars=\\\{\}]
{\color{outcolor}Out[{\color{outcolor}129}]:}    inlf  hours  kidslt6  kidsgt6  age  educ    wage  repwage  hushrs  husage  \textbackslash{}
          0     1   1610        1        0   32    12  3.3540     2.65    2708      34   
          1     1   1656        0        2   30    12  1.3889     2.65    2310      30   
          2     1   1980        1        3   35    12  4.5455     4.04    3072      40   
          3     1    456        0        3   34    12  1.0965     3.25    1920      53   
          4     1   1568        1        2   31    14  4.5918     3.60    2000      32   
          
              {\ldots}     faminc     mtr  motheduc  fatheduc  unem  city  exper   nwifeinc  \textbackslash{}
          0   {\ldots}      16310  0.7215        12         7   5.0     0     14  10.910060   
          1   {\ldots}      21800  0.6615         7         7  11.0     1      5  19.499980   
          2   {\ldots}      21040  0.6915        12         7   5.0     0     15  12.039910   
          3   {\ldots}       7300  0.7815         7         7   5.0     0      6   6.799996   
          4   {\ldots}      27300  0.6215        12        14   9.5     1      7  20.100060   
          
                lwage  expersq  
          0  1.210154      196  
          1  0.328512       25  
          2  1.514138      225  
          3  0.092123       36  
          4  1.524272       49  
          
          [5 rows x 22 columns]
\end{Verbatim}
            
    \subsection{Question 2}\label{question-2}

\begin{itemize}
\tightlist
\item
  Faire les statistiques descriptives du salaire, de l'age et de
  l'éducation pour : 
\end{itemize}

    \subsubsection{L'ensemble des femmes}\label{lensemble-des-femmes}

    \begin{Verbatim}[commandchars=\\\{\}]
{\color{incolor}In [{\color{incolor}130}]:} 
\end{Verbatim}


\begin{Verbatim}[commandchars=\\\{\}]
{\color{outcolor}Out[{\color{outcolor}130}]:}              wage         age        educ
          count  428.000000  428.000000  428.000000
          mean     4.177682   41.971963   12.658879
          std      3.310282    7.721084    2.285376
          min      0.128200   30.000000    5.000000
          25\%      2.262600   35.000000   12.000000
          50\%      3.481900   42.000000   12.000000
          75\%      4.970750   47.250000   14.000000
          max     25.000000   60.000000   17.000000
\end{Verbatim}
            
    Le salaire moyen des femmes dans la base de données est de 4.17. L'âge
moyen est 42 ans. Le nombre d'années d'éducation moyen est de 12.7
années.

    \subsubsection{Pour les femmes dont le salaire du mari est supérieure à
la médiane de
l'échantillon.}\label{pour-les-femmes-dont-le-salaire-du-mari-est-supuxe9rieure-uxe0-la-muxe9diane-de-luxe9chantillon.}

    \begin{Verbatim}[commandchars=\\\{\}]
{\color{incolor}In [{\color{incolor}132}]:} 
\end{Verbatim}


\begin{Verbatim}[commandchars=\\\{\}]
{\color{outcolor}Out[{\color{outcolor}132}]:}              wage         age        educ
          count  214.000000  214.000000  214.000000
          mean     4.896822   42.275701   13.242991
          std      4.041606    7.388843    2.359045
          min      0.161600   30.000000    5.000000
          25\%      2.513850   36.000000   12.000000
          50\%      3.846400   43.000000   12.000000
          75\%      5.854125   48.000000   16.000000
          max     25.000000   59.000000   17.000000
\end{Verbatim}
            
    Le salaire moyen des femmes dont mari gagne plus que la médiane est de
4.90. L'âge moyen est de 42.3 ans, et le nombre d'années d'éducation
moyen est de 13.2 années.

    \subsubsection{Pour les femmes dont le salaire du mari est inférieur à
la médiane de
l'échantillon.}\label{pour-les-femmes-dont-le-salaire-du-mari-est-infuxe9rieur-uxe0-la-muxe9diane-de-luxe9chantillon.}

    \begin{Verbatim}[commandchars=\\\{\}]
{\color{incolor}In [{\color{incolor}134}]:} 
\end{Verbatim}


\begin{Verbatim}[commandchars=\\\{\}]
{\color{outcolor}Out[{\color{outcolor}134}]:}              wage         age        educ
          count  214.000000  214.000000  214.000000
          mean     3.458541   41.668224   12.074766
          std      2.143274    8.045482    2.054200
          min      0.128200   30.000000    6.000000
          25\%      2.117275   35.000000   12.000000
          50\%      2.971800   41.000000   12.000000
          75\%      4.393800   47.000000   12.000000
          max     18.267000   60.000000   17.000000
\end{Verbatim}
            
    Le salaire moyen des femmes dont la mari gagne moins que la médiane est
de 3.46. On voit que les femmes dont le mari gagne plus que la médiane
gagnent en moyenne plus que les femmes dont le mari gagne moins que la
médiane. Par ailleurs, l'écart-type est plus élevé pour les femmes dont
le mari gagne plus que la médiane.

L'âge moyen pour le groupe des femmes dont le mari gagne moins que la
médiane est de 41.7 ans, et le nombre d'années d'éducation moyen est
12.1 années.

On peut supposer l'existence de deux sous-groupes : - les femmes ayant
bénéficié de moins d'éducation et disposant d'un salaire inférieur - les
femmes plus jeunes, et dont le mari est probablement plus jeune, qui
gagnent moins que la médiane.

    \subsubsection{Graphes de distribution}\label{graphes-de-distribution}

    \begin{Verbatim}[commandchars=\\\{\}]
{\color{incolor}In [{\color{incolor}135}]:} 
\end{Verbatim}


    \begin{center}
    \adjustimage{max size={0.9\linewidth}{0.9\paperheight}}{output_15_0.png}
    \end{center}
    { \hspace*{\fill} \\}
    
    \begin{Verbatim}[commandchars=\\\{\}]
{\color{incolor}In [{\color{incolor}136}]:} 
\end{Verbatim}


    \begin{center}
    \adjustimage{max size={0.9\linewidth}{0.9\paperheight}}{output_16_0.png}
    \end{center}
    { \hspace*{\fill} \\}
    
    \begin{Verbatim}[commandchars=\\\{\}]
{\color{incolor}In [{\color{incolor}137}]:} 
\end{Verbatim}


    \begin{center}
    \adjustimage{max size={0.9\linewidth}{0.9\paperheight}}{output_17_0.png}
    \end{center}
    { \hspace*{\fill} \\}
    
    \begin{Verbatim}[commandchars=\\\{\}]
{\color{incolor}In [{\color{incolor}138}]:} 
\end{Verbatim}


    \begin{center}
    \adjustimage{max size={0.9\linewidth}{0.9\paperheight}}{output_18_0.png}
    \end{center}
    { \hspace*{\fill} \\}
    
    \begin{Verbatim}[commandchars=\\\{\}]
{\color{incolor}In [{\color{incolor}139}]:} 
\end{Verbatim}


    \begin{center}
    \adjustimage{max size={0.9\linewidth}{0.9\paperheight}}{output_19_0.png}
    \end{center}
    { \hspace*{\fill} \\}
    
    Par ces distributions, on illustre que lorsque le salaire du mari est
supérieur à la médiane, le salaire de la femme semble être généralement
plus élevé. Ceci devrait s'illustrer par une corrélation positive entre
ces deux valeurs. La variance est également plus importante parmi les
salaires des femmes dont le mari gagne plus que la médiane sur le
box-plot.

    \subsection{Question 3}\label{question-3}

\begin{itemize}
\tightlist
\item
  Faire l'histogramme de la variable wage.
\end{itemize}

    \begin{Verbatim}[commandchars=\\\{\}]
{\color{incolor}In [{\color{incolor}140}]:} 
\end{Verbatim}


    \begin{center}
    \adjustimage{max size={0.9\linewidth}{0.9\paperheight}}{output_22_0.png}
    \end{center}
    { \hspace*{\fill} \\}
    
    \begin{itemize}
\tightlist
\item
  Calculer le log de wage et faire l'histogramme.
\end{itemize}

    \begin{Verbatim}[commandchars=\\\{\}]
{\color{incolor}In [{\color{incolor}141}]:} 
\end{Verbatim}


    \begin{center}
    \adjustimage{max size={0.9\linewidth}{0.9\paperheight}}{output_24_0.png}
    \end{center}
    { \hspace*{\fill} \\}
    
    \begin{itemize}
\tightlist
\item
  Comparez les deux histogrammes et commentez
\end{itemize}

En prenant la variable \textbf{lwage}, la distribution semble plus
proche d'un gaussienne. L'histogramme de \textbf{wage} est asymétrique
car non négatif. C'est confirmé par le score de skewness avec
\textbf{stats.skew}, qui est fortement positif pour la feature
\textbf{wage}.

    \begin{Verbatim}[commandchars=\\\{\}]
{\color{incolor}In [{\color{incolor}142}]:} \PY{n+nb}{print}\PY{p}{(}\PY{n}{stats}\PY{o}{.}\PY{n}{skew}\PY{p}{(}\PY{n}{df\PYZus{}filt}\PY{o}{.}\PY{n}{wage}\PY{p}{)}\PY{p}{)}
          \PY{n+nb}{print}\PY{p}{(}\PY{n}{stats}\PY{o}{.}\PY{n}{skew}\PY{p}{(}\PY{n}{df\PYZus{}filt}\PY{o}{.}\PY{n}{lwage}\PY{p}{)}\PY{p}{)}
\end{Verbatim}


    \begin{Verbatim}[commandchars=\\\{\}]
3.0801391789818724
-0.6851599225277718

    \end{Verbatim}

    On peut supposer que travailler avec la variable lwage permettra de
résoudre certains problèmes posés par wage.

    \subsection{Question 4}\label{question-4}

\begin{itemize}
\tightlist
\item
  Calculer les corrélations motheduc et fatheduc.
\end{itemize}

    \begin{Verbatim}[commandchars=\\\{\}]
{\color{incolor}In [{\color{incolor}143}]:} 
\end{Verbatim}


    \begin{Verbatim}[commandchars=\\\{\}]
Corrélation motheduc vs fatheduc:  0.554
p-value pour H0 pas de corrélation :  0.0

    \end{Verbatim}

    \begin{itemize}
\tightlist
\item
  Commentez.
\end{itemize}

La corrélation est de 55\% entre l'éducation de la mère et celle du
père. Cela s'explique probablement par le fait que les deux personnes au
sein d'un couple appartiennent souvent à la même classe sociale et
bénéficient des mêmes possibilités d'accès à l'éducation.

\begin{itemize}
\tightlist
\item
  Il y a-t-il un problème de multicollinéarité si l'on utilise ces
  variables comme variables explicatives ?
\end{itemize}

On rejette l'hypothèse nulle (pas de correlation) au seuil de 5\%. La
corrélaiton entre les deux variables peut introduire un biais de
multicollinéarité.

Inclure ces deux variables peut surpondérer l'information sur
l'éducation des parents. Cependant, une corrélation de 0.55 n'est pas
une corrélation parfaite, et peut porter de l'information.

    On peut d'ailleurs illustrer la corrélation entre les deux séries de
cette manière. On remarque

    \begin{Verbatim}[commandchars=\\\{\}]
{\color{incolor}In [{\color{incolor}144}]:} 
\end{Verbatim}


    \begin{center}
    \adjustimage{max size={0.9\linewidth}{0.9\paperheight}}{output_32_0.png}
    \end{center}
    { \hspace*{\fill} \\}
    
    Il semble cependant y avoir un léger effet de compensation entre le
nombre d'années d'éducation des parents. En effet, lorsque l'on affiche
la différence entre le nombre d'années d'étude des deux parents, en
fonction du nombre d'années d'étude de la mère, on se rend compte que :
- lorsque la mère réalise très peu d'études, le père réalise
généralement plus d'études - lorsque la mère réalise beaucoup d'études,
le père réalise en moyenne légèrement moins d'études

    \begin{Verbatim}[commandchars=\\\{\}]
{\color{incolor}In [{\color{incolor}145}]:} 
\end{Verbatim}


    \begin{center}
    \adjustimage{max size={0.9\linewidth}{0.9\paperheight}}{output_34_0.png}
    \end{center}
    { \hspace*{\fill} \\}
    
    \subsection{Question 5}\label{question-5}

\begin{itemize}
\tightlist
\item
  Faites un graphique en nuage de point entre wage et educ, wage et
  exper, wage et fatheduc. Commentez. S'agit-il d'un effet "toute chose
  étant égale par ailleurs ?"
\end{itemize}

    \begin{Verbatim}[commandchars=\\\{\}]
{\color{incolor}In [{\color{incolor}146}]:} 
\end{Verbatim}


    \begin{center}
    \adjustimage{max size={0.9\linewidth}{0.9\paperheight}}{output_36_0.png}
    \end{center}
    { \hspace*{\fill} \\}
    
    \begin{itemize}
\tightlist
\item
  Il ne s'agit pas d'un effet "toute chose étant égale par ailleurs",
  car pour chaque donnée, le reste des variables ne sont pas constantes.
\item
  D'après la regression linéaire, les années d'éducation supplémentaires
  semblent augmenter significativement le salaire. Nous contrôlerons
  ceci juste après.
\item
  Il semble que la variance ne soit pas la même pour chacune des années
  d'éducation (12 et 17 avec des pics de variance). Cela peut impliquer
  des problèmes d'héteroscedasticité.
\end{itemize}

    \begin{Verbatim}[commandchars=\\\{\}]
{\color{incolor}In [{\color{incolor}147}]:} 
\end{Verbatim}


    \begin{center}
    \adjustimage{max size={0.9\linewidth}{0.9\paperheight}}{output_38_0.png}
    \end{center}
    { \hspace*{\fill} \\}
    
    \begin{itemize}
\tightlist
\item
  Certaines observations abérantes biaisent la pente.
\item
  L'effet semble moins significatifs que l'éducation sur le salaire.
\item
  Il existe un biais par rapport au nombre d'années d'éducation en
  amont. En effet, les jeunes diplomés avec peu d'expérience peuvent
  atteindre le même salaire d'entrée que des personnes travailllant
  depuis leur plus jeune âge.
\end{itemize}

    On remarque que des années d'éducation supplémentaires ont tendance à
faire augmenter le salaire. Cependant, l'expérience a un effet beaucoup
moins prononcé. On remarque notamment quelques points pour lesquels les
années d'étude sont relativement élevées et l'expérience très faible,
mais pour autant le salaire est particulièrement élevé.

    \begin{Verbatim}[commandchars=\\\{\}]
{\color{incolor}In [{\color{incolor}23}]:} 
\end{Verbatim}


    \begin{center}
    \adjustimage{max size={0.9\linewidth}{0.9\paperheight}}{output_41_0.png}
    \end{center}
    { \hspace*{\fill} \\}
    
    Des années d'éducation supplémentaires du père semblent apporter un
meilleur salaire aux enfants. Cependant, la variance ne semble pas
uniforme selon le nombre d'années d'études du père.

    \subsection{Question 6}\label{question-6}

\begin{itemize}
\tightlist
\item
  Quelle est l'hypothèse fondamentale qui garantie des estimateurs non
  biaisés ? 
\end{itemize}

L'hypothèse de normalité des résidus garantit l'obtention du meilleur
estimateur linéaire non-biaisé (BLUE). Ainsi, les résidus sont centrés
en zéro, de variance constante à travers le temps (iid). On parle alors
d'homoscedasticité.

\begin{itemize}
\tightlist
\item
  Expliquer le biais de variable omise.
\end{itemize}

On estime un modèle en prenant certaines variables, mais il en existent
d'autres, que l'on ne peut pas contrôler par manque de données. Une
variable omise viole l'hspothèse de normalité des résidus, car l'effet
des variables omises se retrouve en partie dans les résidus.

    \subsection{Question 7}\label{question-7}

\begin{itemize}
\item
  Faire la régression de wage en utilisant les variables explicatives
  une constante, city, educ, exper, nwifeinc, kidslt6, kidsgt6. 
\item
  Commentez l'histogramme des résidus.
\end{itemize}

    \begin{Verbatim}[commandchars=\\\{\}]
{\color{incolor}In [{\color{incolor}26}]:} 
\end{Verbatim}


    \begin{center}
    \adjustimage{max size={0.9\linewidth}{0.9\paperheight}}{output_45_0.png}
    \end{center}
    { \hspace*{\fill} \\}
    
    \begin{Verbatim}[commandchars=\\\{\}]
{\color{incolor}In [{\color{incolor}27}]:} 
\end{Verbatim}


    \begin{Verbatim}[commandchars=\\\{\}]
Score Skew \& Kurtosis :  345.8247086376311

    \end{Verbatim}

    \begin{itemize}
\item
  Les résidus ne sont pas gaussiens mais centrés en zéro.
\item
  La variance des résidus possède des valeurs extrêmes, qui entâchent
  l'hypothèse de normalité des résidus. Si on supprime ces valeurs
  extrêmes des résidus et que l'on trace l'historgramme d'une gaussienne
  de meme moyenne/variance, les deux courbes sont relativement proches,
  malgré une légère asymétrie des résidus.
\end{itemize}

    \begin{Verbatim}[commandchars=\\\{\}]
{\color{incolor}In [{\color{incolor}28}]:} 
\end{Verbatim}


    \begin{Verbatim}[commandchars=\\\{\}]
                            OLS Regression Results                            
==============================================================================
Dep. Variable:                   wage   R-squared:                       0.127
Model:                            OLS   Adj. R-squared:                  0.115
Method:                 Least Squares   F-statistic:                     10.23
Date:                Fri, 26 Apr 2019   Prob (F-statistic):           1.41e-10
Time:                        17:40:10   Log-Likelihood:                -1090.0
No. Observations:                 428   AIC:                             2194.
Df Residuals:                     421   BIC:                             2222.
Df Model:                           6                                         
Covariance Type:            nonrobust                                         
==============================================================================
                 coef    std err          t      P>|t|      [0.025      0.975]
------------------------------------------------------------------------------
const         -2.4035      0.963     -2.495      0.013      -4.297      -0.510
city           0.3698      0.327      1.132      0.258      -0.272       1.012
educ           0.4600      0.070      6.546      0.000       0.322       0.598
exper          0.0238      0.021      1.141      0.255      -0.017       0.065
nwifeinc       0.0152      0.015      0.984      0.326      -0.015       0.046
kidslt6        0.0362      0.397      0.091      0.927      -0.744       0.816
kidsgt6       -0.0619      0.125     -0.494      0.622      -0.308       0.185
==============================================================================
Omnibus:                      345.825   Durbin-Watson:                   2.056
Prob(Omnibus):                  0.000   Jarque-Bera (JB):             6499.375
Skew:                           3.389   Prob(JB):                         0.00
Kurtosis:                      20.847   Cond. No.                         178.
==============================================================================

Warnings:
[1] Standard Errors assume that the covariance matrix of the errors is correctly specified.

    \end{Verbatim}

    \begin{itemize}
\tightlist
\item
  Seule l'éducaton apparait comme variable significative quand on
  utilise une constante.
\end{itemize}

    \subsection{Question 8}\label{question-8}

\begin{itemize}
\tightlist
\item
  Faire la régression de lwage sur une constante, city, educ, exper,
  nwifeinc, kidslt6, kidsgt6. Comparer l'histogramme obtenu à celui de
  la question 7. 
\end{itemize}

    \begin{Verbatim}[commandchars=\\\{\}]
{\color{incolor}In [{\color{incolor}30}]:} 
\end{Verbatim}


    \begin{center}
    \adjustimage{max size={0.9\linewidth}{0.9\paperheight}}{output_51_0.png}
    \end{center}
    { \hspace*{\fill} \\}
    
    \begin{Verbatim}[commandchars=\\\{\}]
Score Skew \& Kurtosis :  79.5424673464374

    \end{Verbatim}

    \begin{Verbatim}[commandchars=\\\{\}]
{\color{incolor}In [{\color{incolor}31}]:} 
\end{Verbatim}


    \begin{center}
    \adjustimage{max size={0.9\linewidth}{0.9\paperheight}}{output_52_0.png}
    \end{center}
    { \hspace*{\fill} \\}
    
    Le passage en log corrige l'heteroscedasticité remarquée à la question
7. La distribution des résidus est relativement proche de la Gaussienne,
malgré une légère sur-concentration en zéro.

    \begin{Verbatim}[commandchars=\\\{\}]
{\color{incolor}In [{\color{incolor}32}]:} 
\end{Verbatim}


    \begin{Verbatim}[commandchars=\\\{\}]
                            OLS Regression Results                            
==============================================================================
Dep. Variable:                  lwage   R-squared:                       0.156
Model:                            OLS   Adj. R-squared:                  0.144
Method:                 Least Squares   F-statistic:                     12.92
Date:                Fri, 26 Apr 2019   Prob (F-statistic):           2.00e-13
Time:                        17:40:11   Log-Likelihood:                -431.92
No. Observations:                 428   AIC:                             877.8
Df Residuals:                     421   BIC:                             906.3
Df Model:                           6                                         
Covariance Type:            nonrobust                                         
==============================================================================
                 coef    std err          t      P>|t|      [0.025      0.975]
------------------------------------------------------------------------------
const         -0.3990      0.207     -1.927      0.055      -0.806       0.008
city           0.0353      0.070      0.503      0.616      -0.103       0.173
educ           0.1022      0.015      6.771      0.000       0.073       0.132
exper          0.0155      0.004      3.452      0.001       0.007       0.024
nwifeinc       0.0049      0.003      1.466      0.143      -0.002       0.011
kidslt6       -0.0453      0.085     -0.531      0.596      -0.213       0.122
kidsgt6       -0.0117      0.027     -0.434      0.664      -0.065       0.041
==============================================================================
Omnibus:                       79.542   Durbin-Watson:                   1.979
Prob(Omnibus):                  0.000   Jarque-Bera (JB):              287.193
Skew:                          -0.795   Prob(JB):                     4.33e-63
Kurtosis:                       6.685   Cond. No.                         178.
==============================================================================

Warnings:
[1] Standard Errors assume that the covariance matrix of the errors is correctly specified.

    \end{Verbatim}

    Les variables éducation et expérience apparaissent alors comme
significatives.

    \subsection{Question 9}\label{question-9}

\begin{itemize}
\tightlist
\item
  Tester l'hypothèse de non significativité de exper avec un seuil de
  significativité de 1\%, 5\% et 10\% (test alternatif des deux côtés). 
\end{itemize}

On continue de travailler avec logwage.

    Les hypothèses de notre modèle peuvent s'exprimer comme suit :

\[ H_0 : \beta_{exper} = 0 \]

\[ H_1 : \beta_{exper} != 0 \]

On calcule la stat de test suivante :

\[ t_{exper} = \frac { \hat{\beta}_{exper}} {\hat{\sigma}_{exper}} \]

Puis l'on compare cette statistique de test à une valeur critique
\[ {t_{n-k}}^{\alpha \%} \] .

Si la stat de test est supérieure à la valeur critique, on rejette \$
H\_0 \$. Autrement, on ne peut pas conclure au rejet de l'hypothèse
nulle.

    \begin{Verbatim}[commandchars=\\\{\}]
{\color{incolor}In [{\color{incolor}34}]:} 
\end{Verbatim}


    \begin{Verbatim}[commandchars=\\\{\}]
n, k :  428 7
sig2: 0.4479571976072196
t\_exper :  3.4517182808127607

    \end{Verbatim}

    \begin{itemize}
\tightlist
\item
  Commentez les p-values.
\end{itemize}

    \begin{Verbatim}[commandchars=\\\{\}]
{\color{incolor}In [{\color{incolor}35}]:} 
\end{Verbatim}


    \begin{Verbatim}[commandchars=\\\{\}]
Seuil: 10\%
p\_value de t\_exper :  0.0006133650790143275
0.00 < 0.10: On rejette l'hypothèse de non-significativité 

Seuil: 5\%
p\_value de t\_exper :  0.0006133650790143275
0.00 < 0.05: On rejette l'hypothèse de non-significativité 

Seuil: 1\%
p\_value de t\_exper :  0.0006133650790143275
0.00 < 0.01: On rejette l'hypothèse de non-significativité 


    \end{Verbatim}

    La p-value associée à la variable Experience étant particulièrement
faible, on rejette l'hypothèse de non-significativité de la variable sur
lwage.

    \subsection{Question 10}\label{question-10}

\begin{itemize}
\tightlist
\item
  Tester l'hypothèse que le coefficient associé à educ est égal à 10\%
  avec un seuil de significativité de 5\% (test à alternatif des deux
  côtés)
\end{itemize}

    On teste désormais :

\[ t_{educ} = ( \hat{\beta}_{educ} - 0.1 ) / ( \hat{\sigma}_{educ} ) \]

et l'on compare à nouveau cette statistique de test à la valeur
critique.

    \begin{Verbatim}[commandchars=\\\{\}]
{\color{incolor}In [{\color{incolor}36}]:} 
\end{Verbatim}


    \begin{Verbatim}[commandchars=\\\{\}]
T-stat :  0.14882666468792646
Seuil: 5\%
p-value : 0.8817616705976787
0.88 > 0.05: On ne rejette pas l'hypothèse de non-significativité 


    \end{Verbatim}

    La p-value vaut 0.88. On ne rejette donc pas l'hypothèse que le
coefficient associé à l'éducation est de 10\%.

    \subsection{Question 11}\label{question-11}

\begin{itemize}
\tightlist
\item
  Tester l'hypothèse jointe que le rendement de l'éducation est de 10\%
  et que celui de l'expérience professionnelle est de 5\%.
\end{itemize}

    Pour réaliser un test d'hypothèses jointes, on estime une statistique de
test de Fisher entre le modèle contraint et le modèle non-contrait selon
les hypothèses :

\[ H_0 : \beta_{educ} = 0.1, \beta_{exper} = 0.05 \]

On définit SSR comme la Somme des résidus au carré. On estime donc deux
modèles, un modèle non-contraint, et un modèle contraint.

\[ F_{educ + exper} = ( SSR_c - SSR_{nc} ) / ( ddl_c - ddl_{nc} ) \times ( ddl_{nc} ) / (SSR_{nc}) \]

On compare ensuite cette statistique de test à la valeur critique.

    \begin{Verbatim}[commandchars=\\\{\}]
{\color{incolor}In [{\color{incolor}37}]:} 
\end{Verbatim}


    \begin{Verbatim}[commandchars=\\\{\}]
La p\_value est de : 9.53570555850547e-11\%

    \end{Verbatim}

    La p-value est inférieure au seuil alpha, on va donc rejeter H0.

    \subsection{Question 12}\label{question-12}

\begin{itemize}
\tightlist
\item
  De combien augmente le salaire en pourcentage avec 10 années
  d'expérience ?
\end{itemize}

    \begin{Verbatim}[commandchars=\\\{\}]
{\color{incolor}In [{\color{incolor}38}]:} 
\end{Verbatim}


    \begin{Verbatim}[commandchars=\\\{\}]
                            OLS Regression Results                            
==============================================================================
Dep. Variable:                  lwage   R-squared:                       0.156
Model:                            OLS   Adj. R-squared:                  0.144
Method:                 Least Squares   F-statistic:                     12.92
Date:                Fri, 26 Apr 2019   Prob (F-statistic):           2.00e-13
Time:                        17:40:12   Log-Likelihood:                -431.92
No. Observations:                 428   AIC:                             877.8
Df Residuals:                     421   BIC:                             906.3
Df Model:                           6                                         
Covariance Type:            nonrobust                                         
==============================================================================
                 coef    std err          t      P>|t|      [0.025      0.975]
------------------------------------------------------------------------------
const         -0.3990      0.207     -1.927      0.055      -0.806       0.008
city           0.0353      0.070      0.503      0.616      -0.103       0.173
educ           0.1022      0.015      6.771      0.000       0.073       0.132
exper          0.0155      0.004      3.452      0.001       0.007       0.024
nwifeinc       0.0049      0.003      1.466      0.143      -0.002       0.011
kidslt6       -0.0453      0.085     -0.531      0.596      -0.213       0.122
kidsgt6       -0.0117      0.027     -0.434      0.664      -0.065       0.041
==============================================================================
Omnibus:                       79.542   Durbin-Watson:                   1.979
Prob(Omnibus):                  0.000   Jarque-Bera (JB):              287.193
Skew:                          -0.795   Prob(JB):                     4.33e-63
Kurtosis:                       6.685   Cond. No.                         178.
==============================================================================

Warnings:
[1] Standard Errors assume that the covariance matrix of the errors is correctly specified.

    \end{Verbatim}

    \begin{Verbatim}[commandchars=\\\{\}]
{\color{incolor}In [{\color{incolor}39}]:} 
\end{Verbatim}


    \begin{Verbatim}[commandchars=\\\{\}]
En 10 ans, le log du salaire augmente en moyenne de  15.49 \%

    \end{Verbatim}

    On peut alors déterminer un intervalle de confiance autour de cette
valeur :

    \begin{Verbatim}[commandchars=\\\{\}]
{\color{incolor}In [{\color{incolor}40}]:} 
\end{Verbatim}


    \begin{Verbatim}[commandchars=\\\{\}]
L'intervalle de confiance est : [ 6.6682\% ; 24.3076\%]

    \end{Verbatim}

    \subsection{Question 13}\label{question-13}

\begin{itemize}
\tightlist
\item
  Tester l'égalité des coefficients associés aux variables kidsgt6 et
  kidslt6. Interprétez.
\end{itemize}

    Afin de contraindre le modèle, nous créeons une variable
\texttt{kid\ =\ kidsgt6\ +\ kidslt6}. On ré-écrit l'équation et on
estime le coefficient associé à la variable 1 qui comprend l'effet de
\texttt{kidsgt6\ -\ kidslt6}. Si le coefficient n'est pas
significativement différent de 0, on ne rejette pas l'hypothèse nulle
d'égalité des deux coefficients.

    \begin{Verbatim}[commandchars=\\\{\}]
{\color{incolor}In [{\color{incolor}41}]:} \PY{n+nb}{print}\PY{p}{(}\PY{n}{p\PYZus{}val}\PY{p}{)}
\end{Verbatim}


    \begin{Verbatim}[commandchars=\\\{\}]
0.7102679748432641

    \end{Verbatim}

    On ne rejette pas l'hypothèse d'égalité des coefficients car la p-value
est de 0.71 et est supérieure à 5\%.

    \subsection{Question 14}\label{question-14}

\begin{itemize}
\tightlist
\item
  En utilisant le modèle de la question 7, faire le test
  d'hétéroscédasticité de forme linéaire en donnant la p-valeur. 
\end{itemize}

On réalise un F-test dans lequel on inclut uniquement la constante dans
le modèle contraint. Ainsi, on teste le fait qu'il n'y ait pas de
différence entre le modèle contraint et non-contraint. H0 corresond donc
à une hypothèse d'homoscedasticité.

    \begin{Verbatim}[commandchars=\\\{\}]
{\color{incolor}In [{\color{incolor}43}]:} \PY{n+nb}{print}\PY{p}{(}\PY{n}{p\PYZus{}val}\PY{p}{)}
\end{Verbatim}


    \begin{Verbatim}[commandchars=\\\{\}]
0.09130097553302419

    \end{Verbatim}

    On obtient une p-value de 9\%, qui implique un rejet de l'hypothèse à
10\% mais une non-rejet à 5\%.

    \begin{itemize}
\tightlist
\item
  Corriger le problème par rapport à la variable la plus importante en
  utilisant la méthode des MCG. 
\end{itemize}

    On applique la même méthode avec un estimateur des Moindres Carrés
Généralisés cette fois-ci, en calculant le log des résidus, afin d'en
déduire des poids pour le modèle Feasible Weighted Least Squares.

    \begin{Verbatim}[commandchars=\\\{\}]
{\color{incolor}In [{\color{incolor}47}]:} \PY{n+nb}{print}\PY{p}{(}\PY{n}{p\PYZus{}val}\PY{p}{)}
\end{Verbatim}


    \begin{Verbatim}[commandchars=\\\{\}]
0.6271456383048878

    \end{Verbatim}

    La p-value est désormais de 63\%, ce qui implique que l'on ne peut pas
rejeter l'hypothèse d'homoscedasticité des résidus. On a bien corrigé le
problème d'hétéroscedasticité.

    \begin{itemize}
\tightlist
\item
  Comparer les écarts-types des coefficients estimés avec ceux obtenus à
  la question 7. Commenter.
\end{itemize}

    \begin{Verbatim}[commandchars=\\\{\}]
{\color{incolor}In [{\color{incolor}48}]:} 
\end{Verbatim}


    \begin{Verbatim}[commandchars=\\\{\}]
                            OLS Regression Results                            
==============================================================================
Dep. Variable:                      y   R-squared:                       0.008
Model:                            OLS   Adj. R-squared:                 -0.006
Method:                 Least Squares   F-statistic:                    0.5795
Date:                Fri, 26 Apr 2019   Prob (F-statistic):              0.747
Time:                        17:40:12   Log-Likelihood:                -2120.8
No. Observations:                 428   AIC:                             4256.
Df Residuals:                     421   BIC:                             4284.
Df Model:                           6                                         
Covariance Type:            nonrobust                                         
==============================================================================
                 coef    std err          t      P>|t|      [0.025      0.975]
------------------------------------------------------------------------------
const         13.1731     16.795      0.784      0.433     -19.838      46.185
city          -3.5689      8.093     -0.441      0.659     -19.476      12.338
educ           0.0396      1.193      0.033      0.974      -2.305       2.384
exper         -0.2592      0.213     -1.216      0.225      -0.678       0.160
nwifeinc       0.0534      0.206      0.260      0.795      -0.351       0.457
kidslt6       -0.5866      6.666     -0.088      0.930     -13.689      12.515
kidsgt6       -1.2778      1.231     -1.038      0.300      -3.698       1.143
==============================================================================
Omnibus:                      642.845   Durbin-Watson:                   2.023
Prob(Omnibus):                  0.000   Jarque-Bera (JB):            94065.650
Skew:                           8.249   Prob(JB):                         0.00
Kurtosis:                      73.729   Cond. No.                         297.
==============================================================================

Warnings:
[1] Standard Errors assume that the covariance matrix of the errors is correctly specified.

    \end{Verbatim}

    Les écarts-types sont beaucoup plus élevés avec avec ce modèle.

    \subsection{Question 15}\label{question-15}

\begin{itemize}
\tightlist
\item
  Tester le changement de structure de la question 8 entre les femmes
  qui ont moins de 30 ans, entre 30 et 43 ans, plus de 43 ans (3 groupes
  mutuellement exclusifs). Donnez les p-valeurs. 
\end{itemize}

    Nous avons imaginé deux approches pour cette question : - Une analyse de
variance (One-way ANOVA) - Un test de Chow entre les différents groupes

    Commençons par l'ANOVA :

    \begin{Verbatim}[commandchars=\\\{\}]
{\color{incolor}In [{\color{incolor}49}]:} 
\end{Verbatim}


\begin{Verbatim}[commandchars=\\\{\}]
{\color{outcolor}Out[{\color{outcolor}49}]:} F\_onewayResult(statistic=0.21829228903413253, pvalue=0.8039805155171003)
\end{Verbatim}
            
    La p-value étant de 0.8039, on ne peut pas conclure à un effet
significatif de l'âge.

    On peut également tester nos conclusions à cette question en réalisant
un test de Chow entre : - le groupe \textless{}30 et le reste - le
groupe \textless{}43 et le reste - le groupe entre 30 et 43 et le reste

    On construit la statistique de test sous le test de Chow de la manière
suivante :

\[ F_{chow} = ( SSR - (SSR_{gr1} + SSR_{gr2}) ) / ( ddl - (ddl_{gr1} + ddl_{gr2}) ) \times ( ddl_{gr1} + ddl_{gr2} ) / ( (SSR_{gr1} + SSR_{gr2} ) \]

    \paragraph{Groupe \textless{}30 vs. reste}\label{groupe-30-vs.-reste}

    \begin{Verbatim}[commandchars=\\\{\}]
{\color{incolor}In [{\color{incolor}50}]:} \PY{n+nb}{print}\PY{p}{(}\PY{n}{p\PYZus{}val}\PY{p}{)}
\end{Verbatim}


    \begin{Verbatim}[commandchars=\\\{\}]
0.7946563995696743

    \end{Verbatim}

    \paragraph{Groupe \textless{}43 vs. reste}\label{groupe-43-vs.-reste}

    \begin{Verbatim}[commandchars=\\\{\}]
{\color{incolor}In [{\color{incolor}51}]:} \PY{n+nb}{print}\PY{p}{(}\PY{n}{p\PYZus{}val}\PY{p}{)}
\end{Verbatim}


    \begin{Verbatim}[commandchars=\\\{\}]
0.30997341357260577

    \end{Verbatim}

    \paragraph{Groupe 30-43 vs. reste}\label{groupe-30-43-vs.-reste}

    \begin{Verbatim}[commandchars=\\\{\}]
{\color{incolor}In [{\color{incolor}52}]:} \PY{n+nb}{print}\PY{p}{(}\PY{n}{p\PYZus{}val}\PY{p}{)}
\end{Verbatim}


    \begin{Verbatim}[commandchars=\\\{\}]
0.6109025332616165

    \end{Verbatim}

    Toutes les p-values sont supérieures au seuil alpha de 5\%. Ainsi, on ne
peut pas rejeter l'hypothèse H0 qu'il n'y a aucun changement de régime.
Cela conforte les résultats de notre ANOVA.

    \subsection{Question 16}\label{question-16}

\begin{itemize}
\tightlist
\item
  A partir de la variable kidslt6, créer un ensemble de variables
  binaires pour le nombre d'enfants de moins de 6 ans. 
\end{itemize}

    On crée les variables suivantes : - \texttt{kidslt6\_1} pour les femmes
ayant 1 enfant de moins de 6 ans - \texttt{kidslt6\_2} pour les femmes
ayant 2 enfants de moins de 6 ans

On supprime la variable \texttt{kidslt\_6}.

    \begin{itemize}
\tightlist
\item
  Refaire la question 8 avec ces variables et en utilisant comme
  référence les femmes qui ont des enfants de plus de 6 ans. (Faire la
  régression de lwage sur une constante, city, educ, exper, nwifeinc,
  kidslt6, kidsgt6.)
\end{itemize}

    \begin{Verbatim}[commandchars=\\\{\}]
{\color{incolor}In [{\color{incolor}56}]:} 
\end{Verbatim}


    \begin{center}
    \adjustimage{max size={0.9\linewidth}{0.9\paperheight}}{output_106_0.png}
    \end{center}
    { \hspace*{\fill} \\}
    
    \begin{itemize}
\tightlist
\item
  Ces catégories sont-elles mutuellement exclusives ?
\end{itemize}

    Telles que les variables sont construites, il n'est pas possible d'avoir
1 enfant de moins de 6 ans, et également 2 enfant de moins de 6 ans par
exemple. Pour cette raison, les catégories sont mutuellement exclusives.
On supprime par ailleurs la variable correspondant à 0 enfants, car si
une femme n'a ni 1 enfant, ni 2 (dans ce set de données), alors elle n'a
pas d'enfant de moins de 6 ans. Par ailleurs, on crée une variable gt6
qui vaut 1 lorsque la femme n'a pas d'enfant de plus de 6 ans, par
rapport au cas de base où elle a un enfant de plus de 6 ans.

    \begin{itemize}
\tightlist
\item
  Interprétez les paramètres associés aux variables binaires.
\end{itemize}

    \begin{Verbatim}[commandchars=\\\{\}]
{\color{incolor}In [{\color{incolor}57}]:} 
\end{Verbatim}


    \begin{Verbatim}[commandchars=\\\{\}]
                            OLS Regression Results                            
==============================================================================
Dep. Variable:                  lwage   R-squared:                       0.155
Model:                            OLS   Adj. R-squared:                  0.141
Method:                 Least Squares   F-statistic:                     11.03
Date:                Fri, 26 Apr 2019   Prob (F-statistic):           8.15e-13
Time:                        17:40:13   Log-Likelihood:                -432.00
No. Observations:                 428   AIC:                             880.0
Df Residuals:                     420   BIC:                             912.5
Df Model:                           7                                         
Covariance Type:            nonrobust                                         
==============================================================================
                 coef    std err          t      P>|t|      [0.025      0.975]
------------------------------------------------------------------------------
const         -0.4314      0.192     -2.245      0.025      -0.809      -0.054
city           0.0366      0.070      0.520      0.603      -0.102       0.175
educ           0.1028      0.015      6.830      0.000       0.073       0.132
exper          0.0159      0.005      3.503      0.001       0.007       0.025
nwifeinc       0.0049      0.003      1.465      0.144      -0.002       0.011
kidslt6\_1     -0.0560      0.108     -0.519      0.604      -0.268       0.156
kidslt6\_2     -0.0662      0.259     -0.256      0.798      -0.575       0.442
gt6            0.0114      0.074      0.153      0.878      -0.134       0.157
==============================================================================
Omnibus:                       79.748   Durbin-Watson:                   1.978
Prob(Omnibus):                  0.000   Jarque-Bera (JB):              286.571
Skew:                          -0.798   Prob(JB):                     5.91e-63
Kurtosis:                       6.677   Cond. No.                         221.
==============================================================================

Warnings:
[1] Standard Errors assume that the covariance matrix of the errors is correctly specified.

    \end{Verbatim}

    Le coefficient associé à la variable binaire indique un que si la
condition pour satisfaire la variable est respectée, on observe un
changement de y égal au coefficient par rapport au cas par défaut
pré-défini.

    \begin{itemize}
\tightlist
\item
  Faire le test de non significativité de l'ensemble des variables
  binaires. Donnez les p-valeurs.
\end{itemize}

    Aucune des variables ne semble être significative, car les p-valeurs
sont toutes élevées, et on ne peut pas rejeter l'hypothèse nulle de
non-significativité. Les coefficients, bien que non-significatifs,
semblent indiquer que le fait de n'avoir aucun enfant est un plus pour
le salaire.

    \begin{Verbatim}[commandchars=\\\{\}]
{\color{incolor}In [{\color{incolor}58}]:} \PY{n+nb}{print}\PY{p}{(}\PY{n}{p\PYZus{}val}\PY{p}{)}
\end{Verbatim}


    \begin{Verbatim}[commandchars=\\\{\}]
0.14344360131309886

    \end{Verbatim}

    La p-value de 14.3 \% indique que l'on ne peut pas rejeter l'hypothèse
nulle de non-significativité de l'ensemble des variables binaires.

    \subsection{Question 17}\label{question-17}

A partir de l'échantillon global, faire une régression de inlf sur une
constante, city, educ, age, kidslt6, kidsgt6.

    \begin{Verbatim}[commandchars=\\\{\}]
{\color{incolor}In [{\color{incolor}59}]:} 
\end{Verbatim}


    \begin{center}
    \adjustimage{max size={0.9\linewidth}{0.9\paperheight}}{output_117_0.png}
    \end{center}
    { \hspace*{\fill} \\}
    
    \begin{Verbatim}[commandchars=\\\{\}]
{\color{incolor}In [{\color{incolor}60}]:} 
\end{Verbatim}


    \begin{Verbatim}[commandchars=\\\{\}]
                            OLS Regression Results                            
==============================================================================
Dep. Variable:                   inlf   R-squared:                       0.124
Model:                            OLS   Adj. R-squared:                  0.118
Method:                 Least Squares   F-statistic:                     21.20
Date:                Fri, 26 Apr 2019   Prob (F-statistic):           7.29e-20
Time:                        17:40:13   Log-Likelihood:                -489.44
No. Observations:                 753   AIC:                             990.9
Df Residuals:                     747   BIC:                             1019.
Df Model:                           5                                         
Covariance Type:            nonrobust                                         
==============================================================================
                 coef    std err          t      P>|t|      [0.025      0.975]
------------------------------------------------------------------------------
const          0.7076      0.162      4.365      0.000       0.389       1.026
city          -0.0341      0.036     -0.944      0.346      -0.105       0.037
educ           0.0434      0.008      5.656      0.000       0.028       0.058
age           -0.0130      0.003     -5.081      0.000      -0.018      -0.008
kidslt6       -0.3075      0.036     -8.498      0.000      -0.378      -0.236
kidsgt6       -0.0173      0.014     -1.231      0.219      -0.045       0.010
==============================================================================
Omnibus:                        7.463   Durbin-Watson:                   0.246
Prob(Omnibus):                  0.024   Jarque-Bera (JB):               76.244
Skew:                          -0.245   Prob(JB):                     2.78e-17
Kurtosis:                       1.520   Cond. No.                         432.
==============================================================================

Warnings:
[1] Standard Errors assume that the covariance matrix of the errors is correctly specified.

    \end{Verbatim}

    \begin{itemize}
\tightlist
\item
  Interprétez les coefficients estimés.
\end{itemize}

    Les coefficients estimés indiquent l'impact, estimé en pourcents, de
l'augmentation d'une des variables explicatives sur la probabilité
d'observation d'un 1 ou d'un 0 sur la variables \texttt{inlf}. Cette
interprétation est statistiquement fausse, mais permet une bonne
explicabilité du résultat dans les cadres de classification binaire.

D'après ce modèle, l'éducation, l'âge et le nombre d'enfants de moins de
6 ans sont des variables significatives.

    \subsection{Question 18}\label{question-18}

\begin{itemize}
\tightlist
\item
  Estimer le modèle probit de inlf sur une constante, city, educ, age,
  kidslt6, kidsgt6. 
\end{itemize}

    \begin{Verbatim}[commandchars=\\\{\}]
{\color{incolor}In [{\color{incolor}62}]:} 
\end{Verbatim}


    \begin{Verbatim}[commandchars=\\\{\}]
Optimization terminated successfully.
         Current function value: 0.617205
         Iterations 5
                          Probit Regression Results                           
==============================================================================
Dep. Variable:                      y   No. Observations:                  753
Model:                         Probit   Df Residuals:                      747
Method:                           MLE   Df Model:                            5
Date:                Fri, 26 Apr 2019   Pseudo R-squ.:                 0.09734
Time:                        17:40:13   Log-Likelihood:                -464.76
converged:                       True   LL-Null:                       -514.87
                                        LLR p-value:                 4.714e-20
==============================================================================
                 coef    std err          z      P>|z|      [0.025      0.975]
------------------------------------------------------------------------------
const          0.6050      0.467      1.297      0.195      -0.309       1.520
city          -0.0863      0.102     -0.842      0.400      -0.287       0.115
educ           0.1234      0.023      5.469      0.000       0.079       0.168
age           -0.0375      0.007     -5.008      0.000      -0.052      -0.023
kidslt6       -0.8846      0.112     -7.882      0.000      -1.105      -0.665
kidsgt6       -0.0542      0.040     -1.351      0.177      -0.133       0.024
==============================================================================

    \end{Verbatim}

    \begin{itemize}
\tightlist
\item
  Faire le test de non significativité jointes des coefficients associés
  à kidslt6 et à kidsgt6. 
\end{itemize}

    Etant donné qu'il est demandé de réaliser le test par approche de
vraisemblance à la question 20, nous en concluons qu'il est demandé ici
d'appliquer un F-Test, bien qu'il ne soit pas adapté au modèle probit.

    \begin{Verbatim}[commandchars=\\\{\}]
{\color{incolor}In [{\color{incolor}63}]:} \PY{n+nb}{print}\PY{p}{(}\PY{n}{p\PYZus{}val}\PY{p}{)}
\end{Verbatim}


    \begin{Verbatim}[commandchars=\\\{\}]
Optimization terminated successfully.
         Current function value: 0.617205
         Iterations 5
Optimization terminated successfully.
         Current function value: 0.663577
         Iterations 4
3.3306690738754696e-16

    \end{Verbatim}

    La p-value est quasiment à 0, ce qui implique que l'on peut rejeter
l'hypothèse de non-significativité jointe des coefficients associés à
kidsle6 et kidsgt6 à 5\%.

    \begin{itemize}
\tightlist
\item
  Comparez le résultat du test à celui de la question 13 
\end{itemize}

    A la question 13, on ne rejettait pas l'hypothèse d'égalité des deux
coefficients. Ici, on rejette l'hypothèse de non-significativité jointe
des coefficients.

    \subsection{Question 19}\label{question-19}

\begin{itemize}
\tightlist
\item
  Calculer les effets partiels pour l'ensemble des variables
  explicatives : dp(y=1)/dxk (k = 1, ..., K), où K est le nombre de
  variables explicatives. 
\end{itemize}

    \[ \delta p(x) / \delta x_j = g(\beta_0 + X \beta) \beta_j \]

\[ g(z) = \delta G / \delta z (z) \]

    \begin{Verbatim}[commandchars=\\\{\}]
{\color{incolor}In [{\color{incolor}64}]:} 
\end{Verbatim}


    \begin{Verbatim}[commandchars=\\\{\}]
Optimization terminated successfully.
         Current function value: 0.617205
         Iterations 5
       Probit Marginal Effects       
=====================================
Dep. Variable:                   inlf
Method:                          dydx
At:                           overall
==============================================================================
                dy/dx    std err          z      P>|z|      [0.025      0.975]
------------------------------------------------------------------------------
x1            -0.0304      0.036     -0.843      0.399      -0.101       0.040
x2             0.0435      0.007      5.811      0.000       0.029       0.058
x3            -0.0132      0.003     -5.264      0.000      -0.018      -0.008
x4            -0.3116      0.035     -9.006      0.000      -0.379      -0.244
x5            -0.0191      0.014     -1.355      0.175      -0.047       0.009
==============================================================================

    \end{Verbatim}

    \begin{itemize}
\tightlist
\item
  Comparer vos résultats à ceux obtenus à la question 17. Commentez.
\end{itemize}

    On remarque que globalement, les effets ont le même ordre de grandeur.
Cela signifie notamment que l'absence de transformation ne pénalise pas
trop la qualité du modèle. Cela conduit à la significativité des mêmes
paramètres. Cependant, l'interprétation dans les modèles Probit ou Logit
n'est plus aussi directe.

    \subsection{Question 20}\label{question-20}

\begin{itemize}
\tightlist
\item
  Faire le test de non significativité jointes des coefficients associés
  à kidslt6 et à kidsgt6 en utilisant la méthode du rapport de
  vraisemblance. 
\end{itemize}

    \begin{Verbatim}[commandchars=\\\{\}]
{\color{incolor}In [{\color{incolor}68}]:} \PY{n+nb}{print}\PY{p}{(}\PY{n}{LR}\PY{p}{,} \PY{n}{p}\PY{p}{)}
\end{Verbatim}


    \begin{Verbatim}[commandchars=\\\{\}]
69.8366275999017 6.841778332359197e-16

    \end{Verbatim}

    Le rapport de vraisemblance vaut 69, et la p-value est très proche de 0.

    \begin{itemize}
\tightlist
\item
  Comparez aux résultats de la question 18. 
\end{itemize}

    La p-value est inférieure à 5\%, on peut donc rejeter l'hypothèse nulle
de non-significativité jointe. On remarque que la conclusion est
différente de la question 18, car les tests de Fisher ne sont notamment
pas adaptés aux modèles probit ou logit.

    \section{Partie II - Séries
Temporelles}\label{partie-ii---suxe9ries-temporelles}

    \subsection{Question 1}\label{question-1}

\begin{itemize}
\tightlist
\item
  Importer les données du fichier quarterly.xls
\end{itemize}

    \begin{Verbatim}[commandchars=\\\{\}]
{\color{incolor}In [{\color{incolor}69}]:} \PY{n}{df}\PY{o}{.}\PY{n}{head}\PY{p}{(}\PY{p}{)}
\end{Verbatim}


\begin{Verbatim}[commandchars=\\\{\}]
{\color{outcolor}Out[{\color{outcolor}69}]:}         DATE   FFR  Tbill  Tb1yr    r5   r10  PPINSA  Finished    CPI  \textbackslash{}
         0 1960-01-01  3.93   3.87   4.57  4.64  4.49   31.67     33.20  29.40   
         1 1960-04-01  3.70   2.99   3.87  4.30  4.26   31.73     33.40  29.57   
         2 1960-07-01  2.94   2.36   3.07  3.67  3.83   31.63     33.43  29.59   
         3 1960-10-01  2.30   2.31   2.99  3.75  3.89   31.70     33.67  29.78   
         4 1961-01-01  2.00   2.35   2.87  3.64  3.79   31.80     33.63  29.84   
         
            CPICORE   M1NSA   M2SA   M2NSA  Unemp  IndProd    RGDP  Potent  Deflator  \textbackslash{}
         0    18.92  140.53  896.1  299.40   5.13    23.93  2845.3  2824.2    18.521   
         1    19.00  138.40  903.3  300.03   5.23    23.41  2832.0  2851.2    18.579   
         2    19.07  139.60  919.4  305.50   5.53    23.02  2836.6  2878.7    18.648   
         3    19.14  142.67  932.8  312.30   6.27    22.47  2800.2  2906.7    18.700   
         4    19.17  142.23  948.9  317.10   6.80    22.13  2816.9  2934.8    18.743   
         
              Curr  
         0  31.830  
         1  31.862  
         2  32.217  
         3  32.624  
         4  32.073  
\end{Verbatim}
            
    \subsection{Question 2}\label{question-2}

\begin{itemize}
\tightlist
\item
  Calculer inf, le taux d'inflation à partir de la variable CPI. 
\end{itemize}

    Le taux d'inflation est donné par :

\$ df\_\{inf\} = ( df\_\{CPI\_t\} - df\_\{CPI\_\{t-1\}\} ) / (
df\_\{CPI\_t\} ) \$

    \begin{itemize}
\tightlist
\item
  Faire un graphique dans le temps de inf. 
\end{itemize}

    \begin{Verbatim}[commandchars=\\\{\}]
{\color{incolor}In [{\color{incolor}73}]:} 
\end{Verbatim}


    \begin{center}
    \adjustimage{max size={0.9\linewidth}{0.9\paperheight}}{output_145_0.png}
    \end{center}
    { \hspace*{\fill} \\}
    
    \begin{itemize}
\tightlist
\item
  Commentez.
\end{itemize}

    L'inflation était élevée jusqu'en 1981, puis a chuté peu avant les
années 1990. On remarque une déflation sur la période de crise
financière de 2008. Le régime d'inflation semblait plus instable avant
les années 90, puis contrôlé entre 1990 et 2008, avant la crise
financière.

    \subsection{Question 3}\label{question-3}

\begin{itemize}
\tightlist
\item
  Interpréter l'autocorrélogramme et l'autocorrélogrammes partiels de
  inf. 
\end{itemize}

    \begin{Verbatim}[commandchars=\\\{\}]
{\color{incolor}In [{\color{incolor}74}]:} 
\end{Verbatim}


    \begin{center}
    \adjustimage{max size={0.9\linewidth}{0.9\paperheight}}{output_149_0.png}
    \end{center}
    { \hspace*{\fill} \\}
    
    \begin{itemize}
\tightlist
\item
  L'autocorrélogramme indique que l'autocorrelation diminue avec le
  temps comme dans un processus ARMA(p,q)
\item
  L'autocorrélogramme partiel oscille autour de 0 comme dans un
  processus de type Moving Average MA(1)
\end{itemize}

Cela signifie notamment qu'il existe une influence non-négligeable du
passé pour la détermination des valeurs présentes. On peut supposer que
la série n'est pas stationnaire.

    \begin{itemize}
\tightlist
\item
  Quelle est la différence entre ces deux graphiques ?
\end{itemize}

    L'autocorrélogramme donne l'influence d'une série à un temps t-k dans le
passé sur la valeur de la série au temps t, indépendemment du reste des
observations. L'autocorrélocgramme partiel réalise la régression de
toutes du la valeur présente sur toutes les valeurs passées jusqu'au
temps t-k. Ainsi, on identifie les effets joints des différents années.

    \subsection{Question 4}\label{question-4}

\begin{itemize}
\tightlist
\item
  Quelle est la différence entre la stationnarité et l'ergodicité ?
  Pourquoi a-t-on besoin de ces deux conditions?
\end{itemize}

    \begin{itemize}
\tightlist
\item
  La stationarité est un état atteint lorsque ys, ys+1, ys+2 ... ne
  dépend pas de s. En d'autres termes, le futur et le présent sont
  relativement similaires.
\item
  L'ergodicité est le processus par lequel l'on oublie les conditions
  initiales, et l'autocorrélation d''ordre k tend vers 0 quand k tend
  vers l'infini.
\item
  Ces deux conditions sont nécessaires afin de pouvoir appliquer le
  théorème d'ergodicité qui garantit que la moyenne des valeurs des
  observations tend vers l'espérance de la série. Autrement dit, on
  s'assure que la série ne diverge pas de son espérance.
\end{itemize}

    \begin{itemize}
\tightlist
\item
  Expliquez le terme "spurious regression".
\end{itemize}

    \begin{itemize}
\tightlist
\item
  Le terme "spurious regression" se refère au fait que deux variables
  soient corrélées mais qu'aucun lien de causalité ne puis pour autant
  être établi entre les variables.
\end{itemize}

    \subsection{Question 5}\label{question-5}

\begin{itemize}
\tightlist
\item
  Proposer une modélisation AR(p) de inf, en utilisant tous les outils
  vus au cours.
\end{itemize}

    \begin{Verbatim}[commandchars=\\\{\}]
{\color{incolor}In [{\color{incolor}75}]:} 
\end{Verbatim}


    \begin{Verbatim}[commandchars=\\\{\}]
                              ARMA Model Results                              
==============================================================================
Dep. Variable:                    inf   No. Observations:                  211
Model:                     ARMA(5, 0)   Log Likelihood                 832.908
Method:                       css-mle   S.D. of innovations              0.005
Date:                Fri, 26 Apr 2019   AIC                          -1651.816
Time:                        17:40:16   BIC                          -1628.353
Sample:                             0   HQIC                         -1642.331
                                                                              
==============================================================================
                 coef    std err          z      P>|z|      [0.025      0.975]
------------------------------------------------------------------------------
const          0.0093      0.002      4.018      0.000       0.005       0.014
ar.L1.inf      0.6044      0.068      8.857      0.000       0.471       0.738
ar.L2.inf     -0.0588      0.080     -0.737      0.462      -0.215       0.098
ar.L3.inf      0.3353      0.076      4.389      0.000       0.186       0.485
ar.L4.inf     -0.1316      0.080     -1.651      0.100      -0.288       0.025
ar.L5.inf      0.1190      0.068      1.742      0.083      -0.015       0.253
                                    Roots                                    
=============================================================================
                  Real          Imaginary           Modulus         Frequency
-----------------------------------------------------------------------------
AR.1            1.0786           -0.0000j            1.0786           -0.0000
AR.2           -0.8581           -1.1850j            1.4631           -0.3497
AR.3           -0.8581           +1.1850j            1.4631            0.3497
AR.4            0.8714           -1.6967j            1.9074           -0.1745
AR.5            0.8714           +1.6967j            1.9074            0.1745
-----------------------------------------------------------------------------

    \end{Verbatim}

    Afin d'identifier le paramètre p optimal, on cherche à minimiser le
critère Akaike Information Criterion (AIC).

    \begin{Verbatim}[commandchars=\\\{\}]
{\color{incolor}In [{\color{incolor}77}]:} 
\end{Verbatim}


    \begin{center}
    \adjustimage{max size={0.9\linewidth}{0.9\paperheight}}{output_160_0.png}
    \end{center}
    { \hspace*{\fill} \\}
    
    Le critère AIC est minimisé pour une valeur de p valant 3. On peut
s'intéresser à la performance en prédiction d'un tel modèle.

    \begin{Verbatim}[commandchars=\\\{\}]
{\color{incolor}In [{\color{incolor}79}]:} 
\end{Verbatim}


    \begin{center}
    \adjustimage{max size={0.9\linewidth}{0.9\paperheight}}{output_162_0.png}
    \end{center}
    { \hspace*{\fill} \\}
    
    \begin{Verbatim}[commandchars=\\\{\}]
{\color{incolor}In [{\color{incolor}80}]:} 
\end{Verbatim}


    \begin{Verbatim}[commandchars=\\\{\}]
Lag: 3
Coefficients: [ 0.00962934  0.57949951 -0.01686016  0.29775042]

    \end{Verbatim}

    \subsection{Question 6}\label{question-6}

\begin{itemize}
\tightlist
\item
  Estimer le modèle de la courbe de Philips qui explique le taux de
  chômage (Unemp) en fonction du taux d'inflation courant et une
  constante.
\end{itemize}

    \begin{Verbatim}[commandchars=\\\{\}]
{\color{incolor}In [{\color{incolor}81}]:} 
\end{Verbatim}


    \begin{center}
    \adjustimage{max size={0.9\linewidth}{0.9\paperheight}}{output_165_0.png}
    \end{center}
    { \hspace*{\fill} \\}
    
    \begin{Verbatim}[commandchars=\\\{\}]
{\color{incolor}In [{\color{incolor}82}]:} 
\end{Verbatim}


    \begin{Verbatim}[commandchars=\\\{\}]
                            OLS Regression Results                            
==============================================================================
Dep. Variable:                  Unemp   R-squared:                       0.000
Model:                            OLS   Adj. R-squared:                 -0.005
Method:                 Least Squares   F-statistic:                   0.01214
Date:                Fri, 26 Apr 2019   Prob (F-statistic):              0.912
Time:                        17:41:07   Log-Likelihood:                -400.28
No. Observations:                 211   AIC:                             804.6
Df Residuals:                     209   BIC:                             811.3
Df Model:                           1                                         
Covariance Type:            nonrobust                                         
==============================================================================
                 coef    std err          t      P>|t|      [0.025      0.975]
------------------------------------------------------------------------------
const          6.0708      0.181     33.576      0.000       5.714       6.427
inf            0.0159      0.144      0.110      0.912      -0.269       0.301
==============================================================================
Omnibus:                       13.872   Durbin-Watson:                   0.044
Prob(Omnibus):                  0.001   Jarque-Bera (JB):               15.356
Skew:                           0.660   Prob(JB):                     0.000463
Kurtosis:                       2.937   Cond. No.                         2.99
==============================================================================

Warnings:
[1] Standard Errors assume that the covariance matrix of the errors is correctly specified.

    \end{Verbatim}

    Le coefficient semble indiquer qu'une augmentation de 1\% de l'inflation
augmente le chômage de 0.016\%. Il ne semble pas que la variable CPI
soit significative à 5\% étant donné que la p-value est supérieure à ce
seuil.

    \subsection{Question 7}\label{question-7}

\begin{itemize}
\tightlist
\item
  Tester l'autocorrélation des erreurs.
\end{itemize}

    On peut commencer par représenter graphiquement la valeur des résidus :

    \begin{Verbatim}[commandchars=\\\{\}]
{\color{incolor}In [{\color{incolor}84}]:} 
\end{Verbatim}


    \begin{center}
    \adjustimage{max size={0.9\linewidth}{0.9\paperheight}}{output_170_0.png}
    \end{center}
    { \hspace*{\fill} \\}
    
    \begin{Verbatim}[commandchars=\\\{\}]
{\color{incolor}In [{\color{incolor}85}]:} 
\end{Verbatim}


    \begin{center}
    \adjustimage{max size={0.9\linewidth}{0.9\paperheight}}{output_171_0.png}
    \end{center}
    { \hspace*{\fill} \\}
    
    Pour tester l'autocorrélation des erreurs, on peut appliquer le test de
Durbin Watson sur les résidus. Les résidus, après régression, devraient
avoir une moyenne de 0 et pas de corrélation. Ainsi, si le test
d'autocorrélation de Durbin Watson vaut environ 2, il n'y a pas
d'autocorrélation, s'il vaut proche de 0, il y a une autocorrélation
positive, et s'il vaut environ 4, l'autocorrélation est négative.

    \begin{Verbatim}[commandchars=\\\{\}]
{\color{incolor}In [{\color{incolor}86}]:} \PY{c+c1}{\PYZsh{} Durbin Watson test}
         \PY{n}{durbin\PYZus{}watson}\PY{p}{(}\PY{n}{df}\PY{o}{.}\PY{n}{res}\PY{p}{)}
\end{Verbatim}


\begin{Verbatim}[commandchars=\\\{\}]
{\color{outcolor}Out[{\color{outcolor}86}]:} 0.044194128074712014
\end{Verbatim}
            
    Ici, la valeur du test de DW est proche de 0. On en conclut que
l'autocorrélation doit être positive. Si l'on regarde
l'autocorrélogramme, on confirme ce constat.

    \begin{Verbatim}[commandchars=\\\{\}]
{\color{incolor}In [{\color{incolor}87}]:} 
\end{Verbatim}


    \begin{center}
    \adjustimage{max size={0.9\linewidth}{0.9\paperheight}}{output_175_0.png}
    \end{center}
    { \hspace*{\fill} \\}
    
    \subsection{Question 8}\label{question-8}

\begin{itemize}
\tightlist
\item
  Corriger l'autocorrélation des erreurs par la méthode vue en cours.
\end{itemize}

    Afin de corriger l'autocorrélation, on peut estimer un nouveau modèle
égal à :

\[ y_t - \rho y_{t-1} = \tilde{y}_t = (1-\rho) \beta_0 + \beta_1 \tilde{X}_t + \epsilon_t \]

    \begin{Verbatim}[commandchars=\\\{\}]
{\color{incolor}In [{\color{incolor}88}]:} \PY{c+c1}{\PYZsh{} Estimation de rho}
         \PY{n+nb}{print}\PY{p}{(}\PY{n}{rho}\PY{p}{)}
\end{Verbatim}


    \begin{Verbatim}[commandchars=\\\{\}]
0.979923779970792

    \end{Verbatim}

    On estime maintenant avec le nouveau modèle laggé la statistique de
Durbin Watson :

    \begin{Verbatim}[commandchars=\\\{\}]
{\color{incolor}In [{\color{incolor}92}]:} \PY{n}{durbin\PYZus{}watson}\PY{p}{(}\PY{n}{df}\PY{p}{[}\PY{l+s+s1}{\PYZsq{}}\PY{l+s+s1}{res\PYZus{}c}\PY{l+s+s1}{\PYZsq{}}\PY{p}{]}\PY{p}{)}
\end{Verbatim}


\begin{Verbatim}[commandchars=\\\{\}]
{\color{outcolor}Out[{\color{outcolor}92}]:} 0.718292189000225
\end{Verbatim}
            
    Le test de Durbin Watson est plus proche de 2. On a donc corrigé une
partie de l'autocorrélation.

    \begin{Verbatim}[commandchars=\\\{\}]
{\color{incolor}In [{\color{incolor}93}]:} 
\end{Verbatim}


    \begin{center}
    \adjustimage{max size={0.9\linewidth}{0.9\paperheight}}{output_182_0.png}
    \end{center}
    { \hspace*{\fill} \\}
    
    Nous pouvons donc utiliser ces valeurs corrigées pour la suite. (Nous ne
savons pas si c'est ce qui était attendu, mais cela nous paraissait
logique).

    \subsection{Question 9}\label{question-9}

\begin{itemize}
\tightlist
\item
  Tester la stabilité de la relation chômage-inflation sur deux
  sous-périodes de taille identique.
\end{itemize}

    On peut commencer par regarder les paramètres lorsque l'on coupe notre
période en 2 :

    \begin{Verbatim}[commandchars=\\\{\}]
{\color{incolor}In [{\color{incolor}94}]:} 
\end{Verbatim}


    \begin{Verbatim}[commandchars=\\\{\}]
                            OLS Regression Results                            
==============================================================================
Dep. Variable:                      y   R-squared:                       0.007
Model:                            OLS   Adj. R-squared:                 -0.003
Method:                 Least Squares   F-statistic:                    0.7071
Date:                Fri, 26 Apr 2019   Prob (F-statistic):              0.402
Time:                        17:41:26   Log-Likelihood:                -204.20
No. Observations:                 105   AIC:                             412.4
Df Residuals:                     103   BIC:                             417.7
Df Model:                           1                                         
Covariance Type:            nonrobust                                         
==============================================================================
                 coef    std err          t      P>|t|      [0.025      0.975]
------------------------------------------------------------------------------
const          5.9377      0.287     20.662      0.000       5.368       6.508
x1            15.6225     18.579      0.841      0.402     -21.224      52.469
==============================================================================
Omnibus:                        3.884   Durbin-Watson:                   0.059
Prob(Omnibus):                  0.143   Jarque-Bera (JB):                3.782
Skew:                           0.462   Prob(JB):                        0.151
Kurtosis:                       2.889   Cond. No.                         111.
==============================================================================

Warnings:
[1] Standard Errors assume that the covariance matrix of the errors is correctly specified.

    \end{Verbatim}

    \begin{Verbatim}[commandchars=\\\{\}]
{\color{incolor}In [{\color{incolor}95}]:} 
\end{Verbatim}


    \begin{Verbatim}[commandchars=\\\{\}]
                            OLS Regression Results                            
==============================================================================
Dep. Variable:                      y   R-squared:                       0.031
Model:                            OLS   Adj. R-squared:                  0.022
Method:                 Least Squares   F-statistic:                     3.315
Date:                Fri, 26 Apr 2019   Prob (F-statistic):             0.0715
Time:                        17:41:26   Log-Likelihood:                -193.35
No. Observations:                 106   AIC:                             390.7
Df Residuals:                     104   BIC:                             396.0
Df Model:                           1                                         
Covariance Type:            nonrobust                                         
==============================================================================
                 coef    std err          t      P>|t|      [0.025      0.975]
------------------------------------------------------------------------------
const          6.4269      0.259     24.822      0.000       5.913       6.940
x1           -54.4051     29.882     -1.821      0.072    -113.663       4.853
==============================================================================
Omnibus:                       12.639   Durbin-Watson:                   0.082
Prob(Omnibus):                  0.002   Jarque-Bera (JB):               14.065
Skew:                           0.890   Prob(JB):                     0.000883
Kurtosis:                       3.121   Cond. No.                         203.
==============================================================================

Warnings:
[1] Standard Errors assume that the covariance matrix of the errors is correctly specified.

    \end{Verbatim}

    On peut maintenant réaliser un test de changement de structure (test de
Chow).

    \begin{Verbatim}[commandchars=\\\{\}]
{\color{incolor}In [{\color{incolor}99}]:} \PY{n+nb}{print}\PY{p}{(}\PY{n}{F}\PY{p}{,} \PY{n}{p\PYZus{}val}\PY{p}{)}
\end{Verbatim}


    \begin{Verbatim}[commandchars=\\\{\}]
4.147753695614069 0.017132140479924507

    \end{Verbatim}

    La p-value est inférieure au seuil de 5\%. On rejette donc l'hypothèse
de stabilité entre les deux périodes pour la séparation que nous avons
sélectionné.

    \subsection{Question 10}\label{question-10}

\begin{itemize}
\tightlist
\item
  Faites les tests changement de structure de Chow.
\end{itemize}

    On commence par representer graphiquement les séries d'inflation et de
chômage laggées.

    \begin{Verbatim}[commandchars=\\\{\}]
{\color{incolor}In [{\color{incolor}100}]:} 
\end{Verbatim}


    \begin{center}
    \adjustimage{max size={0.9\linewidth}{0.9\paperheight}}{output_193_0.png}
    \end{center}
    { \hspace*{\fill} \\}
    
    \begin{itemize}
\tightlist
\item
  Détecter le point de rupture.
\end{itemize}

    \begin{Verbatim}[commandchars=\\\{\}]
{\color{incolor}In [{\color{incolor}103}]:} 
\end{Verbatim}


    \begin{center}
    \adjustimage{max size={0.9\linewidth}{0.9\paperheight}}{output_195_0.png}
    \end{center}
    { \hspace*{\fill} \\}
    
    \begin{Verbatim}[commandchars=\\\{\}]
{\color{incolor}In [{\color{incolor}104}]:} \PY{c+c1}{\PYZsh{} Période de rupture}
          \PY{n}{df}\PY{o}{.}\PY{n}{DATE}\PY{p}{[}\PY{n+nb}{int}\PY{p}{(}\PY{l+m+mf}{0.1}\PY{o}{*}\PY{n+nb}{len}\PY{p}{(}\PY{n}{df}\PY{p}{)}\PY{p}{)} \PY{o}{+} \PY{n}{np}\PY{o}{.}\PY{n}{argmin}\PY{p}{(}\PY{n}{p\PYZus{}val}\PY{p}{)}\PY{p}{]}
\end{Verbatim}


\begin{Verbatim}[commandchars=\\\{\}]
{\color{outcolor}Out[{\color{outcolor}104}]:} Timestamp('1982-10-01 00:00:00')
\end{Verbatim}
            
    L'hypothèse nulle du test de Chow indique que l'on suppose qu'il n'y a
pas de changement de structure. Ainsi, lorsque la p-value est inférieure
à 5\%, on doit faire l'hypothèse que la structure change dans le temps.
Avant l'année 82, il n'y avait pas/peu de stabilité dans le régime
d'inflation.

La crise pétrolière des années 70 a failli provoquer un changement de
structure. En 1982, un point de rupture a été atteint.

    \subsection{Question 11}\label{question-11}

\begin{itemize}
\tightlist
\item
  Estimer la courbe de Philips en supprimant l'inflation courante des
  variables explicatives mais en ajoutant les délais d'ordre 1, 2, 3 et
  4 de l'inflation et du chômage. 
\end{itemize}

    \begin{Verbatim}[commandchars=\\\{\}]
{\color{incolor}In [{\color{incolor}107}]:} \PY{c+c1}{\PYZsh{} Variables 0\PYZhy{}3 : Unemp.}
          \PY{c+c1}{\PYZsh{} Variables 4\PYZhy{}7 : infl.}
\end{Verbatim}


    \begin{Verbatim}[commandchars=\\\{\}]
                            OLS Regression Results                            
==============================================================================
Dep. Variable:                      8   R-squared:                       0.463
Model:                            OLS   Adj. R-squared:                  0.441
Method:                 Least Squares   F-statistic:                     21.21
Date:                Fri, 26 Apr 2019   Prob (F-statistic):           4.55e-23
Time:                        17:41:32   Log-Likelihood:                -2.0160
No. Observations:                 206   AIC:                             22.03
Df Residuals:                     197   BIC:                             51.98
Df Model:                           8                                         
Covariance Type:            nonrobust                                         
==============================================================================
                 coef    std err          t      P>|t|      [0.025      0.975]
------------------------------------------------------------------------------
const          0.0469      0.020      2.308      0.022       0.007       0.087
0              0.6929      0.070      9.849      0.000       0.554       0.832
1             -0.0130      0.088     -0.148      0.882      -0.187       0.161
2              0.0601      0.088      0.685      0.494      -0.113       0.233
3             -0.1377      0.072     -1.925      0.056      -0.279       0.003
4              3.0798      3.853      0.799      0.425      -4.518      10.678
5              1.7130      4.141      0.414      0.680      -6.454       9.880
6              5.1964      4.058      1.281      0.202      -2.805      13.198
7              5.2705      3.758      1.402      0.162      -2.141      12.682
==============================================================================
Omnibus:                       34.516   Durbin-Watson:                   2.009
Prob(Omnibus):                  0.000   Jarque-Bera (JB):               78.607
Skew:                           0.763   Prob(JB):                     8.52e-18
Kurtosis:                       5.613   Cond. No.                         340.
==============================================================================

Warnings:
[1] Standard Errors assume that the covariance matrix of the errors is correctly specified.

    \end{Verbatim}

    Les variables significatives dans ce nouveau modèle sont : - le chômage
au temps t-1 - le chômage au temps t-4

    \begin{itemize}
\tightlist
\item
   Faire le test de Granger de non causalité de l'inflation sur le
  chômage. Donnez la p-valeur. 
\end{itemize}

    L'hypothèse nulle pour les tests de causalité de Granger est que la
série temporelle dans la deuxième colonne, x2, ne cause PAS la série
temporelle dans la première colonne, x1.

La causalité de Granger signifie que les valeurs passées de x2 ont un
effet statistiquement significatif sur la valeur actuelle de x1, en
tenant compte des valeurs passées de x1 comme variables explicatives.

Nous rejetons l'hypothèse nulle que x2 ne cause pas x1 au sens de
Granger si les valeurs p sont inférieures au seuil désiré.

    \begin{Verbatim}[commandchars=\\\{\}]
{\color{incolor}In [{\color{incolor}108}]:} \PY{n}{granger}\PY{p}{(}\PY{n}{df}\PY{p}{[}\PY{p}{[}\PY{l+s+s1}{\PYZsq{}}\PY{l+s+s1}{Unemp\PYZus{}lag}\PY{l+s+s1}{\PYZsq{}}\PY{p}{,} \PY{l+s+s1}{\PYZsq{}}\PY{l+s+s1}{inf\PYZus{}lag}\PY{l+s+s1}{\PYZsq{}}\PY{p}{]}\PY{p}{]}\PY{p}{,} \PY{l+m+mi}{5}\PY{p}{)}
\end{Verbatim}


    \begin{Verbatim}[commandchars=\\\{\}]

Granger Causality
number of lags (no zero) 1
ssr based F test:         F=0.3433  , p=0.5586  , df\_denom=207, df\_num=1
ssr based chi2 test:   chi2=0.3482  , p=0.5551  , df=1
likelihood ratio test: chi2=0.3480  , p=0.5553  , df=1
parameter F test:         F=0.3433  , p=0.5586  , df\_denom=207, df\_num=1

Granger Causality
number of lags (no zero) 2
ssr based F test:         F=0.5393  , p=0.5840  , df\_denom=204, df\_num=2
ssr based chi2 test:   chi2=1.1051  , p=0.5755  , df=2
likelihood ratio test: chi2=1.1022  , p=0.5763  , df=2
parameter F test:         F=0.5393  , p=0.5840  , df\_denom=204, df\_num=2

Granger Causality
number of lags (no zero) 3
ssr based F test:         F=0.8206  , p=0.4839  , df\_denom=201, df\_num=3
ssr based chi2 test:   chi2=2.5475  , p=0.4668  , df=3
likelihood ratio test: chi2=2.5320  , p=0.4695  , df=3
parameter F test:         F=0.8206  , p=0.4839  , df\_denom=201, df\_num=3

Granger Causality
number of lags (no zero) 4
ssr based F test:         F=0.8129  , p=0.5183  , df\_denom=198, df\_num=4
ssr based chi2 test:   chi2=3.3995  , p=0.4933  , df=4
likelihood ratio test: chi2=3.3719  , p=0.4976  , df=4
parameter F test:         F=0.8129  , p=0.5183  , df\_denom=198, df\_num=4

Granger Causality
number of lags (no zero) 5
ssr based F test:         F=1.1297  , p=0.3459  , df\_denom=195, df\_num=5
ssr based chi2 test:   chi2=5.9672  , p=0.3094  , df=5
likelihood ratio test: chi2=5.8824  , p=0.3178  , df=5
parameter F test:         F=1.1297  , p=0.3459  , df\_denom=195, df\_num=5

    \end{Verbatim}

\begin{Verbatim}[commandchars=\\\{\}]
{\color{outcolor}Out[{\color{outcolor}108}]:} \{1: (\{'lrtest': (0.3479507614470094, 0.5552754528335102, 1),
             'params\_ftest': (0.34326433653111804, 0.5585892708239268, 207.0, 1.0),
             'ssr\_chi2test': (0.3482391819880981, 0.5551115828057411, 1),
             'ssr\_ftest': (0.34326433653112526, 0.5585892708239268, 207.0, 1)\},
            [<statsmodels.regression.linear\_model.RegressionResultsWrapper at 0x1c27b57e10>,
             <statsmodels.regression.linear\_model.RegressionResultsWrapper at 0x1c26bdac18>,
             array([[0., 1., 0.]])]),
           2: (\{'lrtest': (1.1022042952622542, 0.576314276803943, 2),
             'params\_ftest': (0.539338792500007, 0.5839628648705697, 204.0, 2.0),
             'ssr\_chi2test': (1.1051157611029379, 0.5754759274931314, 2),
             'ssr\_ftest': (0.5393387924999984, 0.5839628648705697, 204.0, 2)\},
            [<statsmodels.regression.linear\_model.RegressionResultsWrapper at 0x1c1e68c908>,
             <statsmodels.regression.linear\_model.RegressionResultsWrapper at 0x10b3e9400>,
             array([[0., 0., 1., 0., 0.],
                    [0., 0., 0., 1., 0.]])]),
           3: (\{'lrtest': (2.5320350129870803, 0.46952944812668085, 3),
             'params\_ftest': (0.8205919320314315, 0.48386507661399847, 201.0, 3.0),
             'ssr\_chi2test': (2.547509281530395, 0.46676624303068226, 3),
             'ssr\_ftest': (0.8205919320314254, 0.48386507661399847, 201.0, 3)\},
            [<statsmodels.regression.linear\_model.RegressionResultsWrapper at 0x1c26e6e358>,
             <statsmodels.regression.linear\_model.RegressionResultsWrapper at 0x1c26e6e320>,
             array([[0., 0., 0., 1., 0., 0., 0.],
                    [0., 0., 0., 0., 1., 0., 0.],
                    [0., 0., 0., 0., 0., 1., 0.]])]),
           4: (\{'lrtest': (3.3718812726583565, 0.49762443173547544, 4),
             'params\_ftest': (0.8129224208084789, 0.5182535412167016, 198.0, 4.0),
             'ssr\_chi2test': (3.39949375974459, 0.4933241285296037, 4),
             'ssr\_ftest': (0.8129224208084889, 0.518253541216691, 198.0, 4)\},
            [<statsmodels.regression.linear\_model.RegressionResultsWrapper at 0x1c26e6e438>,
             <statsmodels.regression.linear\_model.RegressionResultsWrapper at 0x1c26e6e5c0>,
             array([[0., 0., 0., 0., 1., 0., 0., 0., 0.],
                    [0., 0., 0., 0., 0., 1., 0., 0., 0.],
                    [0., 0., 0., 0., 0., 0., 1., 0., 0.],
                    [0., 0., 0., 0., 0., 0., 0., 1., 0.]])]),
           5: (\{'lrtest': (5.882360219525708, 0.31783463281265495, 5),
             'params\_ftest': (1.1297033821092815, 0.34594303534456483, 195.0, 5.0),
             'ssr\_chi2test': (5.967151197807944, 0.3094283859411353, 5),
             'ssr\_ftest': (1.129703382109271, 0.3459430353445758, 195.0, 5)\},
            [<statsmodels.regression.linear\_model.RegressionResultsWrapper at 0x1c26e6e978>,
             <statsmodels.regression.linear\_model.RegressionResultsWrapper at 0x1c26e6ea58>,
             array([[0., 0., 0., 0., 0., 1., 0., 0., 0., 0., 0.],
                    [0., 0., 0., 0., 0., 0., 1., 0., 0., 0., 0.],
                    [0., 0., 0., 0., 0., 0., 0., 1., 0., 0., 0.],
                    [0., 0., 0., 0., 0., 0., 0., 0., 1., 0., 0.],
                    [0., 0., 0., 0., 0., 0., 0., 0., 0., 1., 0.]])])\}
\end{Verbatim}
            
    L'intégralité des p-values sont supérieures au seuil de 5\%. Nous
rejetons donc pas l'hypothèse nulle que l'inflation ne cause pas de
chômage.

    \subsection{Question 12}\label{question-12}

\begin{itemize}
\tightlist
\item
  Représentez graphiquement les délais distribués et commentez. 
\end{itemize}

    Un délai distribué peut-être exprimé comme :
\[ y_t = \alpha_0 + \delta_0 z_t + \delta_1 z_{t-1} + \delta_2 z_{t-2} + \delta_3 z_{t-3} + \delta_4 z_{t-4} + u_t \]

    \begin{Verbatim}[commandchars=\\\{\}]
{\color{incolor}In [{\color{incolor}109}]:} 
\end{Verbatim}


    \begin{Verbatim}[commandchars=\\\{\}]
                            OLS Regression Results                            
==============================================================================
Dep. Variable:                      8   R-squared:                       0.455
Model:                            OLS   Adj. R-squared:                  0.442
Method:                 Least Squares   F-statistic:                     33.41
Date:                Fri, 26 Apr 2019   Prob (F-statistic):           1.02e-24
Time:                        17:41:32   Log-Likelihood:                -3.4753
No. Observations:                 206   AIC:                             18.95
Df Residuals:                     200   BIC:                             38.92
Df Model:                           5                                         
Covariance Type:            nonrobust                                         
==============================================================================
                 coef    std err          t      P>|t|      [0.025      0.975]
------------------------------------------------------------------------------
const          0.0539      0.020      2.749      0.007       0.015       0.093
0              0.6924      0.070      9.917      0.000       0.555       0.830
1             -0.0230      0.087     -0.265      0.791      -0.194       0.148
2              0.0577      0.087      0.665      0.507      -0.113       0.229
3             -0.1571      0.069     -2.267      0.024      -0.294      -0.020
4              2.2438      3.435      0.653      0.514      -4.530       9.017
==============================================================================
Omnibus:                       37.726   Durbin-Watson:                   1.981
Prob(Omnibus):                  0.000   Jarque-Bera (JB):               88.869
Skew:                           0.821   Prob(JB):                     5.04e-20
Kurtosis:                       5.767   Cond. No.                         207.
==============================================================================

Warnings:
[1] Standard Errors assume that the covariance matrix of the errors is correctly specified.

    \end{Verbatim}

    \begin{Verbatim}[commandchars=\\\{\}]
{\color{incolor}In [{\color{incolor}110}]:} 
\end{Verbatim}


    \begin{center}
    \adjustimage{max size={0.9\linewidth}{0.9\paperheight}}{output_208_0.png}
    \end{center}
    { \hspace*{\fill} \\}
    
    \begin{itemize}
\tightlist
\item
  Calculer l'impact à long de terme de l'inflation sur le chômage.
\end{itemize}

    L'effet de long terme est donné par :
\[ \delta_0 + \delta_1 + \delta_2 + \delta_3 + ... \]

    \begin{Verbatim}[commandchars=\\\{\}]
{\color{incolor}In [{\color{incolor}111}]:} \PY{n}{model}\PY{o}{.}\PY{n}{params}\PY{p}{[}\PY{l+m+mi}{1}\PY{p}{:}\PY{p}{]}\PY{o}{.}\PY{n}{sum}\PY{p}{(}\PY{p}{)}
\end{Verbatim}


\begin{Verbatim}[commandchars=\\\{\}]
{\color{outcolor}Out[{\color{outcolor}111}]:} 2.8137925103027586
\end{Verbatim}
            

    % Add a bibliography block to the postdoc
    
    
    
    \end{document}
