
% Default to the notebook output style

    


% Inherit from the specified cell style.




    
\documentclass[11pt]{article}

    
    
    \usepackage[T1]{fontenc}
    % Nicer default font (+ math font) than Computer Modern for most use cases
    \usepackage{mathpazo}

    % Basic figure setup, for now with no caption control since it's done
    % automatically by Pandoc (which extracts ![](path) syntax from Markdown).
    \usepackage{graphicx}
    % We will generate all images so they have a width \maxwidth. This means
    % that they will get their normal width if they fit onto the page, but
    % are scaled down if they would overflow the margins.
    \makeatletter
    \def\maxwidth{\ifdim\Gin@nat@width>\linewidth\linewidth
    \else\Gin@nat@width\fi}
    \makeatother
    \let\Oldincludegraphics\includegraphics
    % Set max figure width to be 80% of text width, for now hardcoded.
    \renewcommand{\includegraphics}[1]{\Oldincludegraphics[width=.8\maxwidth]{#1}}
    % Ensure that by default, figures have no caption (until we provide a
    % proper Figure object with a Caption API and a way to capture that
    % in the conversion process - todo).
    \usepackage{caption}
    \DeclareCaptionLabelFormat{nolabel}{}
    \captionsetup{labelformat=nolabel}

    \usepackage{adjustbox} % Used to constrain images to a maximum size 
    \usepackage{xcolor} % Allow colors to be defined
    \usepackage{enumerate} % Needed for markdown enumerations to work
    \usepackage{geometry} % Used to adjust the document margins
    \usepackage{amsmath} % Equations
    \usepackage{amssymb} % Equations
    \usepackage{textcomp} % defines textquotesingle
    % Hack from http://tex.stackexchange.com/a/47451/13684:
    \AtBeginDocument{%
        \def\PYZsq{\textquotesingle}% Upright quotes in Pygmentized code
    }
    \usepackage{upquote} % Upright quotes for verbatim code
    \usepackage{eurosym} % defines \euro
    \usepackage[mathletters]{ucs} % Extended unicode (utf-8) support
    \usepackage[utf8x]{inputenc} % Allow utf-8 characters in the tex document
    \usepackage{fancyvrb} % verbatim replacement that allows latex
    \usepackage{grffile} % extends the file name processing of package graphics 
                         % to support a larger range 
    % The hyperref package gives us a pdf with properly built
    % internal navigation ('pdf bookmarks' for the table of contents,
    % internal cross-reference links, web links for URLs, etc.)
    \usepackage{hyperref}
    \usepackage{longtable} % longtable support required by pandoc >1.10
    \usepackage{booktabs}  % table support for pandoc > 1.12.2
    \usepackage[inline]{enumitem} % IRkernel/repr support (it uses the enumerate* environment)
    \usepackage[normalem]{ulem} % ulem is needed to support strikethroughs (\sout)
                                % normalem makes italics be italics, not underlines
    

    
    
    % Colors for the hyperref package
    \definecolor{urlcolor}{rgb}{0,.145,.698}
    \definecolor{linkcolor}{rgb}{.71,0.21,0.01}
    \definecolor{citecolor}{rgb}{.12,.54,.11}

    % ANSI colors
    \definecolor{ansi-black}{HTML}{3E424D}
    \definecolor{ansi-black-intense}{HTML}{282C36}
    \definecolor{ansi-red}{HTML}{E75C58}
    \definecolor{ansi-red-intense}{HTML}{B22B31}
    \definecolor{ansi-green}{HTML}{00A250}
    \definecolor{ansi-green-intense}{HTML}{007427}
    \definecolor{ansi-yellow}{HTML}{DDB62B}
    \definecolor{ansi-yellow-intense}{HTML}{B27D12}
    \definecolor{ansi-blue}{HTML}{208FFB}
    \definecolor{ansi-blue-intense}{HTML}{0065CA}
    \definecolor{ansi-magenta}{HTML}{D160C4}
    \definecolor{ansi-magenta-intense}{HTML}{A03196}
    \definecolor{ansi-cyan}{HTML}{60C6C8}
    \definecolor{ansi-cyan-intense}{HTML}{258F8F}
    \definecolor{ansi-white}{HTML}{C5C1B4}
    \definecolor{ansi-white-intense}{HTML}{A1A6B2}

    % commands and environments needed by pandoc snippets
    % extracted from the output of `pandoc -s`
    \providecommand{\tightlist}{%
      \setlength{\itemsep}{0pt}\setlength{\parskip}{0pt}}
    \DefineVerbatimEnvironment{Highlighting}{Verbatim}{commandchars=\\\{\}}
    % Add ',fontsize=\small' for more characters per line
    \newenvironment{Shaded}{}{}
    \newcommand{\KeywordTok}[1]{\textcolor[rgb]{0.00,0.44,0.13}{\textbf{{#1}}}}
    \newcommand{\DataTypeTok}[1]{\textcolor[rgb]{0.56,0.13,0.00}{{#1}}}
    \newcommand{\DecValTok}[1]{\textcolor[rgb]{0.25,0.63,0.44}{{#1}}}
    \newcommand{\BaseNTok}[1]{\textcolor[rgb]{0.25,0.63,0.44}{{#1}}}
    \newcommand{\FloatTok}[1]{\textcolor[rgb]{0.25,0.63,0.44}{{#1}}}
    \newcommand{\CharTok}[1]{\textcolor[rgb]{0.25,0.44,0.63}{{#1}}}
    \newcommand{\StringTok}[1]{\textcolor[rgb]{0.25,0.44,0.63}{{#1}}}
    \newcommand{\CommentTok}[1]{\textcolor[rgb]{0.38,0.63,0.69}{\textit{{#1}}}}
    \newcommand{\OtherTok}[1]{\textcolor[rgb]{0.00,0.44,0.13}{{#1}}}
    \newcommand{\AlertTok}[1]{\textcolor[rgb]{1.00,0.00,0.00}{\textbf{{#1}}}}
    \newcommand{\FunctionTok}[1]{\textcolor[rgb]{0.02,0.16,0.49}{{#1}}}
    \newcommand{\RegionMarkerTok}[1]{{#1}}
    \newcommand{\ErrorTok}[1]{\textcolor[rgb]{1.00,0.00,0.00}{\textbf{{#1}}}}
    \newcommand{\NormalTok}[1]{{#1}}
    
    % Additional commands for more recent versions of Pandoc
    \newcommand{\ConstantTok}[1]{\textcolor[rgb]{0.53,0.00,0.00}{{#1}}}
    \newcommand{\SpecialCharTok}[1]{\textcolor[rgb]{0.25,0.44,0.63}{{#1}}}
    \newcommand{\VerbatimStringTok}[1]{\textcolor[rgb]{0.25,0.44,0.63}{{#1}}}
    \newcommand{\SpecialStringTok}[1]{\textcolor[rgb]{0.73,0.40,0.53}{{#1}}}
    \newcommand{\ImportTok}[1]{{#1}}
    \newcommand{\DocumentationTok}[1]{\textcolor[rgb]{0.73,0.13,0.13}{\textit{{#1}}}}
    \newcommand{\AnnotationTok}[1]{\textcolor[rgb]{0.38,0.63,0.69}{\textbf{\textit{{#1}}}}}
    \newcommand{\CommentVarTok}[1]{\textcolor[rgb]{0.38,0.63,0.69}{\textbf{\textit{{#1}}}}}
    \newcommand{\VariableTok}[1]{\textcolor[rgb]{0.10,0.09,0.49}{{#1}}}
    \newcommand{\ControlFlowTok}[1]{\textcolor[rgb]{0.00,0.44,0.13}{\textbf{{#1}}}}
    \newcommand{\OperatorTok}[1]{\textcolor[rgb]{0.40,0.40,0.40}{{#1}}}
    \newcommand{\BuiltInTok}[1]{{#1}}
    \newcommand{\ExtensionTok}[1]{{#1}}
    \newcommand{\PreprocessorTok}[1]{\textcolor[rgb]{0.74,0.48,0.00}{{#1}}}
    \newcommand{\AttributeTok}[1]{\textcolor[rgb]{0.49,0.56,0.16}{{#1}}}
    \newcommand{\InformationTok}[1]{\textcolor[rgb]{0.38,0.63,0.69}{\textbf{\textit{{#1}}}}}
    \newcommand{\WarningTok}[1]{\textcolor[rgb]{0.38,0.63,0.69}{\textbf{\textit{{#1}}}}}
    
    
    % Define a nice break command that doesn't care if a line doesn't already
    % exist.
    \def\br{\hspace*{\fill} \\* }
    % Math Jax compatability definitions
    \def\gt{>}
    \def\lt{<}
    % Document parameters
    \title{Lab 2 - Altair}
    
    
    

    % Pygments definitions
    
\makeatletter
\def\PY@reset{\let\PY@it=\relax \let\PY@bf=\relax%
    \let\PY@ul=\relax \let\PY@tc=\relax%
    \let\PY@bc=\relax \let\PY@ff=\relax}
\def\PY@tok#1{\csname PY@tok@#1\endcsname}
\def\PY@toks#1+{\ifx\relax#1\empty\else%
    \PY@tok{#1}\expandafter\PY@toks\fi}
\def\PY@do#1{\PY@bc{\PY@tc{\PY@ul{%
    \PY@it{\PY@bf{\PY@ff{#1}}}}}}}
\def\PY#1#2{\PY@reset\PY@toks#1+\relax+\PY@do{#2}}

\expandafter\def\csname PY@tok@w\endcsname{\def\PY@tc##1{\textcolor[rgb]{0.73,0.73,0.73}{##1}}}
\expandafter\def\csname PY@tok@c\endcsname{\let\PY@it=\textit\def\PY@tc##1{\textcolor[rgb]{0.25,0.50,0.50}{##1}}}
\expandafter\def\csname PY@tok@cp\endcsname{\def\PY@tc##1{\textcolor[rgb]{0.74,0.48,0.00}{##1}}}
\expandafter\def\csname PY@tok@k\endcsname{\let\PY@bf=\textbf\def\PY@tc##1{\textcolor[rgb]{0.00,0.50,0.00}{##1}}}
\expandafter\def\csname PY@tok@kp\endcsname{\def\PY@tc##1{\textcolor[rgb]{0.00,0.50,0.00}{##1}}}
\expandafter\def\csname PY@tok@kt\endcsname{\def\PY@tc##1{\textcolor[rgb]{0.69,0.00,0.25}{##1}}}
\expandafter\def\csname PY@tok@o\endcsname{\def\PY@tc##1{\textcolor[rgb]{0.40,0.40,0.40}{##1}}}
\expandafter\def\csname PY@tok@ow\endcsname{\let\PY@bf=\textbf\def\PY@tc##1{\textcolor[rgb]{0.67,0.13,1.00}{##1}}}
\expandafter\def\csname PY@tok@nb\endcsname{\def\PY@tc##1{\textcolor[rgb]{0.00,0.50,0.00}{##1}}}
\expandafter\def\csname PY@tok@nf\endcsname{\def\PY@tc##1{\textcolor[rgb]{0.00,0.00,1.00}{##1}}}
\expandafter\def\csname PY@tok@nc\endcsname{\let\PY@bf=\textbf\def\PY@tc##1{\textcolor[rgb]{0.00,0.00,1.00}{##1}}}
\expandafter\def\csname PY@tok@nn\endcsname{\let\PY@bf=\textbf\def\PY@tc##1{\textcolor[rgb]{0.00,0.00,1.00}{##1}}}
\expandafter\def\csname PY@tok@ne\endcsname{\let\PY@bf=\textbf\def\PY@tc##1{\textcolor[rgb]{0.82,0.25,0.23}{##1}}}
\expandafter\def\csname PY@tok@nv\endcsname{\def\PY@tc##1{\textcolor[rgb]{0.10,0.09,0.49}{##1}}}
\expandafter\def\csname PY@tok@no\endcsname{\def\PY@tc##1{\textcolor[rgb]{0.53,0.00,0.00}{##1}}}
\expandafter\def\csname PY@tok@nl\endcsname{\def\PY@tc##1{\textcolor[rgb]{0.63,0.63,0.00}{##1}}}
\expandafter\def\csname PY@tok@ni\endcsname{\let\PY@bf=\textbf\def\PY@tc##1{\textcolor[rgb]{0.60,0.60,0.60}{##1}}}
\expandafter\def\csname PY@tok@na\endcsname{\def\PY@tc##1{\textcolor[rgb]{0.49,0.56,0.16}{##1}}}
\expandafter\def\csname PY@tok@nt\endcsname{\let\PY@bf=\textbf\def\PY@tc##1{\textcolor[rgb]{0.00,0.50,0.00}{##1}}}
\expandafter\def\csname PY@tok@nd\endcsname{\def\PY@tc##1{\textcolor[rgb]{0.67,0.13,1.00}{##1}}}
\expandafter\def\csname PY@tok@s\endcsname{\def\PY@tc##1{\textcolor[rgb]{0.73,0.13,0.13}{##1}}}
\expandafter\def\csname PY@tok@sd\endcsname{\let\PY@it=\textit\def\PY@tc##1{\textcolor[rgb]{0.73,0.13,0.13}{##1}}}
\expandafter\def\csname PY@tok@si\endcsname{\let\PY@bf=\textbf\def\PY@tc##1{\textcolor[rgb]{0.73,0.40,0.53}{##1}}}
\expandafter\def\csname PY@tok@se\endcsname{\let\PY@bf=\textbf\def\PY@tc##1{\textcolor[rgb]{0.73,0.40,0.13}{##1}}}
\expandafter\def\csname PY@tok@sr\endcsname{\def\PY@tc##1{\textcolor[rgb]{0.73,0.40,0.53}{##1}}}
\expandafter\def\csname PY@tok@ss\endcsname{\def\PY@tc##1{\textcolor[rgb]{0.10,0.09,0.49}{##1}}}
\expandafter\def\csname PY@tok@sx\endcsname{\def\PY@tc##1{\textcolor[rgb]{0.00,0.50,0.00}{##1}}}
\expandafter\def\csname PY@tok@m\endcsname{\def\PY@tc##1{\textcolor[rgb]{0.40,0.40,0.40}{##1}}}
\expandafter\def\csname PY@tok@gh\endcsname{\let\PY@bf=\textbf\def\PY@tc##1{\textcolor[rgb]{0.00,0.00,0.50}{##1}}}
\expandafter\def\csname PY@tok@gu\endcsname{\let\PY@bf=\textbf\def\PY@tc##1{\textcolor[rgb]{0.50,0.00,0.50}{##1}}}
\expandafter\def\csname PY@tok@gd\endcsname{\def\PY@tc##1{\textcolor[rgb]{0.63,0.00,0.00}{##1}}}
\expandafter\def\csname PY@tok@gi\endcsname{\def\PY@tc##1{\textcolor[rgb]{0.00,0.63,0.00}{##1}}}
\expandafter\def\csname PY@tok@gr\endcsname{\def\PY@tc##1{\textcolor[rgb]{1.00,0.00,0.00}{##1}}}
\expandafter\def\csname PY@tok@ge\endcsname{\let\PY@it=\textit}
\expandafter\def\csname PY@tok@gs\endcsname{\let\PY@bf=\textbf}
\expandafter\def\csname PY@tok@gp\endcsname{\let\PY@bf=\textbf\def\PY@tc##1{\textcolor[rgb]{0.00,0.00,0.50}{##1}}}
\expandafter\def\csname PY@tok@go\endcsname{\def\PY@tc##1{\textcolor[rgb]{0.53,0.53,0.53}{##1}}}
\expandafter\def\csname PY@tok@gt\endcsname{\def\PY@tc##1{\textcolor[rgb]{0.00,0.27,0.87}{##1}}}
\expandafter\def\csname PY@tok@err\endcsname{\def\PY@bc##1{\setlength{\fboxsep}{0pt}\fcolorbox[rgb]{1.00,0.00,0.00}{1,1,1}{\strut ##1}}}
\expandafter\def\csname PY@tok@kc\endcsname{\let\PY@bf=\textbf\def\PY@tc##1{\textcolor[rgb]{0.00,0.50,0.00}{##1}}}
\expandafter\def\csname PY@tok@kd\endcsname{\let\PY@bf=\textbf\def\PY@tc##1{\textcolor[rgb]{0.00,0.50,0.00}{##1}}}
\expandafter\def\csname PY@tok@kn\endcsname{\let\PY@bf=\textbf\def\PY@tc##1{\textcolor[rgb]{0.00,0.50,0.00}{##1}}}
\expandafter\def\csname PY@tok@kr\endcsname{\let\PY@bf=\textbf\def\PY@tc##1{\textcolor[rgb]{0.00,0.50,0.00}{##1}}}
\expandafter\def\csname PY@tok@bp\endcsname{\def\PY@tc##1{\textcolor[rgb]{0.00,0.50,0.00}{##1}}}
\expandafter\def\csname PY@tok@fm\endcsname{\def\PY@tc##1{\textcolor[rgb]{0.00,0.00,1.00}{##1}}}
\expandafter\def\csname PY@tok@vc\endcsname{\def\PY@tc##1{\textcolor[rgb]{0.10,0.09,0.49}{##1}}}
\expandafter\def\csname PY@tok@vg\endcsname{\def\PY@tc##1{\textcolor[rgb]{0.10,0.09,0.49}{##1}}}
\expandafter\def\csname PY@tok@vi\endcsname{\def\PY@tc##1{\textcolor[rgb]{0.10,0.09,0.49}{##1}}}
\expandafter\def\csname PY@tok@vm\endcsname{\def\PY@tc##1{\textcolor[rgb]{0.10,0.09,0.49}{##1}}}
\expandafter\def\csname PY@tok@sa\endcsname{\def\PY@tc##1{\textcolor[rgb]{0.73,0.13,0.13}{##1}}}
\expandafter\def\csname PY@tok@sb\endcsname{\def\PY@tc##1{\textcolor[rgb]{0.73,0.13,0.13}{##1}}}
\expandafter\def\csname PY@tok@sc\endcsname{\def\PY@tc##1{\textcolor[rgb]{0.73,0.13,0.13}{##1}}}
\expandafter\def\csname PY@tok@dl\endcsname{\def\PY@tc##1{\textcolor[rgb]{0.73,0.13,0.13}{##1}}}
\expandafter\def\csname PY@tok@s2\endcsname{\def\PY@tc##1{\textcolor[rgb]{0.73,0.13,0.13}{##1}}}
\expandafter\def\csname PY@tok@sh\endcsname{\def\PY@tc##1{\textcolor[rgb]{0.73,0.13,0.13}{##1}}}
\expandafter\def\csname PY@tok@s1\endcsname{\def\PY@tc##1{\textcolor[rgb]{0.73,0.13,0.13}{##1}}}
\expandafter\def\csname PY@tok@mb\endcsname{\def\PY@tc##1{\textcolor[rgb]{0.40,0.40,0.40}{##1}}}
\expandafter\def\csname PY@tok@mf\endcsname{\def\PY@tc##1{\textcolor[rgb]{0.40,0.40,0.40}{##1}}}
\expandafter\def\csname PY@tok@mh\endcsname{\def\PY@tc##1{\textcolor[rgb]{0.40,0.40,0.40}{##1}}}
\expandafter\def\csname PY@tok@mi\endcsname{\def\PY@tc##1{\textcolor[rgb]{0.40,0.40,0.40}{##1}}}
\expandafter\def\csname PY@tok@il\endcsname{\def\PY@tc##1{\textcolor[rgb]{0.40,0.40,0.40}{##1}}}
\expandafter\def\csname PY@tok@mo\endcsname{\def\PY@tc##1{\textcolor[rgb]{0.40,0.40,0.40}{##1}}}
\expandafter\def\csname PY@tok@ch\endcsname{\let\PY@it=\textit\def\PY@tc##1{\textcolor[rgb]{0.25,0.50,0.50}{##1}}}
\expandafter\def\csname PY@tok@cm\endcsname{\let\PY@it=\textit\def\PY@tc##1{\textcolor[rgb]{0.25,0.50,0.50}{##1}}}
\expandafter\def\csname PY@tok@cpf\endcsname{\let\PY@it=\textit\def\PY@tc##1{\textcolor[rgb]{0.25,0.50,0.50}{##1}}}
\expandafter\def\csname PY@tok@c1\endcsname{\let\PY@it=\textit\def\PY@tc##1{\textcolor[rgb]{0.25,0.50,0.50}{##1}}}
\expandafter\def\csname PY@tok@cs\endcsname{\let\PY@it=\textit\def\PY@tc##1{\textcolor[rgb]{0.25,0.50,0.50}{##1}}}

\def\PYZbs{\char`\\}
\def\PYZus{\char`\_}
\def\PYZob{\char`\{}
\def\PYZcb{\char`\}}
\def\PYZca{\char`\^}
\def\PYZam{\char`\&}
\def\PYZlt{\char`\<}
\def\PYZgt{\char`\>}
\def\PYZsh{\char`\#}
\def\PYZpc{\char`\%}
\def\PYZdl{\char`\$}
\def\PYZhy{\char`\-}
\def\PYZsq{\char`\'}
\def\PYZdq{\char`\"}
\def\PYZti{\char`\~}
% for compatibility with earlier versions
\def\PYZat{@}
\def\PYZlb{[}
\def\PYZrb{]}
\makeatother


    % Exact colors from NB
    \definecolor{incolor}{rgb}{0.0, 0.0, 0.5}
    \definecolor{outcolor}{rgb}{0.545, 0.0, 0.0}



    
    % Prevent overflowing lines due to hard-to-break entities
    \sloppy 
    % Setup hyperref package
    \hypersetup{
      breaklinks=true,  % so long urls are correctly broken across lines
      colorlinks=true,
      urlcolor=urlcolor,
      linkcolor=linkcolor,
      citecolor=citecolor,
      }
    % Slightly bigger margins than the latex defaults
    
    \geometry{verbose,tmargin=1in,bmargin=1in,lmargin=1in,rmargin=1in}
    
    

    \begin{document}
    
    
    \maketitle
    
    

    
    \section{Lab 2 --- Altair}\label{lab-2-altair}

In last week's lab, we used Tableau to create a dashboard from places in
France. Tableau is a great tool, but it has two primary limitations: 1)
while Tableau is quite powerful---and we've only scratched the surface
of what it can do---sometimes you need to do more. 2) A lot of
statistical data work is done programmatically rather than through a
drag-n-drop interface.

This week, we will continue to work with the Places-in-France dataset.
This time, we will use \href{https://altair-viz.github.io}{Altair}, a
Python library for creating statistical visualizations.

The goals of this lab are:

\begin{itemize}
\tightlist
\item
  Get a better understanding of grammar of statistical visualisations:
  spaces, mappings, marks, and encodings,
\item
  Give you a sense of a complementary, programmatic way of creating an
  interactive visualisation,
\item
  Understand the declarative way of thinking used by Altair, Vega-Lite,
  and D3.
\end{itemize}

With that in mind, let's get started.

\subsection{Loading data}\label{loading-data}

\href{https://perso.telecom-paristech.fr/eagan/class/igr204/data/france.csv}{Download
our Places-in-France dataset} and place it in the same folder as this
notebook.

We'll now need to import it into our project. Altair can manage multiple
data formats. In our case, we'll use a version of last week's data
converted into CSV format. (It can also handle TSV, JSON, and Pandas
Dataframes.)

In our case, we're going to just use the name of our CSV file. You could
also use a URL or a full path.

    \begin{Verbatim}[commandchars=\\\{\}]
{\color{incolor}In [{\color{incolor}1}]:} \PY{c+c1}{\PYZsh{} Import the Altair library}
        \PY{c+c1}{\PYZsh{} by convention, we use alt to save typing, but you can just \PYZsq{}import altair\PYZsq{} if you wish}
        \PY{k+kn}{import} \PY{n+nn}{altair} \PY{k}{as} \PY{n+nn}{alt} 
        
        \PY{c+c1}{\PYZsh{} Reference to data}
        \PY{n}{france} \PY{o}{=} \PY{l+s+s1}{\PYZsq{}}\PY{l+s+s1}{france.csv}\PY{l+s+s1}{\PYZsq{}}
\end{Verbatim}


    \begin{Verbatim}[commandchars=\\\{\}]
/anaconda3/lib/python3.6/site-packages/requests/\_\_init\_\_.py:80: RequestsDependencyWarning: urllib3 (1.21.1) or chardet (2.3.0) doesn't match a supported version!
  RequestsDependencyWarning)

    \end{Verbatim}

    \subsection{Making a basic chart}\label{making-a-basic-chart}

Recall that our data set has columns \texttt{Postal\ Code}, \texttt{x},
\texttt{y}, \texttt{inseecode}, \texttt{place}, \texttt{population}, and
\texttt{density}. Let's make a map that plots the \texttt{x} vs
\texttt{y} columns as points.

In the code below, we'll make a \texttt{Chart} from the \texttt{france}
data using \texttt{point} marks. We'll bind (or \emph{encode} the
chart's \texttt{x=} axis to the \texttt{x} attribute of our data set
(and similarly for \texttt{y}). In both cases, we tell Altair that the
data are \texttt{Q}uantitative with \texttt{:Q}. (There's also
\texttt{:N} for nominal/categorical data, \texttt{:O} for ordinal, and
\texttt{:T} for temporal data.)

    \begin{Verbatim}[commandchars=\\\{\}]
{\color{incolor}In [{\color{incolor}3}]:} \PY{n+nb}{map} \PY{o}{=} \PY{n}{alt}\PY{o}{.}\PY{n}{Chart}\PY{p}{(}\PY{n}{france}\PY{p}{)}\PY{o}{.}\PY{n}{mark\PYZus{}point}\PY{p}{(}\PY{p}{)}\PY{o}{.}\PY{n}{encode}\PY{p}{(}
            \PY{n}{x}\PY{o}{=}\PY{l+s+s1}{\PYZsq{}}\PY{l+s+s1}{x:Q}\PY{l+s+s1}{\PYZsq{}}\PY{p}{,}
            \PY{n}{y}\PY{o}{=}\PY{l+s+s1}{\PYZsq{}}\PY{l+s+s1}{y:Q}\PY{l+s+s1}{\PYZsq{}}
        \PY{p}{)}
        
        \PY{n+nb}{map}
\end{Verbatim}

\texttt{\color{outcolor}Out[{\color{outcolor}3}]:}
    
    \begin{center}
    \adjustimage{max size={0.9\linewidth}{0.9\paperheight}}{output_3_0.png}
    \end{center}
    { \hspace*{\fill} \\}
    

    That looks nice, but notice that Altair shows the origin (0, 0) by
default. That's usually what you want, but not in this case. Let's tell
Altair to hide the axes and the origin.

    \begin{Verbatim}[commandchars=\\\{\}]
{\color{incolor}In [{\color{incolor}4}]:} \PY{c+c1}{\PYZsh{}\PYZsh{} Hmm, looks nice, but let\PYZsq{}s change the default scales.}
        \PY{c+c1}{\PYZsh{}\PYZsh{} Need to change from shorthand notation.}
        
        \PY{n+nb}{map} \PY{o}{=} \PY{n}{alt}\PY{o}{.}\PY{n}{Chart}\PY{p}{(}\PY{n}{france}\PY{p}{)}\PY{o}{.}\PY{n}{mark\PYZus{}point}\PY{p}{(}\PY{p}{)}\PY{o}{.}\PY{n}{encode}\PY{p}{(}
            \PY{n}{x}\PY{o}{=}\PY{n}{alt}\PY{o}{.}\PY{n}{X}\PY{p}{(}\PY{l+s+s1}{\PYZsq{}}\PY{l+s+s1}{x:Q}\PY{l+s+s1}{\PYZsq{}}\PY{p}{,} \PY{n}{axis}\PY{o}{=}\PY{k+kc}{None}\PY{p}{)}\PY{p}{,}
            \PY{n}{y}\PY{o}{=}\PY{n}{alt}\PY{o}{.}\PY{n}{Y}\PY{p}{(}\PY{l+s+s1}{\PYZsq{}}\PY{l+s+s1}{y:Q}\PY{l+s+s1}{\PYZsq{}}\PY{p}{,} \PY{n}{axis}\PY{o}{=}\PY{k+kc}{None}\PY{p}{,} \PY{n}{scale}\PY{o}{=}\PY{n}{alt}\PY{o}{.}\PY{n}{Scale}\PY{p}{(}\PY{n}{zero}\PY{o}{=}\PY{k+kc}{False}\PY{p}{)}\PY{p}{)}
        \PY{p}{)}
        
        \PY{n+nb}{map}
\end{Verbatim}

\texttt{\color{outcolor}Out[{\color{outcolor}4}]:}
    
    \begin{center}
    \adjustimage{max size={0.9\linewidth}{0.9\paperheight}}{output_5_0.png}
    \end{center}
    { \hspace*{\fill} \\}
    

    In the previous map, we had used Altair's shorthand syntax to specify
the data bindings. That works for the common case, but here we need more
control, so we have to use an explicit object for the \texttt{x} and
\texttt{y} dimensions: \texttt{alt.X(...)} and \texttt{alt.Y(...)}. We
also need to create a \texttt{Scale} to set its \texttt{zero=False}.

This new chart looks much better, but we really can't see much detail
here. Let's reduce the default mark \texttt{size} to \texttt{1}. We're
also going to adjust the chart \texttt{properties} to set the width and
height to something bigger.

    \begin{Verbatim}[commandchars=\\\{\}]
{\color{incolor}In [{\color{incolor}7}]:} \PY{c+c1}{\PYZsh{}\PYZsh{} Much better, but we really can\PYZsq{}t see much here.  Let\PYZsq{}s make the point sizes smaller and make the canvas bigger}
        
        \PY{n+nb}{map} \PY{o}{=} \PY{n}{alt}\PY{o}{.}\PY{n}{Chart}\PY{p}{(}\PY{n}{france}\PY{p}{)}\PY{o}{.}\PY{n}{mark\PYZus{}point}\PY{p}{(}\PY{n}{size}\PY{o}{=}\PY{l+m+mi}{1}\PY{p}{)}\PY{o}{.}\PY{n}{encode}\PY{p}{(}
            \PY{n}{x}\PY{o}{=}\PY{n}{alt}\PY{o}{.}\PY{n}{X}\PY{p}{(}\PY{l+s+s1}{\PYZsq{}}\PY{l+s+s1}{x:Q}\PY{l+s+s1}{\PYZsq{}}\PY{p}{,} \PY{n}{axis}\PY{o}{=}\PY{k+kc}{None}\PY{p}{)}\PY{p}{,}
            \PY{n}{y}\PY{o}{=}\PY{n}{alt}\PY{o}{.}\PY{n}{Y}\PY{p}{(}\PY{l+s+s1}{\PYZsq{}}\PY{l+s+s1}{y:Q}\PY{l+s+s1}{\PYZsq{}}\PY{p}{,} \PY{n}{axis}\PY{o}{=}\PY{k+kc}{None}\PY{p}{,} \PY{n}{scale}\PY{o}{=}\PY{n}{alt}\PY{o}{.}\PY{n}{Scale}\PY{p}{(}\PY{n}{zero}\PY{o}{=}\PY{k+kc}{False}\PY{p}{)}\PY{p}{)}
        \PY{p}{)}\PY{o}{.}\PY{n}{properties}\PY{p}{(}
            \PY{n}{width}\PY{o}{=}\PY{l+m+mi}{800}\PY{p}{,}
            \PY{n}{height}\PY{o}{=}\PY{l+m+mi}{800}
        \PY{p}{)}
        
        \PY{n+nb}{map}
\end{Verbatim}

\texttt{\color{outcolor}Out[{\color{outcolor}7}]:}
    
    \begin{center}
    \adjustimage{max size={0.9\linewidth}{0.9\paperheight}}{output_7_0.png}
    \end{center}
    { \hspace*{\fill} \\}
    

    In Altair, a chart is made up of three primary things:

\begin{itemize}
\tightlist
\item
  Marks --- the kinds of shapes to draw
\item
  Encodings --- a mapping from data to attributes of the marks
\item
  Properties --- meta-data about the chart
\end{itemize}

Now that we've seen each of them, let's talk a little more about each.

\subsubsection{Marks}\label{marks}

\emph{Marks} are the kinds of shapes that we want to draw. Here is a
summary of the kinds of marks in Altair, taken from the
\href{https://altair-viz.github.io/user_guide/marks.html}{Altair
documentation}.

\begin{longtable}[]{@{}lll@{}}
\toprule
Mark Name & Method & Description\tabularnewline
\midrule
\endhead
area & \texttt{mark\_area()} & A filled area plot.\tabularnewline
bar & \texttt{mark\_bar()} & A bar plot.\tabularnewline
circle & \texttt{mark\_circle()} & A scatter plot with filled
circles.\tabularnewline
geoshape & \texttt{mark\_geoshape()} & A geographic shape\tabularnewline
line & \texttt{mark\_line()} & A line plot.\tabularnewline
point & \texttt{mark\_point()} & A scatter plot with configurable point
shapes.\tabularnewline
rect & \texttt{mark\_rect()} & A filled rectangle, used for
heatmaps\tabularnewline
rule & \texttt{mark\_rule()} & A vertical or horizontal line spanning
the axis.\tabularnewline
square & \texttt{mark\_square()} & A scatter plot with filled
squares.\tabularnewline
text & \texttt{mark\_text()} & A scatter plot with points represented by
text.\tabularnewline
tick & \texttt{mark\_tick()} & A vertical or horizontal tick
mark.\tabularnewline
\bottomrule
\end{longtable}

The basic idea is that each data point will get mapped into one of these
types of marks.

\subsubsection{Encodings}\label{encodings}

Encodings determine the binding between the data point and the mark. In
our example, we've been encoding the \texttt{x} and \texttt{y} data
columns to each mark's \texttt{x}- and \texttt{y}-position. As you can
see in the
\href{https://altair-viz.github.io/user_guide/encoding.html\#encoding-channels}{Altair
documentation}, there are positional encodings, mark property encodings
(as we'll see in the next step), and interaction encodings (as we'll see
later for with tooltips).

Let's go ahead and bind the size of the places to their population in
our data set. In other words, we're going to \emph{encode} the
\texttt{population} data attribute by the \emph{size} of each
\emph{mark}.

    \begin{Verbatim}[commandchars=\\\{\}]
{\color{incolor}In [{\color{incolor}8}]:} \PY{n+nb}{map} \PY{o}{=} \PY{n}{alt}\PY{o}{.}\PY{n}{Chart}\PY{p}{(}\PY{n}{france}\PY{p}{)}\PY{o}{.}\PY{n}{mark\PYZus{}point}\PY{p}{(}\PY{n}{size}\PY{o}{=}\PY{l+m+mi}{1}\PY{p}{)}\PY{o}{.}\PY{n}{encode}\PY{p}{(}
            \PY{n}{x}\PY{o}{=}\PY{n}{alt}\PY{o}{.}\PY{n}{X}\PY{p}{(}\PY{l+s+s1}{\PYZsq{}}\PY{l+s+s1}{x:Q}\PY{l+s+s1}{\PYZsq{}}\PY{p}{,} \PY{n}{axis}\PY{o}{=}\PY{k+kc}{None}\PY{p}{)}\PY{p}{,}
            \PY{n}{y}\PY{o}{=}\PY{n}{alt}\PY{o}{.}\PY{n}{Y}\PY{p}{(}\PY{l+s+s1}{\PYZsq{}}\PY{l+s+s1}{y:Q}\PY{l+s+s1}{\PYZsq{}}\PY{p}{,} \PY{n}{axis}\PY{o}{=}\PY{k+kc}{None}\PY{p}{,} \PY{n}{scale}\PY{o}{=}\PY{n}{alt}\PY{o}{.}\PY{n}{Scale}\PY{p}{(}\PY{n}{zero}\PY{o}{=}\PY{k+kc}{False}\PY{p}{)}\PY{p}{)}\PY{p}{,}
            
            \PY{c+c1}{\PYZsh{} NEW: bind population to size}
            \PY{n}{size}\PY{o}{=}\PY{l+s+s1}{\PYZsq{}}\PY{l+s+s1}{population:Q}\PY{l+s+s1}{\PYZsq{}}\PY{p}{,}
            
            \PY{c+c1}{\PYZsh{} Or, try this alternate binding that clamps the range to [1, 400] instead of the default, [0, 400].}
            \PY{c+c1}{\PYZsh{} size=alt.Size(\PYZsq{}population:Q\PYZsq{}, scale=alt.Scale(range=[1, 400])),}
            
        \PY{p}{)}\PY{o}{.}\PY{n}{properties}\PY{p}{(}
            \PY{n}{width}\PY{o}{=}\PY{l+m+mi}{800}\PY{p}{,}
            \PY{n}{height}\PY{o}{=}\PY{l+m+mi}{800}
        \PY{p}{)}
        
        \PY{n+nb}{map}
\end{Verbatim}

\texttt{\color{outcolor}Out[{\color{outcolor}8}]:}
    
    \begin{center}
    \adjustimage{max size={0.9\linewidth}{0.9\paperheight}}{output_9_0.png}
    \end{center}
    { \hspace*{\fill} \\}
    

    That looks great! We can already start to see some interesting emergent
features from our data, such as the effect of rivers, mountains, and
other influences.

Notice the alternative solution in the above code. Go ahead and try the
other solution. Do you see a difference in the output? What do you think
is going on?

\begin{center}\rule{0.5\linewidth}{\linethickness}\end{center}

So far, we have bindings for \texttt{x}, \texttt{y}, and
\texttt{population}. In the code below, go ahead and bind
\texttt{density} to each place's color. \emph{Hint}: you declare a color
encoding with \texttt{color=}.

    \begin{Verbatim}[commandchars=\\\{\}]
{\color{incolor}In [{\color{incolor}48}]:} \PY{c+c1}{\PYZsh{}\PYZsh{} How would we adapt this to encode density by color?}
         \PY{c+c1}{\PYZsh{}\PYZsh{} Hint: you declare a color encoding with color=}
         
         \PY{n+nb}{map} \PY{o}{=} \PY{n}{alt}\PY{o}{.}\PY{n}{Chart}\PY{p}{(}\PY{n}{france}\PY{p}{)}\PY{o}{.}\PY{n}{mark\PYZus{}point}\PY{p}{(}\PY{n}{size}\PY{o}{=}\PY{l+m+mf}{0.5}\PY{p}{)}\PY{o}{.}\PY{n}{encode}\PY{p}{(}
             \PY{n}{x}\PY{o}{=}\PY{n}{alt}\PY{o}{.}\PY{n}{X}\PY{p}{(}\PY{l+s+s1}{\PYZsq{}}\PY{l+s+s1}{x:Q}\PY{l+s+s1}{\PYZsq{}}\PY{p}{,} \PY{n}{axis}\PY{o}{=}\PY{k+kc}{None}\PY{p}{)}\PY{p}{,}
             \PY{n}{y}\PY{o}{=}\PY{n}{alt}\PY{o}{.}\PY{n}{Y}\PY{p}{(}\PY{l+s+s1}{\PYZsq{}}\PY{l+s+s1}{y:Q}\PY{l+s+s1}{\PYZsq{}}\PY{p}{,} \PY{n}{axis}\PY{o}{=}\PY{k+kc}{None}\PY{p}{,} \PY{n}{scale}\PY{o}{=}\PY{n}{alt}\PY{o}{.}\PY{n}{Scale}\PY{p}{(}\PY{n}{zero}\PY{o}{=}\PY{k+kc}{False}\PY{p}{)}\PY{p}{)}\PY{p}{,}
             \PY{n}{size}\PY{o}{=}\PY{l+s+s1}{\PYZsq{}}\PY{l+s+s1}{population:Q}\PY{l+s+s1}{\PYZsq{}}\PY{p}{,}
         
             \PY{c+c1}{\PYZsh{} \PYZgt{}\PYZgt{}\PYZgt{} your code here \PYZlt{}\PYZlt{}\PYZlt{}}
             \PY{c+c1}{\PYZsh{}color=\PYZsq{}density:Q\PYZsq{}}
             \PY{n}{color}\PY{o}{=}\PY{n}{alt}\PY{o}{.}\PY{n}{Color}\PY{p}{(}\PY{l+s+s1}{\PYZsq{}}\PY{l+s+s1}{density:Q}\PY{l+s+s1}{\PYZsq{}}\PY{p}{,} \PY{n}{scale}\PY{o}{=}\PY{n}{alt}\PY{o}{.}\PY{n}{Scale}\PY{p}{(}\PY{n}{scheme}\PY{o}{=}\PY{l+s+s1}{\PYZsq{}}\PY{l+s+s1}{viridis}\PY{l+s+s1}{\PYZsq{}}\PY{p}{,} \PY{n}{domain}\PY{o}{=}\PY{p}{[}\PY{l+m+mi}{0}\PY{p}{,}\PY{l+m+mi}{2000}\PY{p}{]}\PY{p}{)}\PY{p}{)} \PY{c+c1}{\PYZsh{}, type=\PYZsq{}linear\PYZsq{}}
         
         \PY{p}{)}\PY{o}{.}\PY{n}{properties}\PY{p}{(}
             \PY{n}{width}\PY{o}{=}\PY{l+m+mi}{800}\PY{p}{,}
             \PY{n}{height}\PY{o}{=}\PY{l+m+mi}{800}
         \PY{p}{)}
         
         \PY{n+nb}{map}
\end{Verbatim}

\texttt{\color{outcolor}Out[{\color{outcolor}48}]:}
    
    \begin{center}
    \adjustimage{max size={0.9\linewidth}{0.9\paperheight}}{output_11_0.png}
    \end{center}
    { \hspace*{\fill} \\}
    

    \subsection{Altair}\label{altair}

So far, we've just gotten started with Altair. Before we dig a little
deeper, let's take a closer look at the shorthand notation and the
classes used in Altair.

By default, Altair tries to use reasonable defaults. If you're just
exploring data, they're often good enough. That will let you create
basic charts just by binding, say, \texttt{x} and/or \texttt{y} to some
column in your data table, e.g.
\texttt{x=\textquotesingle{}population:Q\textquotesingle{},\ y=\textquotesingle{}density:Q\textquotesingle{}}.

But if we need to override the defaults, we need to use a more
complicated syntax. Each of the encodings defines a configurable class.
Instead of using a shorthand string, as we did above, we can pass in an
instance of that class. That's what we did in the first step when we
switched from using

\begin{verbatim}
x='x:Q',
\end{verbatim}

to

\begin{verbatim}
x=alt.X('x:Q', axis=None)
\end{verbatim}

How did we know we could control the axis through the \texttt{axis=}
keyword? The
\href{https://altair-viz.github.io/user_guide/encoding.html\#encoding-channels}{Altair
documentation} lists the different encoding classes available. A click
on the \texttt{X} class on that page will take you to the documentation
that enumerates the different parameters you can override from their
default values. Notice that the third attribute down shows:

\begin{verbatim}
axis:anyOf(Axis, None)
\end{verbatim}

Thus, we can either pass in an instance of the Altair \texttt{Axis}
class or the value \texttt{None}---in which case, the documentation
explains, that axis will be removed from the chart.

\subsection{Compared to other
approaches}\label{compared-to-other-approaches}

Some of you may have heard of or even used other libraries for creating
statistical visualisations, including matplotlib, ggplot, Seaborn, and
others. Each of these libraries has a healthy community behind them and
are perfectly reasonable choices to use, and most of them will use
similar concepts to what we are seeing here with Altair.

In this class, however, he use Altair for several reasons:

\begin{itemize}
\tightlist
\item
  It has a nice mapping of Bertin's and Wilkinsin's concepts for
  statistical visualisation (that we saw in class earlier today!) baked
  into their APIs,
\item
  It uses a declarative rather than procedural design approach,
\item
  You can easily export your visualisations to PNG, SVG, or even
  interactive web pages, and
\item
  It generates Vega-Lite descriptions that are then rendered using D3,
  which we'll see later on in the semester.
\end{itemize}

That second point probably needs a little more explanation. In a
procedural syntax, you could imagine saying something like:

\begin{verbatim}
for each data point:
    draw circle of size datum.population at datum.x, datum.y
\end{verbatim}

in our declarative approach, we'd say instead:

\begin{verbatim}
make a chart for data where:
    position is (datum.x, datum.y) and size is datum.population
\end{verbatim}

At first glance, those might look awfully similar, but the key
difference is that, in the first case, we say \emph{how} to do what we
want, whereas in the second case, we simply say \emph{what} we want.

Altair, D3, and Vega all use this second, declarative approach. It takes
some getting used to, but it is part of what makes these libraries
compelling.

    \subsection{Tooltips}\label{tooltips}

So far, our visualisations are not particularly interactive. Let's go
ahead and change that.

The barest, most minimal interaction we can add is to show a tooltip
when we hover over a place. In Altair, we can use the
\href{https://altair-viz.github.io/user_guide/generated/channels/altair.Tooltip.html}{Tooltip}
encoding.

    \begin{Verbatim}[commandchars=\\\{\}]
{\color{incolor}In [{\color{incolor}76}]:} \PY{c+c1}{\PYZsh{} Let\PYZsq{}s add tooltips}
         
         \PY{n+nb}{map} \PY{o}{=} \PY{n+nb}{map}\PY{o}{.}\PY{n}{encode}\PY{p}{(}
             \PY{n}{tooltip}\PY{o}{=}\PY{p}{[}\PY{l+s+s1}{\PYZsq{}}\PY{l+s+s1}{place:N}\PY{l+s+s1}{\PYZsq{}}\PY{p}{,} \PY{l+s+s1}{\PYZsq{}}\PY{l+s+s1}{population:Q}\PY{l+s+s1}{\PYZsq{}}\PY{p}{,} \PY{l+s+s1}{\PYZsq{}}\PY{l+s+s1}{density:Q}\PY{l+s+s1}{\PYZsq{}}\PY{p}{]}\PY{p}{,}
             \PY{n}{color}\PY{o}{=}\PY{n}{alt}\PY{o}{.}\PY{n}{Color}\PY{p}{(}\PY{l+s+s1}{\PYZsq{}}\PY{l+s+s1}{density:Q}\PY{l+s+s1}{\PYZsq{}}\PY{p}{,} \PY{n}{scale}\PY{o}{=}\PY{n}{alt}\PY{o}{.}\PY{n}{Scale}\PY{p}{(}\PY{n}{scheme}\PY{o}{=}\PY{l+s+s1}{\PYZsq{}}\PY{l+s+s1}{viridis}\PY{l+s+s1}{\PYZsq{}}\PY{p}{,} \PY{n}{domain}\PY{o}{=}\PY{p}{[}\PY{l+m+mi}{0}\PY{p}{,}\PY{l+m+mi}{2000}\PY{p}{]}\PY{p}{)}\PY{p}{)} \PY{c+c1}{\PYZsh{}, type=\PYZsq{}linear\PYZsq{}}
         \PY{p}{)}
         
         \PY{n+nb}{map}
\end{Verbatim}

\texttt{\color{outcolor}Out[{\color{outcolor}76}]:}
    
    \begin{center}
    \adjustimage{max size={0.9\linewidth}{0.9\paperheight}}{output_14_0.png}
    \end{center}
    { \hspace*{\fill} \\}
    

    \subsection{Multiple views}\label{multiple-views}

It would be nice to see a different representationof the distribution of
population and densities in France. Let's go ahead and add in a
``histogram'' of population and density\ldots{}.

A histogram is just a bar chart of the different bins of values, so
let's create a chart using \emph{bars} for the \emph{marks}. We want to
show the population on the \emph{x} axis and the number of people in
that bin on the \emph{y} axis:

    \begin{Verbatim}[commandchars=\\\{\}]
{\color{incolor}In [{\color{incolor}77}]:} \PY{n}{population} \PY{o}{=} \PY{n}{alt}\PY{o}{.}\PY{n}{Chart}\PY{p}{(}\PY{n}{france}\PY{p}{,} \PY{n}{width}\PY{o}{=}\PY{l+m+mi}{800}\PY{p}{,} \PY{n}{height}\PY{o}{=}\PY{l+m+mi}{100}\PY{p}{)}\PY{o}{.}\PY{n}{mark\PYZus{}bar}\PY{p}{(}\PY{p}{)}\PY{o}{.}\PY{n}{encode}\PY{p}{(}
             \PY{n}{x}\PY{o}{=}\PY{l+s+s1}{\PYZsq{}}\PY{l+s+s1}{population:Q}\PY{l+s+s1}{\PYZsq{}}\PY{p}{,}
             \PY{n}{y}\PY{o}{=}\PY{l+s+s1}{\PYZsq{}}\PY{l+s+s1}{count(population):Q}\PY{l+s+s1}{\PYZsq{}}\PY{p}{,}
         \PY{p}{)}
         
         \PY{n}{population}
\end{Verbatim}

\texttt{\color{outcolor}Out[{\color{outcolor}77}]:}
    
    \begin{center}
    \adjustimage{max size={0.9\linewidth}{0.9\paperheight}}{output_16_0.png}
    \end{center}
    { \hspace*{\fill} \\}
    

    That's close, but we're not really binning our data. Let's go ahead and
create slices (or bins) of the data on the \emph{x} axis. (In French,
we'd call these \emph{tranches}.)

    \begin{Verbatim}[commandchars=\\\{\}]
{\color{incolor}In [{\color{incolor}78}]:} \PY{n}{population} \PY{o}{=} \PY{n}{alt}\PY{o}{.}\PY{n}{Chart}\PY{p}{(}\PY{n}{france}\PY{p}{,} \PY{n}{width}\PY{o}{=}\PY{l+m+mi}{800}\PY{p}{,} \PY{n}{height}\PY{o}{=}\PY{l+m+mi}{100}\PY{p}{)}\PY{o}{.}\PY{n}{mark\PYZus{}bar}\PY{p}{(}\PY{p}{)}\PY{o}{.}\PY{n}{encode}\PY{p}{(}
             \PY{n}{x}\PY{o}{=}\PY{n}{alt}\PY{o}{.}\PY{n}{X}\PY{p}{(}\PY{l+s+s1}{\PYZsq{}}\PY{l+s+s1}{population:Q}\PY{l+s+s1}{\PYZsq{}}\PY{p}{,} \PY{n+nb}{bin}\PY{o}{=}\PY{k+kc}{True}\PY{p}{)}\PY{p}{,}
             \PY{n}{y}\PY{o}{=}\PY{l+s+s1}{\PYZsq{}}\PY{l+s+s1}{count(population):Q}\PY{l+s+s1}{\PYZsq{}}\PY{p}{,}
         \PY{p}{)}
         
         \PY{n}{population}
\end{Verbatim}

\texttt{\color{outcolor}Out[{\color{outcolor}78}]:}
    
    \begin{center}
    \adjustimage{max size={0.9\linewidth}{0.9\paperheight}}{output_18_0.png}
    \end{center}
    { \hspace*{\fill} \\}
    

    That's better, but the default bins are a bit too big. Let's override
the number of bins. Also, let's show the number of people in each bin
instead of the number of places by replacing the \texttt{count()} with
the \texttt{sum()}.

    \begin{Verbatim}[commandchars=\\\{\}]
{\color{incolor}In [{\color{incolor}79}]:} \PY{n}{population} \PY{o}{=} \PY{n}{alt}\PY{o}{.}\PY{n}{Chart}\PY{p}{(}\PY{n}{france}\PY{p}{,} \PY{n}{width}\PY{o}{=}\PY{l+m+mi}{800}\PY{p}{,} \PY{n}{height}\PY{o}{=}\PY{l+m+mi}{100}\PY{p}{)}\PY{o}{.}\PY{n}{mark\PYZus{}bar}\PY{p}{(}\PY{p}{)}\PY{o}{.}\PY{n}{encode}\PY{p}{(}
             \PY{n}{x}\PY{o}{=}\PY{n}{alt}\PY{o}{.}\PY{n}{X}\PY{p}{(}\PY{l+s+s1}{\PYZsq{}}\PY{l+s+s1}{population:Q}\PY{l+s+s1}{\PYZsq{}}\PY{p}{,} \PY{n+nb}{bin}\PY{o}{=}\PY{n}{alt}\PY{o}{.}\PY{n}{Bin}\PY{p}{(}\PY{n}{maxbins}\PY{o}{=}\PY{l+m+mi}{60}\PY{p}{)}\PY{p}{)}\PY{p}{,}
             \PY{n}{y}\PY{o}{=}\PY{l+s+s1}{\PYZsq{}}\PY{l+s+s1}{sum(population):Q}\PY{l+s+s1}{\PYZsq{}}\PY{p}{,}
         \PY{p}{)}
         
         \PY{n}{population}
\end{Verbatim}

\texttt{\color{outcolor}Out[{\color{outcolor}79}]:}
    
    \begin{center}
    \adjustimage{max size={0.9\linewidth}{0.9\paperheight}}{output_20_0.png}
    \end{center}
    { \hspace*{\fill} \\}
    

    \subsubsection{Density histogram}\label{density-histogram}

How would you make a histogram of the densities? Try changing the code
below do make it into a \emph{density} histogram instead of showing the
population.

    \begin{Verbatim}[commandchars=\\\{\}]
{\color{incolor}In [{\color{incolor}80}]:} \PY{n}{density} \PY{o}{=} \PY{n}{alt}\PY{o}{.}\PY{n}{Chart}\PY{p}{(}\PY{n}{france}\PY{p}{,} \PY{n}{width}\PY{o}{=}\PY{l+m+mi}{800}\PY{p}{,} \PY{n}{height}\PY{o}{=}\PY{l+m+mi}{100}\PY{p}{)}\PY{o}{.}\PY{n}{mark\PYZus{}bar}\PY{p}{(}\PY{p}{)}\PY{o}{.}\PY{n}{encode}\PY{p}{(}
             \PY{n}{x}\PY{o}{=}\PY{n}{alt}\PY{o}{.}\PY{n}{X}\PY{p}{(}\PY{l+s+s1}{\PYZsq{}}\PY{l+s+s1}{density:Q}\PY{l+s+s1}{\PYZsq{}}\PY{p}{,} \PY{n+nb}{bin}\PY{o}{=}\PY{n}{alt}\PY{o}{.}\PY{n}{Bin}\PY{p}{(}\PY{n}{maxbins}\PY{o}{=}\PY{l+m+mi}{60}\PY{p}{)}\PY{p}{)}\PY{p}{,}
             \PY{n}{y}\PY{o}{=}\PY{l+s+s1}{\PYZsq{}}\PY{l+s+s1}{sum(density):Q}\PY{l+s+s1}{\PYZsq{}}\PY{p}{,}
         \PY{p}{)}
         
         \PY{n}{density}
\end{Verbatim}

\texttt{\color{outcolor}Out[{\color{outcolor}80}]:}
    
    \begin{center}
    \adjustimage{max size={0.9\linewidth}{0.9\paperheight}}{output_22_0.png}
    \end{center}
    { \hspace*{\fill} \\}
    

    \section{Making a multi-view visualisation (e.g.,
``dashboard'')}\label{making-a-multi-view-visualisation-e.g.-dashboard}

We now have all of the pieces we need to make a multi-view visualisation
(what we called a ``dashboard'' in Tableau).

In Altair, you can combine multiple charts manually using the
\texttt{\&} and \texttt{\textbar{}} operators to combine them
\emph{vertically} or \emph{horizontally}, respectively.

    \begin{Verbatim}[commandchars=\\\{\}]
{\color{incolor}In [{\color{incolor}81}]:} \PY{n}{population} \PY{o}{\PYZam{}} \PY{n}{density} \PY{o}{\PYZam{}} \PY{n+nb}{map}
\end{Verbatim}

\texttt{\color{outcolor}Out[{\color{outcolor}81}]:}
    
    \begin{center}
    \adjustimage{max size={0.9\linewidth}{0.9\paperheight}}{output_24_0.png}
    \end{center}
    { \hspace*{\fill} \\}
    

    \subsection{Other interactions}\label{other-interactions}

Beyond tooltips, we want to be able to \emph{select} data points,
\emph{filter} out data, and to be able to \emph{link views} so that,
say, a selection in one might filter what is shown in another (or alter
its presentation in some other way).

Let's start with selections. There are three kinds of selections:

\begin{itemize}
\tightlist
\item
  intervals --- a range of values
\item
  single - a single value
\item
  multi - multiple values
\end{itemize}

Intervals let the user ``brush'' over a range of values to select them.
Single selection select a single data point, while multi selections let
the use combine single selections (by default with the shift key).
\href{https://altair-viz.github.io/user_guide/interactions.html}{For
more info, see the Altair selections guide}

Here, let's add a rectangular selection brush to our map.

    \begin{Verbatim}[commandchars=\\\{\}]
{\color{incolor}In [{\color{incolor}82}]:} \PY{n}{brush} \PY{o}{=} \PY{n}{alt}\PY{o}{.}\PY{n}{selection\PYZus{}interval}\PY{p}{(}\PY{p}{)}
         \PY{n+nb}{map}\PY{o}{.}\PY{n}{add\PYZus{}selection}\PY{p}{(}\PY{n}{brush}\PY{p}{)}
\end{Verbatim}

\texttt{\color{outcolor}Out[{\color{outcolor}82}]:}
    
    \begin{center}
    \adjustimage{max size={0.9\linewidth}{0.9\paperheight}}{output_26_0.png}
    \end{center}
    { \hspace*{\fill} \\}
    

    Now try it out. Try dragging a box around a part of the map.

\begin{center}\rule{0.5\linewidth}{\linethickness}\end{center}

Huh, that sort of works. We see a selection appear, but it doesn't
\emph{do} anything with it. Let's fix that.

    \begin{Verbatim}[commandchars=\\\{\}]
{\color{incolor}In [{\color{incolor}86}]:} \PY{c+c1}{\PYZsh{} Let\PYZsq{}s gray out unselected places}
         
         \PY{n+nb}{map} \PY{o}{=} \PY{n}{alt}\PY{o}{.}\PY{n}{Chart}\PY{p}{(}\PY{n}{france}\PY{p}{,} \PY{n}{width}\PY{o}{=}\PY{l+m+mi}{800}\PY{p}{,} \PY{n}{height}\PY{o}{=}\PY{l+m+mi}{800}\PY{p}{)}\PY{o}{.}\PY{n}{mark\PYZus{}point}\PY{p}{(}\PY{n}{size}\PY{o}{=}\PY{l+m+mi}{1}\PY{p}{)}\PY{o}{.}\PY{n}{encode}\PY{p}{(}
             \PY{n}{x}\PY{o}{=}\PY{n}{alt}\PY{o}{.}\PY{n}{X}\PY{p}{(}\PY{l+s+s1}{\PYZsq{}}\PY{l+s+s1}{x:Q}\PY{l+s+s1}{\PYZsq{}}\PY{p}{,} \PY{n}{axis}\PY{o}{=}\PY{k+kc}{None}\PY{p}{)}\PY{p}{,}
             \PY{n}{y}\PY{o}{=}\PY{n}{alt}\PY{o}{.}\PY{n}{Y}\PY{p}{(}\PY{l+s+s1}{\PYZsq{}}\PY{l+s+s1}{y:Q}\PY{l+s+s1}{\PYZsq{}}\PY{p}{,} \PY{n}{axis}\PY{o}{=}\PY{k+kc}{None}\PY{p}{,} \PY{n}{scale}\PY{o}{=}\PY{n}{alt}\PY{o}{.}\PY{n}{Scale}\PY{p}{(}\PY{n}{zero}\PY{o}{=}\PY{k+kc}{False}\PY{p}{)}\PY{p}{)}\PY{p}{,}
             \PY{n}{size}\PY{o}{=}\PY{l+s+s1}{\PYZsq{}}\PY{l+s+s1}{population:Q}\PY{l+s+s1}{\PYZsq{}}\PY{p}{,}
             \PY{n}{tooltip}\PY{o}{=}\PY{p}{[}\PY{l+s+s1}{\PYZsq{}}\PY{l+s+s1}{place:N}\PY{l+s+s1}{\PYZsq{}}\PY{p}{,} \PY{l+s+s1}{\PYZsq{}}\PY{l+s+s1}{population:Q}\PY{l+s+s1}{\PYZsq{}}\PY{p}{,} \PY{l+s+s1}{\PYZsq{}}\PY{l+s+s1}{density:Q}\PY{l+s+s1}{\PYZsq{}}\PY{p}{]}\PY{p}{,}
             \PY{n}{color}\PY{o}{=}\PY{n}{alt}\PY{o}{.}\PY{n}{condition}\PY{p}{(}\PY{n}{brush}\PY{p}{,} \PY{l+s+s1}{\PYZsq{}}\PY{l+s+s1}{density:Q}\PY{l+s+s1}{\PYZsq{}}\PY{p}{,} \PY{n}{alt}\PY{o}{.}\PY{n}{value}\PY{p}{(}\PY{l+s+s1}{\PYZsq{}}\PY{l+s+s1}{lightgrey}\PY{l+s+s1}{\PYZsq{}}\PY{p}{)}\PY{p}{,} \PY{n}{scale}\PY{o}{=}\PY{n}{alt}\PY{o}{.}\PY{n}{Scale}\PY{p}{(}\PY{n}{scheme}\PY{o}{=}\PY{l+s+s1}{\PYZsq{}}\PY{l+s+s1}{viridis}\PY{l+s+s1}{\PYZsq{}}\PY{p}{,} \PY{n}{domain}\PY{o}{=}\PY{p}{[}\PY{l+m+mi}{0}\PY{p}{,}\PY{l+m+mi}{2000}\PY{p}{]}\PY{p}{)}\PY{p}{)}\PY{p}{,}
         \PY{p}{)}\PY{o}{.}\PY{n}{add\PYZus{}selection}\PY{p}{(}\PY{n}{brush}\PY{p}{)}
         
         \PY{n+nb}{map}
\end{Verbatim}

\texttt{\color{outcolor}Out[{\color{outcolor}86}]:}
    
    \begin{center}
    \adjustimage{max size={0.9\linewidth}{0.9\paperheight}}{output_28_0.png}
    \end{center}
    { \hspace*{\fill} \\}
    

    Well, that seems to work. It's a little slow (because we have a
\emph{lot} of \emph{large} views open in this notebook), but moreover,
it doesn't really \emph{do} much useful.

Let's go ahead and link our selection to the histogram. Notice that all
we need to do to filter the histogram based on the selection is to add a
\texttt{.transform\_filter()} to the histogram that uses the map's
selection \texttt{brush}. (See the last line of the histogram.)

    \begin{Verbatim}[commandchars=\\\{\}]
{\color{incolor}In [{\color{incolor}85}]:} \PY{n}{population} \PY{o}{=} \PY{n}{population} \PY{o}{=} \PY{n}{alt}\PY{o}{.}\PY{n}{Chart}\PY{p}{(}\PY{n}{france}\PY{p}{,} \PY{n}{width}\PY{o}{=}\PY{l+m+mi}{800}\PY{p}{,} \PY{n}{height}\PY{o}{=}\PY{l+m+mi}{100}\PY{p}{)}\PY{o}{.}\PY{n}{mark\PYZus{}bar}\PY{p}{(}\PY{p}{)}\PY{o}{.}\PY{n}{encode}\PY{p}{(}
             \PY{n}{x}\PY{o}{=}\PY{n}{alt}\PY{o}{.}\PY{n}{X}\PY{p}{(}\PY{l+s+s1}{\PYZsq{}}\PY{l+s+s1}{population:Q}\PY{l+s+s1}{\PYZsq{}}\PY{p}{,} \PY{n+nb}{bin}\PY{o}{=}\PY{n}{alt}\PY{o}{.}\PY{n}{Bin}\PY{p}{(}\PY{n}{maxbins}\PY{o}{=}\PY{l+m+mi}{60}\PY{p}{)}\PY{p}{)}\PY{p}{,}
             \PY{n}{y}\PY{o}{=}\PY{l+s+s1}{\PYZsq{}}\PY{l+s+s1}{sum(population):Q}\PY{l+s+s1}{\PYZsq{}}\PY{p}{,}
         \PY{p}{)}\PY{o}{.}\PY{n}{transform\PYZus{}filter}\PY{p}{(}\PY{n}{brush}\PY{p}{)}
         
         \PY{n}{population} \PY{o}{\PYZam{}} \PY{n+nb}{map}
\end{Verbatim}

\texttt{\color{outcolor}Out[{\color{outcolor}85}]:}
    
    \begin{center}
    \adjustimage{max size={0.9\linewidth}{0.9\paperheight}}{output_30_0.png}
    \end{center}
    { \hspace*{\fill} \\}
    

    That's great, but it would be nice if we could link the views both ways:
a selection in the map updates the histogram or a selection in the
histogram updates the map.

To do that, we'll create two selections:

    \begin{Verbatim}[commandchars=\\\{\}]
{\color{incolor}In [{\color{incolor}69}]:} \PY{n}{brush} \PY{o}{=} \PY{n}{alt}\PY{o}{.}\PY{n}{selection\PYZus{}interval}\PY{p}{(}\PY{p}{)}
         \PY{n}{pop\PYZus{}selection} \PY{o}{=} \PY{n}{alt}\PY{o}{.}\PY{n}{selection\PYZus{}interval}\PY{p}{(}\PY{n}{encodings}\PY{o}{=}\PY{p}{[}\PY{l+s+s1}{\PYZsq{}}\PY{l+s+s1}{x}\PY{l+s+s1}{\PYZsq{}}\PY{p}{]}\PY{p}{)}
\end{Verbatim}


    Notice that the new selection uses an optional \texttt{encodings}
parameter to indicate that it only selects data items along the
\texttt{x} encoding from the chart (which we have mapped to the
\texttt{x} data attribute).

Now update the code below to link the two charts using the new brush.
(This code will show an error if you haven't linked things in
correctly.)

    \begin{Verbatim}[commandchars=\\\{\}]
{\color{incolor}In [{\color{incolor}101}]:} \PY{n+nb}{map} \PY{o}{=} \PY{n}{alt}\PY{o}{.}\PY{n}{Chart}\PY{p}{(}\PY{n}{france}\PY{p}{,} \PY{n}{width}\PY{o}{=}\PY{l+m+mi}{800}\PY{p}{,} \PY{n}{height}\PY{o}{=}\PY{l+m+mi}{800}\PY{p}{)}\PY{o}{.}\PY{n}{mark\PYZus{}point}\PY{p}{(}\PY{n}{size}\PY{o}{=}\PY{l+m+mi}{1}\PY{p}{)}\PY{o}{.}\PY{n}{encode}\PY{p}{(}
              \PY{n}{x}\PY{o}{=}\PY{n}{alt}\PY{o}{.}\PY{n}{X}\PY{p}{(}\PY{l+s+s1}{\PYZsq{}}\PY{l+s+s1}{x:Q}\PY{l+s+s1}{\PYZsq{}}\PY{p}{,} \PY{n}{axis}\PY{o}{=}\PY{k+kc}{None}\PY{p}{)}\PY{p}{,}
              \PY{n}{y}\PY{o}{=}\PY{n}{alt}\PY{o}{.}\PY{n}{Y}\PY{p}{(}\PY{l+s+s1}{\PYZsq{}}\PY{l+s+s1}{y:Q}\PY{l+s+s1}{\PYZsq{}}\PY{p}{,} \PY{n}{axis}\PY{o}{=}\PY{k+kc}{None}\PY{p}{,} \PY{n}{scale}\PY{o}{=}\PY{n}{alt}\PY{o}{.}\PY{n}{Scale}\PY{p}{(}\PY{n}{zero}\PY{o}{=}\PY{k+kc}{False}\PY{p}{)}\PY{p}{)}\PY{p}{,}
              \PY{n}{size}\PY{o}{=}\PY{l+s+s1}{\PYZsq{}}\PY{l+s+s1}{population:Q}\PY{l+s+s1}{\PYZsq{}}\PY{p}{,}
              \PY{n}{color}\PY{o}{=}\PY{n}{alt}\PY{o}{.}\PY{n}{condition}\PY{p}{(}\PY{n}{pop\PYZus{}selection}\PY{p}{,} \PY{l+s+s1}{\PYZsq{}}\PY{l+s+s1}{density:Q}\PY{l+s+s1}{\PYZsq{}}\PY{p}{,} \PY{n}{alt}\PY{o}{.}\PY{n}{value}\PY{p}{(}\PY{l+s+s1}{\PYZsq{}}\PY{l+s+s1}{lightgrey}\PY{l+s+s1}{\PYZsq{}}\PY{p}{)}\PY{p}{,} \PY{n}{scale}\PY{o}{=}\PY{n}{alt}\PY{o}{.}\PY{n}{Scale}\PY{p}{(}\PY{n}{scheme}\PY{o}{=}\PY{l+s+s1}{\PYZsq{}}\PY{l+s+s1}{viridis}\PY{l+s+s1}{\PYZsq{}}\PY{p}{,} \PY{n}{domain}\PY{o}{=}\PY{p}{[}\PY{l+m+mi}{0}\PY{p}{,}\PY{l+m+mi}{2000}\PY{p}{]}\PY{p}{)}\PY{p}{)}\PY{p}{,}
          \PY{p}{)}\PY{o}{.}\PY{n}{add\PYZus{}selection}\PY{p}{(}\PY{n}{brush}\PY{p}{)}
          
          \PY{n}{population} \PY{o}{=} \PY{n}{alt}\PY{o}{.}\PY{n}{Chart}\PY{p}{(}\PY{n}{france}\PY{p}{,} \PY{n}{width}\PY{o}{=}\PY{l+m+mi}{800}\PY{p}{,} \PY{n}{height}\PY{o}{=}\PY{l+m+mi}{100}\PY{p}{)}\PY{o}{.}\PY{n}{mark\PYZus{}bar}\PY{p}{(}\PY{p}{)}\PY{o}{.}\PY{n}{encode}\PY{p}{(}
              \PY{n}{x}\PY{o}{=}\PY{n}{alt}\PY{o}{.}\PY{n}{X}\PY{p}{(}\PY{l+s+s1}{\PYZsq{}}\PY{l+s+s1}{population:Q}\PY{l+s+s1}{\PYZsq{}}\PY{p}{,} \PY{n+nb}{bin}\PY{o}{=}\PY{n}{alt}\PY{o}{.}\PY{n}{Bin}\PY{p}{(}\PY{n}{maxbins}\PY{o}{=}\PY{l+m+mi}{250}\PY{p}{)}\PY{p}{)}\PY{p}{,}
              \PY{n}{y}\PY{o}{=}\PY{l+s+s1}{\PYZsq{}}\PY{l+s+s1}{sum(population):Q}\PY{l+s+s1}{\PYZsq{}}
          \PY{p}{)}\PY{o}{.}\PY{n}{add\PYZus{}selection}\PY{p}{(}\PY{n}{pop\PYZus{}selection}\PY{p}{)}\PY{o}{.}\PY{n}{transform\PYZus{}filter}\PY{p}{(}\PY{n}{brush}\PY{p}{)}
          
          \PY{n}{population} \PY{o}{\PYZam{}} \PY{n+nb}{map}
\end{Verbatim}

\texttt{\color{outcolor}Out[{\color{outcolor}101}]:}
    
    \begin{center}
    \adjustimage{max size={0.9\linewidth}{0.9\paperheight}}{output_34_0.png}
    \end{center}
    { \hspace*{\fill} \\}
    

    \subsection{Exercises}\label{exercises}

\begin{itemize}
\tightlist
\item
  Add linked selection to density ``histogram''
\item
  Try different encodings: explore
  \href{https://altair-viz.github.io/user_guide/generated/core/altair.Scale.html\#altair.Scale}{alt.Scale},
  \href{https://altair-viz.github.io/user_guide/generated/channels/altair.Color.html\#altair.Color}{alt.Color},
  etc.
\item
  Try different marks: mark\_rect, mark\_tick, mark\_circle, \ldots{}.
  \href{https://altair-viz.github.io/user_guide/marks.html}{Marks
  documentation}
\item
  Try making a heatmap of places in France.
\end{itemize}

    \begin{Verbatim}[commandchars=\\\{\}]
{\color{incolor}In [{\color{incolor}3}]:} \PY{c+c1}{\PYZsh{} Add linked selection to density “histogram”}
        
        \PY{n}{brush} \PY{o}{=} \PY{n}{alt}\PY{o}{.}\PY{n}{selection\PYZus{}interval}\PY{p}{(}\PY{p}{)}
        \PY{n}{dens\PYZus{}selection} \PY{o}{=} \PY{n}{alt}\PY{o}{.}\PY{n}{selection\PYZus{}interval}\PY{p}{(}\PY{n}{encodings}\PY{o}{=}\PY{p}{[}\PY{l+s+s1}{\PYZsq{}}\PY{l+s+s1}{x}\PY{l+s+s1}{\PYZsq{}}\PY{p}{]}\PY{p}{)}
        
        \PY{n+nb}{map} \PY{o}{=} \PY{n}{alt}\PY{o}{.}\PY{n}{Chart}\PY{p}{(}\PY{n}{france}\PY{p}{,} \PY{n}{width}\PY{o}{=}\PY{l+m+mi}{800}\PY{p}{,} \PY{n}{height}\PY{o}{=}\PY{l+m+mi}{800}\PY{p}{)}\PY{o}{.}\PY{n}{mark\PYZus{}point}\PY{p}{(}\PY{n}{size}\PY{o}{=}\PY{l+m+mi}{1}\PY{p}{)}\PY{o}{.}\PY{n}{encode}\PY{p}{(}
            \PY{n}{x}\PY{o}{=}\PY{n}{alt}\PY{o}{.}\PY{n}{X}\PY{p}{(}\PY{l+s+s1}{\PYZsq{}}\PY{l+s+s1}{x:Q}\PY{l+s+s1}{\PYZsq{}}\PY{p}{,} \PY{n}{axis}\PY{o}{=}\PY{k+kc}{None}\PY{p}{)}\PY{p}{,}
            \PY{n}{y}\PY{o}{=}\PY{n}{alt}\PY{o}{.}\PY{n}{Y}\PY{p}{(}\PY{l+s+s1}{\PYZsq{}}\PY{l+s+s1}{y:Q}\PY{l+s+s1}{\PYZsq{}}\PY{p}{,} \PY{n}{axis}\PY{o}{=}\PY{k+kc}{None}\PY{p}{,} \PY{n}{scale}\PY{o}{=}\PY{n}{alt}\PY{o}{.}\PY{n}{Scale}\PY{p}{(}\PY{n}{zero}\PY{o}{=}\PY{k+kc}{False}\PY{p}{)}\PY{p}{)}\PY{p}{,}
            \PY{n}{size}\PY{o}{=}\PY{l+s+s1}{\PYZsq{}}\PY{l+s+s1}{population:Q}\PY{l+s+s1}{\PYZsq{}}\PY{p}{,}
            \PY{n}{tooltip}\PY{o}{=}\PY{p}{[}\PY{l+s+s1}{\PYZsq{}}\PY{l+s+s1}{place:N}\PY{l+s+s1}{\PYZsq{}}\PY{p}{,} \PY{l+s+s1}{\PYZsq{}}\PY{l+s+s1}{population:Q}\PY{l+s+s1}{\PYZsq{}}\PY{p}{,} \PY{l+s+s1}{\PYZsq{}}\PY{l+s+s1}{density:Q}\PY{l+s+s1}{\PYZsq{}}\PY{p}{]}\PY{p}{,}
            \PY{n}{color}\PY{o}{=}\PY{n}{alt}\PY{o}{.}\PY{n}{condition}\PY{p}{(}\PY{n}{dens\PYZus{}selection}\PY{p}{,} \PY{l+s+s1}{\PYZsq{}}\PY{l+s+s1}{density:Q}\PY{l+s+s1}{\PYZsq{}}\PY{p}{,} \PY{n}{alt}\PY{o}{.}\PY{n}{value}\PY{p}{(}\PY{l+s+s1}{\PYZsq{}}\PY{l+s+s1}{lightgrey}\PY{l+s+s1}{\PYZsq{}}\PY{p}{)}\PY{p}{,} \PY{n}{scale}\PY{o}{=}\PY{n}{alt}\PY{o}{.}\PY{n}{Scale}\PY{p}{(}\PY{n}{scheme}\PY{o}{=}\PY{l+s+s1}{\PYZsq{}}\PY{l+s+s1}{viridis}\PY{l+s+s1}{\PYZsq{}}\PY{p}{,} \PY{n}{domain}\PY{o}{=}\PY{p}{[}\PY{l+m+mi}{0}\PY{p}{,}\PY{l+m+mi}{2000}\PY{p}{]}\PY{p}{)}\PY{p}{)}\PY{p}{,}
        \PY{p}{)}\PY{o}{.}\PY{n}{add\PYZus{}selection}\PY{p}{(}\PY{n}{brush}\PY{p}{)}
        
        \PY{n}{density} \PY{o}{=} \PY{n}{alt}\PY{o}{.}\PY{n}{Chart}\PY{p}{(}\PY{n}{france}\PY{p}{,} \PY{n}{width}\PY{o}{=}\PY{l+m+mi}{800}\PY{p}{,} \PY{n}{height}\PY{o}{=}\PY{l+m+mi}{100}\PY{p}{)}\PY{o}{.}\PY{n}{mark\PYZus{}bar}\PY{p}{(}\PY{p}{)}\PY{o}{.}\PY{n}{encode}\PY{p}{(}
            \PY{n}{x}\PY{o}{=}\PY{n}{alt}\PY{o}{.}\PY{n}{X}\PY{p}{(}\PY{l+s+s1}{\PYZsq{}}\PY{l+s+s1}{density:Q}\PY{l+s+s1}{\PYZsq{}}\PY{p}{,} \PY{n+nb}{bin}\PY{o}{=}\PY{n}{alt}\PY{o}{.}\PY{n}{Bin}\PY{p}{(}\PY{n}{maxbins}\PY{o}{=}\PY{l+m+mi}{250}\PY{p}{)}\PY{p}{)}\PY{p}{,}
            \PY{n}{y}\PY{o}{=}\PY{l+s+s1}{\PYZsq{}}\PY{l+s+s1}{sum(density):Q}\PY{l+s+s1}{\PYZsq{}}
        \PY{p}{)}\PY{o}{.}\PY{n}{add\PYZus{}selection}\PY{p}{(}\PY{n}{dens\PYZus{}selection}\PY{p}{)}\PY{o}{.}\PY{n}{transform\PYZus{}filter}\PY{p}{(}\PY{n}{brush}\PY{p}{)}
        
        \PY{n}{density} \PY{o}{\PYZam{}} \PY{n+nb}{map}
\end{Verbatim}

\texttt{\color{outcolor}Out[{\color{outcolor}3}]:}
    
    \begin{center}
    \adjustimage{max size={0.9\linewidth}{0.9\paperheight}}{output_36_0.png}
    \end{center}
    { \hspace*{\fill} \\}
    

    \begin{Verbatim}[commandchars=\\\{\}]
{\color{incolor}In [{\color{incolor}130}]:} \PY{c+c1}{\PYZsh{} Try different encodings: explore alt.Scale, alt.Color, etc.}
          
          \PY{n+nb}{map} \PY{o}{=} \PY{n}{alt}\PY{o}{.}\PY{n}{Chart}\PY{p}{(}\PY{n}{france}\PY{p}{,} \PY{n}{width}\PY{o}{=}\PY{l+m+mi}{800}\PY{p}{,} \PY{n}{height}\PY{o}{=}\PY{l+m+mi}{800}\PY{p}{)}\PY{o}{.}\PY{n}{mark\PYZus{}point}\PY{p}{(}\PY{n}{size}\PY{o}{=}\PY{l+m+mi}{1}\PY{p}{)}\PY{o}{.}\PY{n}{encode}\PY{p}{(}
              \PY{n}{x}\PY{o}{=}\PY{n}{alt}\PY{o}{.}\PY{n}{X}\PY{p}{(}\PY{l+s+s1}{\PYZsq{}}\PY{l+s+s1}{x:Q}\PY{l+s+s1}{\PYZsq{}}\PY{p}{,} \PY{n}{axis}\PY{o}{=}\PY{k+kc}{None}\PY{p}{)}\PY{p}{,}
              \PY{n}{y}\PY{o}{=}\PY{n}{alt}\PY{o}{.}\PY{n}{Y}\PY{p}{(}\PY{l+s+s1}{\PYZsq{}}\PY{l+s+s1}{y:Q}\PY{l+s+s1}{\PYZsq{}}\PY{p}{,} \PY{n}{axis}\PY{o}{=}\PY{k+kc}{None}\PY{p}{,} \PY{n}{scale}\PY{o}{=}\PY{n}{alt}\PY{o}{.}\PY{n}{Scale}\PY{p}{(}\PY{n}{zero}\PY{o}{=}\PY{k+kc}{False}\PY{p}{)}\PY{p}{)}\PY{p}{,}
              \PY{n}{size}\PY{o}{=}\PY{l+s+s1}{\PYZsq{}}\PY{l+s+s1}{population:Q}\PY{l+s+s1}{\PYZsq{}}\PY{p}{,}
              \PY{n}{tooltip}\PY{o}{=}\PY{p}{[}\PY{l+s+s1}{\PYZsq{}}\PY{l+s+s1}{place:N}\PY{l+s+s1}{\PYZsq{}}\PY{p}{,} \PY{l+s+s1}{\PYZsq{}}\PY{l+s+s1}{population:Q}\PY{l+s+s1}{\PYZsq{}}\PY{p}{,} \PY{l+s+s1}{\PYZsq{}}\PY{l+s+s1}{density:Q}\PY{l+s+s1}{\PYZsq{}}\PY{p}{]}\PY{p}{,}
              \PY{n}{color}\PY{o}{=}\PY{n}{alt}\PY{o}{.}\PY{n}{Color}\PY{p}{(}\PY{l+s+s1}{\PYZsq{}}\PY{l+s+s1}{density:Q}\PY{l+s+s1}{\PYZsq{}}\PY{p}{,} \PY{n}{scale}\PY{o}{=}\PY{n}{alt}\PY{o}{.}\PY{n}{Scale}\PY{p}{(}\PY{n+nb}{type}\PY{o}{=}\PY{l+s+s1}{\PYZsq{}}\PY{l+s+s1}{pow}\PY{l+s+s1}{\PYZsq{}}\PY{p}{,} \PY{n}{scheme}\PY{o}{=}\PY{l+s+s1}{\PYZsq{}}\PY{l+s+s1}{viridis}\PY{l+s+s1}{\PYZsq{}}\PY{p}{,} \PY{n}{domain}\PY{o}{=}\PY{p}{[}\PY{l+m+mi}{0}\PY{p}{,}\PY{l+m+mi}{2000}\PY{p}{]}\PY{p}{)}\PY{p}{)}
          \PY{p}{)}
          
          \PY{n+nb}{map}
\end{Verbatim}

\texttt{\color{outcolor}Out[{\color{outcolor}130}]:}
    
    \begin{center}
    \adjustimage{max size={0.9\linewidth}{0.9\paperheight}}{output_37_0.png}
    \end{center}
    { \hspace*{\fill} \\}
    

    \begin{Verbatim}[commandchars=\\\{\}]
{\color{incolor}In [{\color{incolor}99}]:} \PY{c+c1}{\PYZsh{} Try different marks: mark\PYZus{}rect, mark\PYZus{}tick, mark\PYZus{}circle, …. Marks documentation}
         
         \PY{n+nb}{map} \PY{o}{=} \PY{n}{alt}\PY{o}{.}\PY{n}{Chart}\PY{p}{(}\PY{n}{france}\PY{p}{,} \PY{n}{width}\PY{o}{=}\PY{l+m+mi}{800}\PY{p}{,} \PY{n}{height}\PY{o}{=}\PY{l+m+mi}{800}\PY{p}{)}\PY{o}{.}\PY{n}{mark\PYZus{}circle}\PY{p}{(}\PY{n}{size}\PY{o}{=}\PY{l+m+mi}{100}\PY{p}{)}\PY{o}{.}\PY{n}{encode}\PY{p}{(}
             \PY{n}{x}\PY{o}{=}\PY{n}{alt}\PY{o}{.}\PY{n}{X}\PY{p}{(}\PY{l+s+s1}{\PYZsq{}}\PY{l+s+s1}{x:Q}\PY{l+s+s1}{\PYZsq{}}\PY{p}{,} \PY{n}{axis}\PY{o}{=}\PY{k+kc}{None}\PY{p}{)}\PY{p}{,}
             \PY{n}{y}\PY{o}{=}\PY{n}{alt}\PY{o}{.}\PY{n}{Y}\PY{p}{(}\PY{l+s+s1}{\PYZsq{}}\PY{l+s+s1}{y:Q}\PY{l+s+s1}{\PYZsq{}}\PY{p}{,} \PY{n}{axis}\PY{o}{=}\PY{k+kc}{None}\PY{p}{,} \PY{n}{scale}\PY{o}{=}\PY{n}{alt}\PY{o}{.}\PY{n}{Scale}\PY{p}{(}\PY{n}{zero}\PY{o}{=}\PY{k+kc}{False}\PY{p}{)}\PY{p}{)}\PY{p}{,}
             \PY{n}{size}\PY{o}{=}\PY{l+s+s1}{\PYZsq{}}\PY{l+s+s1}{population:Q}\PY{l+s+s1}{\PYZsq{}}\PY{p}{,}
             \PY{n}{color}\PY{o}{=}\PY{n}{alt}\PY{o}{.}\PY{n}{Color}\PY{p}{(}\PY{l+s+s1}{\PYZsq{}}\PY{l+s+s1}{density:Q}\PY{l+s+s1}{\PYZsq{}}\PY{p}{,} \PY{n}{scale}\PY{o}{=}\PY{n}{alt}\PY{o}{.}\PY{n}{Scale}\PY{p}{(}\PY{n}{scheme}\PY{o}{=}\PY{l+s+s1}{\PYZsq{}}\PY{l+s+s1}{viridis}\PY{l+s+s1}{\PYZsq{}}\PY{p}{,} \PY{n}{domain}\PY{o}{=}\PY{p}{[}\PY{l+m+mi}{0}\PY{p}{,}\PY{l+m+mi}{2000}\PY{p}{]}\PY{p}{)}\PY{p}{)}
         \PY{p}{)}
         
         \PY{n+nb}{map}
\end{Verbatim}

\texttt{\color{outcolor}Out[{\color{outcolor}99}]:}
    
    \begin{center}
    \adjustimage{max size={0.9\linewidth}{0.9\paperheight}}{output_38_0.png}
    \end{center}
    { \hspace*{\fill} \\}
    

    \begin{Verbatim}[commandchars=\\\{\}]
{\color{incolor}In [{\color{incolor}133}]:} \PY{n}{proj} \PY{o}{=} \PY{l+s+s1}{\PYZsq{}}\PY{l+s+s1}{orthographic}\PY{l+s+s1}{\PYZsq{}}
          \PY{n+nb}{map}\PY{o}{.}\PY{n}{project}\PY{p}{(}\PY{n}{proj}\PY{p}{)}
\end{Verbatim}

\texttt{\color{outcolor}Out[{\color{outcolor}133}]:}
    
    \begin{center}
    \adjustimage{max size={0.9\linewidth}{0.9\paperheight}}{output_39_0.png}
    \end{center}
    { \hspace*{\fill} \\}
    

    \begin{Verbatim}[commandchars=\\\{\}]
{\color{incolor}In [{\color{incolor}198}]:} \PY{c+c1}{\PYZsh{} Free Exploration}
          
          \PY{n+nb}{map} \PY{o}{=} \PY{n}{alt}\PY{o}{.}\PY{n}{Chart}\PY{p}{(}\PY{n}{france}\PY{p}{,} \PY{n}{width}\PY{o}{=}\PY{l+m+mi}{800}\PY{p}{,} \PY{n}{height}\PY{o}{=}\PY{l+m+mi}{800}\PY{p}{)}\PY{o}{.}\PY{n}{mark\PYZus{}circle}\PY{p}{(}\PY{n}{size}\PY{o}{=}\PY{l+m+mi}{500}\PY{p}{)}\PY{o}{.}\PY{n}{encode}\PY{p}{(}
              \PY{n}{x}\PY{o}{=}\PY{n}{alt}\PY{o}{.}\PY{n}{X}\PY{p}{(}\PY{l+s+s1}{\PYZsq{}}\PY{l+s+s1}{x:Q}\PY{l+s+s1}{\PYZsq{}}\PY{p}{,} \PY{n}{axis}\PY{o}{=}\PY{k+kc}{None}\PY{p}{,} \PY{n+nb}{bin}\PY{o}{=}\PY{n}{alt}\PY{o}{.}\PY{n}{Bin}\PY{p}{(}\PY{n}{maxbins}\PY{o}{=}\PY{l+m+mi}{60}\PY{p}{)}\PY{p}{)}\PY{p}{,}
              \PY{n}{y}\PY{o}{=}\PY{n}{alt}\PY{o}{.}\PY{n}{Y}\PY{p}{(}\PY{l+s+s1}{\PYZsq{}}\PY{l+s+s1}{y:Q}\PY{l+s+s1}{\PYZsq{}}\PY{p}{,} \PY{n}{axis}\PY{o}{=}\PY{k+kc}{None}\PY{p}{,} \PY{n}{scale}\PY{o}{=}\PY{n}{alt}\PY{o}{.}\PY{n}{Scale}\PY{p}{(}\PY{n}{zero}\PY{o}{=}\PY{k+kc}{False}\PY{p}{)}\PY{p}{,} \PY{n+nb}{bin}\PY{o}{=}\PY{n}{alt}\PY{o}{.}\PY{n}{Bin}\PY{p}{(}\PY{n}{maxbins}\PY{o}{=}\PY{l+m+mi}{60}\PY{p}{)}\PY{p}{)}\PY{p}{,}
              \PY{n}{color}\PY{o}{=}\PY{n}{alt}\PY{o}{.}\PY{n}{Color}\PY{p}{(}\PY{l+s+s1}{\PYZsq{}}\PY{l+s+s1}{population:Q}\PY{l+s+s1}{\PYZsq{}}\PY{p}{,} \PY{n}{aggregate}\PY{o}{=}\PY{l+s+s1}{\PYZsq{}}\PY{l+s+s1}{count}\PY{l+s+s1}{\PYZsq{}}\PY{p}{,} \PY{n}{scale}\PY{o}{=}\PY{n}{alt}\PY{o}{.}\PY{n}{Scale}\PY{p}{(}\PY{n}{scheme}\PY{o}{=}\PY{l+s+s1}{\PYZsq{}}\PY{l+s+s1}{oranges}\PY{l+s+s1}{\PYZsq{}}\PY{p}{)}\PY{p}{)}
          \PY{p}{)}
          
          \PY{n+nb}{map}
\end{Verbatim}

\texttt{\color{outcolor}Out[{\color{outcolor}198}]:}
    
    \begin{center}
    \adjustimage{max size={0.9\linewidth}{0.9\paperheight}}{output_40_0.png}
    \end{center}
    { \hspace*{\fill} \\}
    

    \subsection{Bonus exercises}\label{bonus-exercises}

    \begin{Verbatim}[commandchars=\\\{\}]
{\color{incolor}In [{\color{incolor}190}]:} \PY{c+c1}{\PYZsh{} Try making a heatmap of places in France}
          
          \PY{n+nb}{map} \PY{o}{=} \PY{n}{alt}\PY{o}{.}\PY{n}{Chart}\PY{p}{(}\PY{n}{france}\PY{p}{,} \PY{n}{width}\PY{o}{=}\PY{l+m+mi}{800}\PY{p}{,} \PY{n}{height}\PY{o}{=}\PY{l+m+mi}{800}\PY{p}{)}\PY{o}{.}\PY{n}{mark\PYZus{}square}\PY{p}{(}\PY{n}{size}\PY{o}{=}\PY{l+m+mi}{90}\PY{p}{)}\PY{o}{.}\PY{n}{encode}\PY{p}{(}
              \PY{n}{x}\PY{o}{=}\PY{n}{alt}\PY{o}{.}\PY{n}{X}\PY{p}{(}\PY{l+s+s1}{\PYZsq{}}\PY{l+s+s1}{x:Q}\PY{l+s+s1}{\PYZsq{}}\PY{p}{,} \PY{n}{axis}\PY{o}{=}\PY{k+kc}{None}\PY{p}{,} \PY{n+nb}{bin}\PY{o}{=}\PY{n}{alt}\PY{o}{.}\PY{n}{Bin}\PY{p}{(}\PY{n}{maxbins}\PY{o}{=}\PY{l+m+mi}{90}\PY{p}{)}\PY{p}{)}\PY{p}{,}
              \PY{n}{y}\PY{o}{=}\PY{n}{alt}\PY{o}{.}\PY{n}{Y}\PY{p}{(}\PY{l+s+s1}{\PYZsq{}}\PY{l+s+s1}{y:Q}\PY{l+s+s1}{\PYZsq{}}\PY{p}{,} \PY{n}{axis}\PY{o}{=}\PY{k+kc}{None}\PY{p}{,} \PY{n}{scale}\PY{o}{=}\PY{n}{alt}\PY{o}{.}\PY{n}{Scale}\PY{p}{(}\PY{n}{zero}\PY{o}{=}\PY{k+kc}{False}\PY{p}{)}\PY{p}{,} \PY{n+nb}{bin}\PY{o}{=}\PY{n}{alt}\PY{o}{.}\PY{n}{Bin}\PY{p}{(}\PY{n}{maxbins}\PY{o}{=}\PY{l+m+mi}{90}\PY{p}{)}\PY{p}{)}\PY{p}{,}
              
              \PY{n}{color}\PY{o}{=}\PY{n}{alt}\PY{o}{.}\PY{n}{Color}\PY{p}{(}\PY{l+s+s1}{\PYZsq{}}\PY{l+s+s1}{population:Q}\PY{l+s+s1}{\PYZsq{}}\PY{p}{,} \PY{n}{aggregate}\PY{o}{=}\PY{l+s+s1}{\PYZsq{}}\PY{l+s+s1}{count}\PY{l+s+s1}{\PYZsq{}}\PY{p}{,} \PY{n}{scale}\PY{o}{=}\PY{n}{alt}\PY{o}{.}\PY{n}{Scale}\PY{p}{(}\PY{n}{scheme}\PY{o}{=}\PY{l+s+s1}{\PYZsq{}}\PY{l+s+s1}{viridis}\PY{l+s+s1}{\PYZsq{}}\PY{p}{)}\PY{p}{)}
          \PY{p}{)}
          
          \PY{n+nb}{map}
\end{Verbatim}

\texttt{\color{outcolor}Out[{\color{outcolor}190}]:}
    
    \begin{center}
    \adjustimage{max size={0.9\linewidth}{0.9\paperheight}}{output_42_0.png}
    \end{center}
    { \hspace*{\fill} \\}
    

    \begin{Verbatim}[commandchars=\\\{\}]
{\color{incolor}In [{\color{incolor}201}]:} \PY{c+c1}{\PYZsh{} Or a heatmap of population densities in France}
          
          \PY{n+nb}{map} \PY{o}{=} \PY{n}{alt}\PY{o}{.}\PY{n}{Chart}\PY{p}{(}\PY{n}{france}\PY{p}{,} \PY{n}{width}\PY{o}{=}\PY{l+m+mi}{800}\PY{p}{,} \PY{n}{height}\PY{o}{=}\PY{l+m+mi}{800}\PY{p}{)}\PY{o}{.}\PY{n}{mark\PYZus{}square}\PY{p}{(}\PY{n}{size}\PY{o}{=}\PY{l+m+mi}{90}\PY{p}{)}\PY{o}{.}\PY{n}{encode}\PY{p}{(}
              \PY{n}{x}\PY{o}{=}\PY{n}{alt}\PY{o}{.}\PY{n}{X}\PY{p}{(}\PY{l+s+s1}{\PYZsq{}}\PY{l+s+s1}{x:Q}\PY{l+s+s1}{\PYZsq{}}\PY{p}{,} \PY{n}{axis}\PY{o}{=}\PY{k+kc}{None}\PY{p}{,} \PY{n+nb}{bin}\PY{o}{=}\PY{n}{alt}\PY{o}{.}\PY{n}{Bin}\PY{p}{(}\PY{n}{maxbins}\PY{o}{=}\PY{l+m+mi}{90}\PY{p}{)}\PY{p}{)}\PY{p}{,}
              \PY{n}{y}\PY{o}{=}\PY{n}{alt}\PY{o}{.}\PY{n}{Y}\PY{p}{(}\PY{l+s+s1}{\PYZsq{}}\PY{l+s+s1}{y:Q}\PY{l+s+s1}{\PYZsq{}}\PY{p}{,} \PY{n}{axis}\PY{o}{=}\PY{k+kc}{None}\PY{p}{,} \PY{n}{scale}\PY{o}{=}\PY{n}{alt}\PY{o}{.}\PY{n}{Scale}\PY{p}{(}\PY{n}{zero}\PY{o}{=}\PY{k+kc}{False}\PY{p}{)}\PY{p}{,} \PY{n+nb}{bin}\PY{o}{=}\PY{n}{alt}\PY{o}{.}\PY{n}{Bin}\PY{p}{(}\PY{n}{maxbins}\PY{o}{=}\PY{l+m+mi}{90}\PY{p}{)}\PY{p}{)}\PY{p}{,}
              
              \PY{n}{color}\PY{o}{=}\PY{n}{alt}\PY{o}{.}\PY{n}{Color}\PY{p}{(}\PY{l+s+s1}{\PYZsq{}}\PY{l+s+s1}{density:Q}\PY{l+s+s1}{\PYZsq{}}\PY{p}{,} \PY{n}{aggregate}\PY{o}{=}\PY{l+s+s1}{\PYZsq{}}\PY{l+s+s1}{mean}\PY{l+s+s1}{\PYZsq{}}\PY{p}{,} \PY{n}{scale}\PY{o}{=}\PY{n}{alt}\PY{o}{.}\PY{n}{Scale}\PY{p}{(}\PY{n}{scheme}\PY{o}{=}\PY{l+s+s1}{\PYZsq{}}\PY{l+s+s1}{oranges}\PY{l+s+s1}{\PYZsq{}}\PY{p}{)}\PY{p}{)}
          \PY{p}{)}
          
          \PY{n+nb}{map}
\end{Verbatim}

\texttt{\color{outcolor}Out[{\color{outcolor}201}]:}
    
    \begin{center}
    \adjustimage{max size={0.9\linewidth}{0.9\paperheight}}{output_43_0.png}
    \end{center}
    { \hspace*{\fill} \\}
    

    \begin{Verbatim}[commandchars=\\\{\}]
{\color{incolor}In [{\color{incolor}210}]:} \PY{n}{map1} \PY{o}{=} \PY{n}{alt}\PY{o}{.}\PY{n}{Chart}\PY{p}{(}\PY{n}{france}\PY{p}{,} \PY{n}{width}\PY{o}{=}\PY{l+m+mi}{300}\PY{p}{,} \PY{n}{height}\PY{o}{=}\PY{l+m+mi}{300}\PY{p}{)}\PY{o}{.}\PY{n}{mark\PYZus{}square}\PY{p}{(}\PY{n}{size}\PY{o}{=}\PY{l+m+mi}{90}\PY{p}{)}\PY{o}{.}\PY{n}{encode}\PY{p}{(}
              \PY{n}{x}\PY{o}{=}\PY{n}{alt}\PY{o}{.}\PY{n}{X}\PY{p}{(}\PY{l+s+s1}{\PYZsq{}}\PY{l+s+s1}{x:Q}\PY{l+s+s1}{\PYZsq{}}\PY{p}{,} \PY{n}{axis}\PY{o}{=}\PY{k+kc}{None}\PY{p}{,} \PY{n+nb}{bin}\PY{o}{=}\PY{n}{alt}\PY{o}{.}\PY{n}{Bin}\PY{p}{(}\PY{n}{maxbins}\PY{o}{=}\PY{l+m+mi}{90}\PY{p}{)}\PY{p}{)}\PY{p}{,}
              \PY{n}{y}\PY{o}{=}\PY{n}{alt}\PY{o}{.}\PY{n}{Y}\PY{p}{(}\PY{l+s+s1}{\PYZsq{}}\PY{l+s+s1}{y:Q}\PY{l+s+s1}{\PYZsq{}}\PY{p}{,} \PY{n}{axis}\PY{o}{=}\PY{k+kc}{None}\PY{p}{,} \PY{n}{scale}\PY{o}{=}\PY{n}{alt}\PY{o}{.}\PY{n}{Scale}\PY{p}{(}\PY{n}{zero}\PY{o}{=}\PY{k+kc}{False}\PY{p}{)}\PY{p}{,} \PY{n+nb}{bin}\PY{o}{=}\PY{n}{alt}\PY{o}{.}\PY{n}{Bin}\PY{p}{(}\PY{n}{maxbins}\PY{o}{=}\PY{l+m+mi}{90}\PY{p}{)}\PY{p}{)}\PY{p}{,}
              \PY{n}{color}\PY{o}{=}\PY{n}{alt}\PY{o}{.}\PY{n}{Color}\PY{p}{(}\PY{l+s+s1}{\PYZsq{}}\PY{l+s+s1}{population:Q}\PY{l+s+s1}{\PYZsq{}}\PY{p}{,} \PY{n}{aggregate}\PY{o}{=}\PY{l+s+s1}{\PYZsq{}}\PY{l+s+s1}{mean}\PY{l+s+s1}{\PYZsq{}}\PY{p}{,} \PY{n}{scale}\PY{o}{=}\PY{n}{alt}\PY{o}{.}\PY{n}{Scale}\PY{p}{(}\PY{n}{scheme}\PY{o}{=}\PY{l+s+s1}{\PYZsq{}}\PY{l+s+s1}{oranges}\PY{l+s+s1}{\PYZsq{}}\PY{p}{)}\PY{p}{)}
          \PY{p}{)}
          
          \PY{n}{map2} \PY{o}{=} \PY{n}{alt}\PY{o}{.}\PY{n}{Chart}\PY{p}{(}\PY{n}{france}\PY{p}{,} \PY{n}{width}\PY{o}{=}\PY{l+m+mi}{300}\PY{p}{,} \PY{n}{height}\PY{o}{=}\PY{l+m+mi}{300}\PY{p}{)}\PY{o}{.}\PY{n}{mark\PYZus{}square}\PY{p}{(}\PY{n}{size}\PY{o}{=}\PY{l+m+mi}{90}\PY{p}{)}\PY{o}{.}\PY{n}{encode}\PY{p}{(}
              \PY{n}{x}\PY{o}{=}\PY{n}{alt}\PY{o}{.}\PY{n}{X}\PY{p}{(}\PY{l+s+s1}{\PYZsq{}}\PY{l+s+s1}{x:Q}\PY{l+s+s1}{\PYZsq{}}\PY{p}{,} \PY{n}{axis}\PY{o}{=}\PY{k+kc}{None}\PY{p}{,} \PY{n+nb}{bin}\PY{o}{=}\PY{n}{alt}\PY{o}{.}\PY{n}{Bin}\PY{p}{(}\PY{n}{maxbins}\PY{o}{=}\PY{l+m+mi}{90}\PY{p}{)}\PY{p}{)}\PY{p}{,}
              \PY{n}{y}\PY{o}{=}\PY{n}{alt}\PY{o}{.}\PY{n}{Y}\PY{p}{(}\PY{l+s+s1}{\PYZsq{}}\PY{l+s+s1}{y:Q}\PY{l+s+s1}{\PYZsq{}}\PY{p}{,} \PY{n}{axis}\PY{o}{=}\PY{k+kc}{None}\PY{p}{,} \PY{n}{scale}\PY{o}{=}\PY{n}{alt}\PY{o}{.}\PY{n}{Scale}\PY{p}{(}\PY{n}{zero}\PY{o}{=}\PY{k+kc}{False}\PY{p}{)}\PY{p}{,} \PY{n+nb}{bin}\PY{o}{=}\PY{n}{alt}\PY{o}{.}\PY{n}{Bin}\PY{p}{(}\PY{n}{maxbins}\PY{o}{=}\PY{l+m+mi}{90}\PY{p}{)}\PY{p}{)}\PY{p}{,}
              \PY{n}{color}\PY{o}{=}\PY{n}{alt}\PY{o}{.}\PY{n}{Color}\PY{p}{(}\PY{l+s+s1}{\PYZsq{}}\PY{l+s+s1}{density:Q}\PY{l+s+s1}{\PYZsq{}}\PY{p}{,} \PY{n}{aggregate}\PY{o}{=}\PY{l+s+s1}{\PYZsq{}}\PY{l+s+s1}{mean}\PY{l+s+s1}{\PYZsq{}}\PY{p}{,} \PY{n}{scale}\PY{o}{=}\PY{n}{alt}\PY{o}{.}\PY{n}{Scale}\PY{p}{(}\PY{n}{scheme}\PY{o}{=}\PY{l+s+s1}{\PYZsq{}}\PY{l+s+s1}{oranges}\PY{l+s+s1}{\PYZsq{}}\PY{p}{)}\PY{p}{)}
          \PY{p}{)}
          
          \PY{n}{chart} \PY{o}{=} \PY{n}{alt}\PY{o}{.}\PY{n}{hconcat}\PY{p}{(}\PY{p}{)}
          \PY{k}{for} \PY{n}{origin} \PY{o+ow}{in} \PY{p}{[}\PY{n}{map1}\PY{p}{,} \PY{n}{map2}\PY{p}{]}\PY{p}{:}
              \PY{n}{chart} \PY{o}{|}\PY{o}{=} \PY{n}{origin}
          \PY{n}{chart}
\end{Verbatim}

\texttt{\color{outcolor}Out[{\color{outcolor}210}]:}
    
    \begin{center}
    \adjustimage{max size={0.9\linewidth}{0.9\paperheight}}{output_44_0.png}
    \end{center}
    { \hspace*{\fill} \\}
    


    % Add a bibliography block to the postdoc
    
    
    
    \end{document}
